\documentclass[a4paper,12pt,BCOR8.25mm,headsepline,final,twoside,bibtotoc,liststotoc,fleqn]{scrbook}

\usepackage[utf8]{inputenc}
\usepackage{amsfonts} %defines the \frak and \Bbb commands and set up the fonts msam (extra math symbols A), msbm (extra math symbols B, and blackboard bold), eufm (Euler Fraktur), extra sizes of cmmib (bold math italic and bold lowercase Greek), and cmbsy (bold math symbols and bold script), for use in mathematics.
\usepackage{amstext} %defines the amsmath \text command.
\usepackage{amssymb} %defines the names of all the math symbols available with the AMS fonts collection
\usepackage{amsbsy} % defines the amsmath \boldsymbol and (poor man’s bold) \pmb commands.
\usepackage{amscd} % defines some command for easing the generation of commutative diagrams.
\usepackage{amsmath} %\align, \subequation...

\hbadness=1000 % Gejammere ueber overfull/underfull boxes einstellbar (default=1000)

\usepackage{colortbl}%Tabellenfarben
\usepackage{ae,aecompl}
%\usepackage{hyphenat}
\definecolor{black}{rgb}{0,0,0}
\usepackage[plainpages=false, pdfpagelabels, colorlinks=true, linkcolor=black, menucolor=black, urlcolor=black, citecolor=black]{hyperref}


%\usepackage{subeqn} % fuer verschachtelt numerierte Gleichungen ist in amsmath enthalten
\usepackage{url}

% define the type of the thesis: Studienarbeit/Diplomarbeit
% (uncomment the appropriate type)
\newcommand{\thethesis}[0]{%
  Diplomarbeit
}

% find all usages of \field and replace the whole expression
\newcommand{\field}[1]{%
  {\itshape \{#1\}}
}





% macros to check whether running PDFLaTeX or not
\newif\ifpdf
\ifx\pdfoutput\undefined
\pdffalse % we are not running PDFLaTeX
\else
\pdfoutput=1 % we are running PDFLaTeX
\pdftrue
\pdfpkresolution 600
\pdfimageresolution 300
\pdfinfo{
   /Author (Johannes Held)
   /Title  (Die Zugrifsschicht der laufzeitadaptierten CoBRA DB)
   /Subject ()
   /Keywords (OSGi; Komponenten; Laufzeitadaption; Framework; DBMS; SOA)
}
\fi

% command to check draft option
\makeatletter
\newcommand*{\ifoptiondraft}{%
  \expandafter
  \@if@pti@ns\expandafter{\@classoptionslist}{final}%
  \@secondoftwo{%
    \expandafter
    \@if@pti@ns\expandafter{\@classoptionslist}{draft}%
    \@firstoftwo\@secondoftwo
  }%
}
\makeatother

\ifoptiondraft{%
  \usepackage[firsttwo,bottomafter]{draftcopy}
}

% include some useful packages
\usepackage{scrtime}                          % gain access to time stamps
\usepackage{scrpage2}                         % headers and footers
\usepackage{makeidx}                          % support for makeidx
\usepackage{array}                            % better table support
\usepackage{multicol}                         % spanning columns
\usepackage{multirow}                         % spanning rows
\usepackage{microtype}

\usepackage[printonlyused]{acronym}

% include right printer driver for graphicx
\ifpdf
  %\usepackage[pdftex]{graphicx}
  \usepackage{pdfpages} % lädt Paket graphics selbständig
  \pdfcompresslevel=9
\else
  \usepackage[dvips]{graphicx}
\fi

\usepackage{subfigure}
\usepackage[ngerman,english]{babel}           % switch language
\usepackage{float}                           
\usepackage[intoc]{nomencl}                          % list of abbreviations
\usepackage{algorithmic}                      % typesetting of algorithms
\usepackage[plain,chapter]{algorithm}         % typesetting of algorithms
\usepackage{stfloats}                         % used to have footnotes at bottom of the page
\usepackage[final]{listings}                  % typesetting of code listings 

\lstset{language=Java}
\lstset{basicstyle=\ttfamily\small\mdseries}
\definecolor{darkgrey}{rgb}{0.95,0.95,0.95}
\definecolor{darkgreen}{rgb}{0.3,0.6,0.3}
\definecolor{darkred}{rgb}{0.8,0.2,0.2}
\definecolor{darkblue}{rgb}{0.1,0.15,0.85}
%\lstset{backgroundcolor=\color{darkgrey}}
\lstset{stringstyle=\color{darkred}}
\lstset{numberstyle=\color{darkgreen}}
\lstset{commentstyle=\color{darkgreen}}
\lstset{keywordstyle=\color{darkblue}}
\lstset{linewidth=\textwidth, showstringspaces=false}
\lstset{captionpos=tb}
\lstset{tabsize=1}
\lstset{breaklines=true}
\lstset{frame=tlRb}
\lstset{frameround=fftt}
\lstset{breakatwhitespace=true}
\lstset{morekeywords={String,Class,Object}}
\lstset{numbers=left}
\lstset{float=htb}
\lstset{numberstyle=\ttfamily\tiny}
\lstset{numbersep=10pt}

% define command \missing
\newcommand{\missing}[1]{\,\,\textcolor{red}{(\marginpar[\hfill!$\longrightarrow$]{$\longleftarrow$!}{\bfseries 
    Missing:}\,\emph{#1})}\,\,}
\newcommand{\note}[1]{\marginpar[#1]{#1}}
\newcommand{\code}[1]{\texttt{#1}}

% environment to typeset sub-figures
\newbox\subfigbox
\makeatletter
        \newenvironment{subfloat}
                {\def\caption##1{\gdef\subcapsave{\relax##1}}%
                 \let\subcapsave\@empty
                 \setbox\subfigbox\hbox
                         \bgroup}
                  {\egroup
                 \subfigure[\subcapsave]{\box\subfigbox}}
\makeatother

% list of abbreviations
\let\abbrev\nomenclature
\renewcommand{\nomname}{Liste der Abkürzungen}
\setlength{\nomlabelwidth}{.25\hsize}
\renewcommand{\nomlabel}[1]{#1 \dotfill}
\setlength{\nomitemsep}{-\parsep}
\makeglossary
\newcommand{\markup}[1]{\textbf{#1}}

% define new style for TOC
\makeatletter
\renewcommand{\numberline}[1]{%
        \makebox[0.9cm][l]{#1}\hspace{1mm}}
\renewcommand{\l@chapter}[2]{%
        \addvspace{2ex}%
        \pagebreak[3]%
        \noindent%
        \makebox[0pt][l]{%
        \rule[-3pt]{\textwidth}{0pt}}%
        {\large\textsf{\textbf{#1}}}\hfill#2%
        \par%
        \nopagebreak%
        \addvspace{1ex}%
}
\renewcommand{\l@section}[2]{%
        \addvspace{0.5ex}%
        \noindent\hspace{1cm}%
        #1\dotfill#2%
        \par%
        \nopagebreak[2]%
}
\renewcommand{\l@subsection}[2]{%
        \addvspace{0.2ex}%
        \noindent\hspace{2cm}%
        #1\dotfill#2%
        \par%
}       
\makeatother

% define new style for index 
\makeatletter
\newcommand*{\heading}[1]{%
        \makebox[0pt][l]{%
                \rule[-3pt]{\linewidth}{0pt}}%
        \textsf{\textbf{\Large #1}}\hfill\nopagebreak\vspace{4pt}}
\renewenvironment{theindex}{%
        \setlength{\columnseprule}{0.4pt}
        \setlength{\columnsep}{2em}
        \begin{multicols}{2}[\chapter*{\indexname}]
                \parindent\z@
                \parskip\z@ \@plus .3\p@\relax
                \let\item\@idxitem}%
        {\end{multicols}\clearpage}
\makeatother

% page break
\clubpenalty = 10000
\widowpenalty = 10000

% prepare generation of index
\makeindex

% put footnotes below floats at the bottom
\fnbelowfloat

\setlength{\parindent}{0em}


\hyphenation{Ei-gen-schaf-ten}

\includeonly{
%intro/titlepage,
%intro/rechtsbehelf,
%intro/abstract,
%main/einleitung,
main/grundlagen,
main/evaluation_p2p,
%main/konzeption_pubsub,
%main/related_work,
%main/zusammenfassung,
%main/anhang
}

\begin{document}


\frontmatter
\pagenumbering{alph}

\ifpdf
  \includepdf[pages={1,{}}]{intro/Deckblatt.pdf}
\fi

\begin{titlepage}
  
  \begin{center}
    
    {\Huge \bf
Entwicklung eines Frameworks für die Verteilungsoptimierung in Publish/Subscribe Systemen auf Basis eines strukturierten\\[0.4cm] P2P-Overlay Netzwerks} 
    
    \vspace*{1cm}
    Diplomarbeit im Fach Informatik
    \vspace{2cm}
    
    {\large vorgelegt von} \\
    \vspace*{0.7cm}
    {\Large \bf Johannes Held} \\
    \vspace*{0.7cm}
    {\large geb. 26.05.1983 in Nürnberg} 
    
    \vspace{1cm}
    
    angefertigt am 

    \vspace{1cm}
    
    {\bf 
      Institut für Informatik \\
      Lehrstuhl für Informatik 6\\
      Datenmanagement \\
      Friedrich-Alexander-Universität Erlangen--Nürnberg \\
      (Prof. Dr. Klaus Meyer-Wegener)
      }
    
    \vspace{1cm}
\end{center}
\begin{tabbing}
    Betreuer: \= Univ.-Prof. Dr. Richard Lenz \\
    \> Dipl.-Inf. Thomas Fischer 
\end{tabbing}
    \vspace{1cm}
    
    
\begin{tabbing}
	Trallalalalalalalalalalla \= \kill
	Beginn der Arbeit:   \> 01.06.2010 \\
  Abgabe der Arbeit:   \> 01.12.2010
\end{tabbing}
    
  
\end{titlepage}

\clearpage{\pagestyle{empty}\cleardoublepage}


\thispagestyle{empty}
\selectlanguage{ngerman}
Ich versichere, dass ich die Arbeit ohne fremde Hilfe und ohne Benutzung anderer als der angegebenen Quellen angefertigt habe und dass die Arbeit in gleicher oder ähnlicher Form noch keiner anderen Prüfungsbehörde vorgelegen hat und von dieser als Teil einer Prüfungsleistung angenommen wurde. Alle Ausführungen, die wörtlich oder sinngemäß übernommen wurden, sind als solche gekennzeichnet.
\vspace{2cm}

Der Universität Erlangen-Nürnberg, vertreten durch die Informatik 6 (Datenmanagement), wird für Zwecke der Forschung und Lehre ein einfaches, kostenloses, zeitlich und örtlich unbeschränktes Nutzungsrecht an den Arbeitsergebnissen der \thethesis einschließlich etwaiger Schutzrechte und Urheberrechte eingeräumt.

\vspace{2cm}
Erlangen, den 25.09.2010

\vspace{2cm}
Johannes Held \hfill \ 

\vspace{0,5cm}

\clearpage{\pagestyle{empty}\cleardoublepage}


\selectlanguage{ngerman}

\pagestyle{useheadings}

\setlength{\parskip}{0.7em}
\pagenumbering{Roman}
\chapter*{Abstract}
\section*{DA}
\selectlanguage{english}


\selectlanguage{ngerman} 
\clearpage{\pagestyle{empty}\cleardoublepage}
\chapter*{Kurzfassung}
\section*{DA}


\selectlanguage{ngerman}
%\clearpage{\pagestyle{plain}\cleardoublepage}

\tableofcontents
\listoffigures
%\clearpage{\pagestyle{plain}\cleardoublepage}
\listoftables
%\clearpage{\pagestyle{plain}\cleardoublepage}
\lstlistoflistings
%\listofalgorithms

%\clearpage{\pagestyle{plain}\cleardoublepage}
\chapter*{Abkürzungsverzeichnis}

\pagestyle{useheadings}
\markleft{Abkürzungsverzeichnis}
\markright{Abkürzungsverzeichnis}
\addcontentsline{toc}{chapter}{Abkürzungsverzeichnis}

\vspace{\topskip}


\begin{acronym}[xxxxxxxxxxxx]
	\setlength{\itemsep}{-\parsep}
	\setlength{\itemindent}{1.5em}
%A
	\acro{aoi}[AOI]{\emph{Area of Interest}}
	\acro{api}[API]{\emph{Application Programming Interface}}
%B

%C	
%	\vspace{\parsep} 
 \acro{cast}[CAST]{\emph{group anycast/multicast}}

%D	
%	\vspace{\parsep}
	 \acro{dht}[DHT]{\emph{Distributed Hashtable}}
	\acro{dolr}[DOLR]{\emph{Decentralized Object Location and Routing}}
%E	
%	\vspace{\parsep}
%	\acro{er}[ER-Diagramm]{Entity-Relationship-Diagramm}
%F

%G	
%	\vspace{\parsep}

%H

%I	
%	\vspace{\parsep}
%	\acro{iso}[ISO]{International Organization for Standardization}
	
%J	
%	\vspace{\parsep}
	
%K
%\vspace{\parsep}
\acro{kbr}[KBR]{\emph{key-based Routing}}



%L

%M
%	\vspace{\parsep}
	\acro{m2etis}[M$^2$etis]{\emph{\textbf{M}assive \textbf{M}ultiuser \textbf{E}ven\textbf{t} \textbf{I}nfra\textbf{S}tructure}}	
	\acro{mmog}[MMOG]{\emph{Massive Multiplayer Online Game}}
	\acro{mmve}[MMVE]{\emph{Massive Multiuser Virtual Environment}}

%N
%	\vspace{\parsep}
	\acro{npc}[NPC]{\emph{non player character}}


%O	
	
%P	
%\vspace{\parsep}
	\acro{p2p}[p2p]{\emph{Peer--to--Peer}}

%Q
%	\vspace{\parsep}
	

%R	
%	\vspace{\parsep}
	
%S	
%	\vspace{\parsep}
	\acro{stl}[STL]{\emph{C++ Standard Template Library}}
%T	
%	\vspace{\parsep}
	\acro{tmp}[TMP]{\emph{Template Meta-Programmierung}}
%U

%V
%	\vspace{\parsep}
	\acro{vast}[VAST]{\emph{Voronoi-based Adaptive Scalable Transfer}}
	\acro{von}[VON]{\emph{Voronoi-based Overlay Network}}

%W

%X	
%	\vspace{\parsep}

%Y

%Z

\end{acronym}



\mainmatter

% CONTENT START

\chapter{Motivation}
\label{chap:einleitung}
Die Computerspiele sind inzwischen Teil des Miteinanders und durchaus weit verbreitet in sozialen Netwerken. Computerspiele veränderten sich, dank der fortschreitenden Übertragungs\-techniken, von Einzelspieler-Spielen beziehungsweise rundenbasierten Gruppenspielen an einem PC zu im Netzwerk gespielten \acp{mmog}. Dies bedeutet, dass Spieler gemeinsam in einer virtuellen Welt, einem sogenannten \ac{mmve}, interagieren. \acp{mmog} verlangen häufig eine monatliche Gebühr, somit ergeben sich gewissen Ansprüche an diese. Die virtuelle Welt muss konsistent, für alle Spieler gleich, sein. Zusätzlich zur dauerhaften Verfügbarkeit, meist 24/7, um Spieler verstreut über den ganzen Erdball zu bedienen, müssen bestimmte Aktionen eines Spielers gespeichert werden, damit dieser beim erneuten Eintreten in die virtuelle Welt den alten Zustand seines Avatars wiederfindet \cite{Zhang2008Persistence}. Weiterhin soll ein \ac{mmog} -- ja nach Umsetzung -- vielfältige Arten von Interaktionen zwischen Spielern, Spielern und \acp{npc} erlauben. Bei SecondLife\footnote{\url{http://www.secondlife.com}} steht der Aspekt der Interaktion zwischen Spielern im Vordergrund, während bei Guildwars\footnote{\url{http://www.guildwars.com}} das Lösen von Quests innerhalb der virtuellen Welt und das Verbessern des eigenen Avatars das Spiel bestimmt.
Die Kommunikation in \acp{mmve} wird meist über eine Client/Server-Architektur abgewickelt. Konsistenz wird einfacherweise damit erreicht, dass der Server die Nachrichten per Broadcast an alle Spieler verteilt. Bei $n$ angemeldeten Spielern (im Folgenden auch \emph{Knoten} genannt) in der virtuellen Welt hat dies eine Komplexität von $O(n^2)$. Sicherheit und Dauerhaftigkeit der Welt sind durch den zentralen Knotenpunkt ebenfalls gegeben. Es ist offensichtlich, dass solch ein System nicht beliebig skaliert, wie ein kleines Rechenbeispiel veranschaulicht: Angenommen eine Updatenachricht ist 10 Bytes groß und der Server ist mit 100 mbit/s (12500 kB/s) angebunden. Wenn die Framerate des Spiels auf 30 Hz, das heißt 30 Updates/s, festgelegt ist, sendet jeder Knoten 300 Bytes/s (0,3 kB/s) an den Server, die dieser an $n-1$ Knoten verteilen muss. Die Bandbreite des Servers ist daher mit lediglich $\lfloor\sqrt{12500\,kB/s \div 0,3\,kB/s}\rfloor = 204$ Knoten gesättigt.
Daher sind viele \acp{mmog} in sogenannte \emph{Shards} aufgeteilt. Dies bedeutet, dass es auf verschiedenen Servern verschiedene Instanzen einer virtuellen Welt gibt, die untereinander nicht verbunden sind. Dies schmälert durchaus das Spielvergnügen, denn die Instanzen können auseinanderlaufen, wenn der Betreiber Updates und Addons nicht auf allen Servern simultan einspielt. Weiterhin kann die Aufteilung des Freundeskreis auf verschiedene Server von neuen Spielern als Hindernis angesehen werden: Es ist kaum möglich -- unter normalen Umständen -- mehrere Avatare gleichmäßig zu bespielen. Um diese Trennung in verschiedene Welten aufzubrechen und die Anzahl der vorzuhaltenden Server zu verringern, gibt es Bestrebungen, dezentrale Techniken wie \ac{p2p}-Netzwerke zu nutzen \cite{Knutsson2004Peertopeer, Triebel2008Peertopeer}. Dies bedeutet, dass allein die Knoten im System die Welt untereinander konsistent halten müssen. Zusätzlich ergeben sich aus der dezentralen Verwaltung neue Herausforderungen in Bezug auf betrügerisches Verhalten \cite{Kabus2007Design}. Einige Ansätze wie Donnybrook \cite{Bharambe2008Donnybrook} oder VAST \cite{Backhaus2007Voronoibased}, die Optimierungen auf Basis des \ac{p2p}-Konzeptes durchführen, nutzen einzelne Aspekte des Spiels zur Optimierung der Eventverteilung aus. Donnybrook nimmt hier starken Bezug auf die Sichtbarkeit von anderen Avataren, während VAST die Nachbarschaft eines Avatars bezüglich dessen \ac{aoi} betrachtet und die Verbindungen des unterliegenden Netzwerkes damit abgleicht. Statt das System lediglich auf die Behandlung eines Eventtypen hin zu optimieren, strebt im Gegensatz dazu \ac{m2etis} eine Optimierung aller im Spiel genutzten Eventtypen an. Auch Cugola sieht Publish/Subscribe-Systeme als ein geeignetes Kommunikationssystem für verteilte Knoten \cite{Cugola2002Using}.

\missing{Jetzt noch die Kurve zu PubSub kriegen und schöner beschreiben \emph{was mein Werk ist}!}


In dieser Arbeit werden die erste Komponente von \ac{m2etis} entwickelt: Das grundlegende Framework des zur Übersetzungszeit optimierbare Publish/Subscribe-System sowie dessen Kommunikation mit dem \ac{p2p}-Netzwerk. Die Verbindung aus \ac{p2p}-Netzwerk und Publish/Subscribe-System wird in \ac{m2etis} aufgriffen und ermöglicht neue Dimensionen der Verteilungsoptimierung von Events. Dies soll eine Entlastung des Kommunikationssystems bringen und somit ermöglichen skalierbare \acp{mmve} zu erstellen.

\section*{Aufbau der Arbeit}
Ein einfaches Szenario zeigt typische Spielsituationen und verdeutlicht die Ansätze von \ac{m2etis}. Danach wird in die Grundlagen zu Overlay- und p2p-Netzwerken sowie verteilten Publish/Subscribe-Systeme eingeführt. Das nächste Kapitel dieser Arbeit widmet sich der Beschreibung der drei Overlay-Netzwerke Chord, Pastry/Tapestry und CAN und der Auswahl eines für \ac{m2etis} geeigneten Netzwerks. Das Kapitel über die Konzeption des Frameworks zeigt die Umsetzung der in  Dimensionen zur Klassifizierung von Eventtypen sowie deren Reihenfolge und Zusammenspiel im Verarbeitungsmodell. Durch Einblicke in die prototypische Implementierung werden die Abstraktionsebenen des Frameworks und dessen einfache Handhabbarkeit ersichtlich. Die Einführung in \ac{tmp} stellt zudem die Techniken vor die den Overhead zur Laufzeit reduzieren und findet sich im Anhang dieser Arbeit.

\chapter{Grundlagen}
\label{chap:grundlagen}

\section{Begriffserklärungen}
In dieser Arbeit wird ein \emph{Publish/Subscribe-System über einem \ac{p2p} overlay Netzwerk} beschrieben.

\missing{FUFUFU}

\paragraph{Overlay Netzwerk} Ein Overlay-Netzwerk\index{Netzwerk!Overlay} ist ein logischer Aufsatz auf einem bestehenden Netzwerk. Weiterhin zeichnet es sich dadurch aus, dass ein eigener Adressraum genutzt wird und somit das unterliegende Netzwerk überlagert wird. Knoten im Netzwerk können demnach benachbart sein, ohne eine direkte physikalische Verbindung zu haben.\\
Ein Overlay-Netzwerk bietet neben dem eigenen Adressraum auch Funktionalität zum Versand, Empfang und Routing von Nachrichten \cite{Tannenbaum2003}. 

\paragraph{\ac{p2p}-Netzwerk} p2p-Netzwerke\index{Netzwerk!p2p} beschreiben den Verbund von Knoten, sogenannten Peers, die miteinander gleichberechtigt kommunizieren können \cite{Steinmetz2005}. p2p-Netzwerke werden meist durch Overlay-Netzwerke realisiert, da Techniken wie beispielsweise IP-Multicast \cite{Deering1990Multicast} nicht weit verbreitet sind und diese vom unterliegenden Netzwerk abstrahieren. Durch deren angebotene Funktionen können viele Arten von p2p-Systemen implementiert werden. Grundlagen dieser Systeme sowie deren unterschiedliche Ansätze werden in \Fref{chap:grundlagen:p2p} näher besprochen.\\
Meist wird unter p2p \emph{Filesharing} verstanden, was aber nicht immer der Fall sein muss. Geeignete p2p-Netzwerke können unter Anderem auch als Kommunikationsnetze genutzt werden \cite{Darlagiannis2006Peertopeer}.

Im Folgenden werden die Begriffe \emph{p2p-Netzwerk} und \emph{Overlay-Netzwerk} in dieser Arbeit synonym benutzt. In vielen zitierten Arbeiten wird von \emph{p2p-overlay network} gesprochen und dies drückt die Anforderung dieser Arbeit an solche Netzwerke aus:\\
Ein vom physischen Netzwerk (räumlich) unabhängiger Verbund aus gleichberechtigten Knoten die miteinander kommunizieren können.

\paragraph{Publish/Subscribe-System} Ein Publish/Subscribe-System\index{Publish/Subscribe} beschreibt ein Abosystem. Ein Subscriber schreibt sich bei einem Publisher ein und wird von diesem über Veränderungen benachrichtigt. Solch ein System kann kanalbasiert oder filterbasiert sein. In \Fref{chap:grundlagen:pubsub} sind diese Systeme weitaus genauer aufgeführt. So werden aktuelle Systeme gegen die Anforderungen dieser Arbeit geprüft und daraus Konzepte für das zu entwickelnde System gezogen.

\missing{MÄH, HÄSSLICH!!}

\missing{Irgendein Zitat (Buch)!}


\section{p2p-Netzwerke}
\label{chap:grundlagen:p2p}

p2p-Netzwerke sind die Grundlage aktueller Filesharing-Systeme wie BitTorrent. Der einfache Netzaufbau, die Fehlertoleranz bei ausfallenden Knoten sowie der Wegfall eines Servers lässt diese Netzwerke auch für Computerspiele interessant werden \cite{Knutsson2004Peertopeer, Triebel2008Peertopeer}. Roussopoulos stellt in \cite{Roussopoulos20032} einen Entscheidungsbaum für und wider den Einsatz von p2p-Netzwerken zur Verfügung. Auch für Sensornetzwerke ist der Aufbau eines p2p-Netzwerkes von Vorteil. Es können wertvolle Ressourcen (z.B. Batterie) besser genutzt werden \cite{MuneebAliandKoenLangendoen2007Case, Sioutas2009Building}, da die Kommunikation zwischen den Knoten weniger Sendeleistung benötigt als es eine Verbindungen zu einem -- eventuell weit entfernten -- Hauptrechner benötigen würde. Solche Netzwerke können beispielsweise zur Erkennung von Waldbränden genutzt werden. Sensorknoten werden über dem zu überwachenden Gebiet abgeworfen und sammeln Daten wie Temperatur und Luftfeuchtigkeit. Verbunden in einem großen p2p-Netzwerk werden die Daten untereinander ausgetauscht. Zur dezentralen Verwaltung und Abfrage genügt es an einem Sensorknoten die Daten abzufragen.

p2p-Netzwerke lassen sich grundsätzlich als \emph{unstrukturiert} oder \emph{strukturiert} klassifizieren \cite{Steinmetz2005, Lua2005Survey} und in Generationen einteilen \cite{Bo2003PeertoPeer}:
\begin{itemize*}
	\item \emph{1G} unstrukturierte Netzwerke
	\item \emph{2G} strukturierte Netzwerke
	\item \emph{3G} strukturierte Netzwerke mit Fokus auf Anonymität, Authentifizierung, Schutz vor Zensur und verschiedenen Rollen der einzelnen Knoten
\end{itemize*}

Netzwerke der dritten Generation, wie z.B. GNUnet \cite{Bennett2002GNet} werden in dieser Arbeit nicht näher betrachtet. Die zusätzlich angebotene Funktionalität wird nicht benötigt und würde ob ihrer Komplexität das zu entwickelnde System aufblähen. Die Beschreibung widmet sich im Folgenden den Netzwerken der ersten und zweiten Generation.

\subsection{Unstrukturierte Netzwerke}
Unstrukturierte Netzwerke\index{Netzwerk!Overlay!unstrukturiert} zeichnen sich dadurch aus, dass alle Informationen und Dateien durch Suchalgorithmen \cite{Lv2002} gefunden werden müssen. \\
Ein Client tritt dem Netzwerk bei und stellt seine Suchanfrage in das Netz. Werden entsprechende Peers gefunden die diese Suchanfrage beantworten können, werden zum Transfer Direktverbindungen aufgebaut. Aufgrund der einfachen (meist textbasierten) Struktur der Suchanfragen können diese einen großen Wertebereich abdecken.

Im Wesentlichen lassen sich unstrukturierte Netzwerke in die Typen \emph{zentralisiert} und \emph{dezentralisiert} einteilen.

\paragraph{zentralisiert} Knoten im Netz melden eine Liste der verfügbaren Dateien an bekannte Hauptrechner. Suchanfragen werden ebenfalls an diese Rechner gerichtet. Der Suchende erhält eine Liste von potentiellen Peers, die Dateien seiner Suchanfrage entsprechend anbieten. Diese Dateien werden über direkte Verbindungen zwischen den Peers übertragen. Einige Systeme tauschen weitere potentielle Knoten über Peers aus. Die einzelnen Hauptrechner stellen bei dieser Art von System einen \emph{single point of failure} dar \cite{Eberspaecher2005}.

\paragraph{dezentralisiert} In dezentralen Netzen gibt es keine bekannten Hauptrechner. Damit sich ein neuer Knoten in das Netz einklinken kann, muss diesem mindestens ein bestehender Knoten im Netzwerk bekannt sein. Über diesen Peer tauscht der neue Knoten Informationen aus und baut eine Nachbarschaft auf. Damit Dateien gefunden werden können, wird die Suchanfrage an alle Nachbarn geschickt, sprich die Suchanfrage wird durch das Netzwerk geflutet. Diese senden sie weiter an ihre Nachbarn.\\
Die Suchanfrage kann z.B. mit einer Anzahl an Hops oder einer TTL\footnote{Time to live} in ihrer Reichweite eingegrenzt werden. Potentielle Zyklen müssen bei dieser Art von Suche aufgelöst werden \cite{Lv2002}. Sind Peers gefunden, wird ebenfalls eine direkte Verbindung zur Datenübertragung aufgebaut. 

Beispiele für solche Netzwerke sind Napster (zentralisiert), Gnutella oder BitTorrent\footnote{\url{http://www.bittorrent.com/}} (dezentralisiert).

\subsection{Strukturierte Netzwerke}
Strukturierte Netzwerke\index{Netzwerk!Overlay!strukturiert} der zweiten Generation sind oft dezentraler Natur. Ein Datensatz muss nicht gesucht werden, da anhand der Struktur des Netzes die zuständigen Knoten berechnet werden können. Daten können ebenfalls via Direktverbindung übertragen werden, meist wird jedoch das Netzwerk selbst zum Versand genutzt. Hierbei werden die in Nachrichten gepackte Datensätze über verschiedene Peers geroutet. Der dezentralen Art dieser Netze ist geschuldet, dass ein neuer Knoten mindestens einen Peer aus dem Netzwerk kennen muss. Viele Systeme gehen davon aus, dass der neue Knoten aus einer Liste von Peers, den ihm nächst gelegenem Knoten\footnote{Nähe im Sinne von Latenz beziehungsweise räumlicher Nähe} wählen kann und über diesen den Eintritt in das Netz anstößt.

Strukturierte Netzwerke nutzen die Technik der \ac{dht}, da diese als Grundlage einer Vielzahl von unterschiedlichen Anwendungen dienen kann \cite{Wehrle2005, Ghodsi2006AlgorithmsDHT}. Grundsätzlich ist jedem Knoten ein Schlüssel aus dem Schlüsselraum des Netzwerkes zugeordnet. Ein Knoten ist dabei - je nach genutzter Metrik - für einen Bereich aus dem Schlüsselraum zuständig. Für das zu suchende Datum wird nun der Hashwert berechnet - dieser entspricht einem Schüssel. Der suchende Knoten kennt damit den zuständigen Peer und nutzt das zugrunde liegende Overlay-Netzwerk zur Datenübertragung.\\
Hier wird deutlich, dass diese Netzwerke eher der Kommunikation diesen als dem Austausch von Dateien.

Darauf aufbauend gibt es unterschiedliche Systeme die sich hinsichtlich Organisation, Routing, Ein- und Austritt von Knoten und dem Verhalten im Fehlerfall unterscheiden \cite{Goetz2005, Lua2005Survey}.

Beispiele für solche Netzwerke sind Chord \cite{Hosseini2007Survey}, Pastry \cite{Rowstron2001}, Tapestry \cite{Zhao2001Tapestry,Zhao2004Tapestry} oder CAN \cite{Ratnasamy2001Scalable}. Auf diese wird im \Fref[plain]{chap:evaluation_p2p}, das sich der Evaluation widmet, näher eingegangen.


\section{Verteilte Publish/Subscribe-Systeme}
\label{chap:grundlagen:pubsub}
Konzeptionell lassen sich Publish/Subscribe-Systeme als eventbasierte Systeme betrachten. Auf Grund ihres Aufbaus und der Skalierung  in orthogonalen Dimensionen\index{Publish/Subscribe!orthogonale Dimensionen} \enquote{Raum}, \enquote{Zeit} sowie \enquote{Synchronisation} eignen sich diese gut zur Verteilung von Events in dezentralen Umgebungen \cite{PatrickTh2003Many, Cugola2002Using}.

\begin{figure}[htbp]
\centering
\includegraphics{grafics/pubsub_black_box.pdf}
\caption{Schema eines Publish/Subscribe-Systems}
\label{fig:pubsub_black_box}
\end{figure}

Publisher und Subscriber werden durch das Event-System voneinander getrennt, wie es in \Fref{fig:pubsub_black_box} dargestellt ist.  Publisher und Subscriber sind räumlich voneinander getrennt. Ein Publisher übergibt die Nachricht an das System und hält weder direkte Verbindung mit den Subscribern noch muss der Publisher alle Subscriber kennen. Diese Trennung bezieht sich nicht nur auf verschiedene Komponenten einer Applikation, sondern kann auch über Applikations- oder gar Rechnergrenzen gehen. Die zeitliche Trennung beschreibt, dass sich ein Subscriber am System anmelden kann, obwohl kein Publisher vorhanden ist, analog können Nachrichten publiziert werden, ohne dass Empfänger eingeschrieben sind. Je nach Implementierung können Nachrichten zwischengespeichert werden, um diese neuen Subscribern zuzustellen. Bei einem Fernaufrufsystem wie \emph{remote proceduce call (RPC)} \cite{Birrell1984Implementing} ist dies nicht möglich, da die Gegenseite existieren muss. Das Senden einer Nachricht ist für den Publisher nicht blockierend und Subscriber warten zudem nicht aktiv auf neue Nachrichten, sondern werden meist per Callback über neue Nachrichten informiert. Damit wird die Verarbeitung vom Event-System aus nebenläufig getriggert.

Diese Arbeit beschäftigt sich ausschließlich mit dezentralen Publish/Subscribe-Sys\-temen, denn \ac{m2etis} zielt darauf ab, die Rechner der Nutzer in einem p2p-Netzwerk zu verbinden und darauf aufbauend die Events zu verteilen. Viele der Grundlagen in diesem Kapitel gelten sowohl für klassische zentrale als auch dezentrale Publish/Subscribe-Systeme, allerdings müssen im verteilten Fall die Verwaltungsinformationen ebenfalls dezentral auf allen Knoten gespeichert, beziehungsweise geeignete Verteilungsalgorithmen gefunden werden. Somit relativiert sich die Dimension der räumlichen Trennung, da Publisher wie Subscriber Teil des Eventsystems sind.

Banerjee vergleicht verschiedene Arten zum Aufbau solch eines Multicast-Systemes. \enquote{mesh-first} beschreibt den expliziten Aufbau des Netzwerkes. Die Peers verändern ihre Verbindungen aufgrund bestimmter Metriken und können auch Netzwerkpartitionen beheben und sind somit selbst für das Netzwerk zuständig. Der \enquote{implizte} Ansatz beschreibt Publish/Subscribe-Systeme, die auf einem Overlaynetzwerk aufsetzen und dessen Routingalgorithmus indirekt die Verteilungsstruktur bestimmt \cite{Banerjee2001Comparative}. Ein Beispiel hierfür ist Scribe, das in \Fref{chap:related:scribe} beschrieben wird.

Fiege befasst sich näher mit dem Aspekt der Sicherheit und des Vertrauens zwischen Sender, Empfänger und dem Verteilungsystem \cite{FiegeSecurity}. Behnel stellt verschiedene Aspekte von \enquote{Quality of Service} auf verschiedenen Ebenen eines Publish/Sub\-scribe-Systems vor. Beispielsweise \enquote{Latenz}, \enquote{Bandbreite}, \enquote{Zustellgarantien} für Nachrichten auf Netzwerkebene oder \enquote{Reihenfolge}, \enquote{Validität} oder \enquote{Authentifizierung} von Nachrichten auf Verteilungsebene. Er beschreibt das Verhalten einiger Publish/Subscribe-Systeme hinsichtlich der beschriebenen Aspekte \cite{BeFiMu2006PubSubQoS}. 

Grundsätzlich lassen sich Publish/Subscribe-Systeme in zwei Varianten einteilen: \emph{kanalbasiert}\index{Publish/Subscribe!kanalbasiert} und \emph{filterbasiert}\index{Publish/Subscribe!filterbasiert} \cite{Liu2003Survey}. In kanalbasierten Systemen werden die Nachrichten einzelnen Kategorien zugeordnet. Subscriber können sich für Nachrichten dieser Kategorien anmelden und bekommen diese zugestellt. Filterbasierte Systeme haben diese Einteilung nicht, stattdessen sind Nachrichten typisiert (zum Beispiel nur einfache Datentypen und Zeichenketten) und mit einem Wertebereich versehen. Bei der Anmeldung kann ein Prädikat zur Filterung angegeben werden. Der Knoten empfängt nun nur gefilterte, auf das Prädikat passende Nachrichten.

Verbindet man die Filterung von Nachrichten mit einem kanalbasierten Ansatz, gelangt man zu einem \emph{hybriden} System\index{Publish/Subscribe!hybrid}: Einer Anmeldung an einem Kanal kann ein Prädikat übergeben werden. Beispielsweise wird eine Anmeldung am Kanal für Bewegungsnachrichten auf ein Gebiet eingeschränkt. Die dezentrale Filterung ist jedoch nur möglich, wenn die Nutzdaten vom System lesbar oder mit filterbaren Metainformationen angereichert sind. Zudem müssen die Prädikate im logisch aufgebauten Verteilungssystem bekanntgemacht werden, damit Nachrichten frühzeitig bei der Verteilung gefiltert werden können. Beispielsweise kann der Eventtyp \emph{Gildennachricht}\footnote{vergleiche \Fref[plain]{chap:grundlagen:szenario}} auf einem filterbaren Kanal abgebildet werden. Als Prädikat kann die Gildenzugehörigkeit des Avatars oder eine Liste der Gildenmitglieder, von denen Nachrichten erwünscht sind, angegeben werden. Hyper ist ein Beispiel eines solchen hybriden Systems \cite{Zhang}.\\
Das von \ac{m2etis} zur Verfügung gestellte kanalbasierte Publish/Subscribe-System kann pro Kanal mit einer eigenen Filterungskomponente versehen werden und somit als hybrides System genutzt werden; dies wird in \Fref{chap:konzeption_pubsub} beschrieben.

Ein prominenter Vertreter verteilter, kanalbasierter Systeme ist Scribe, dessen Funktionsweise im nächsten Abschnitt beschrieben wird.

\subsection*{Umsetzung eines kanalbasieren Systemes am Beispiel von Scribe}
\label{chap:related:scribe}
Eine Umsetzung von Publish/Subscribe-Systemen in verteilen Systemen, ist der Aufbau eines Multicast-Trees\index{Multicast-Tree}, d.h. eines durch die Knoten im Netz gebildeten Baumes in dem die Nachrichten verteilt werden. Hierbei wird pro Kanal ein eigener Multicast-Tree aufgebaut. Am Algorithmus von Scribe \cite{Castro2002Scribe}wird diese Struktur beschrieben.

Scribe basiert auf dem strukturierten Overlay-Netzwerk Pastry \cite{Rowstron2001} und erzeugt einen vom Subscriber zum Publischer aufgebauten Baum \emph{reverse path forwarding tree} \cite{Dalal1978}.

\begin{figure}[htbp]
\centering
\resizebox{\textwidth}{!}{%
\includegraphics{grafics/multicast_tree.pdf}}
\caption{Schema eines Multicast-Trees}
\label{fig:multicast_tree}
\end{figure}

\Fref{fig:multicast_tree} zeigt ein Netzwerk mit den sechs Knoten A-F. Die Verbindungen der Knoten werden durch dünne schwarze Linien dargestellt. Beispielsweise hat Knoten C Verbindungen zu A, B, D und F.\\
Der Multicast-Tree benötigt einen Knoten, der die Wurzel (im Folgenden \emph{Root-Knoten} genannt) darstellt. Aus Hashwert des Kanalnamens wird ein Schlüssel berechnet. Derjenige Knoten, der aufgrund der Netzwerkmetrik für diesen Schlüssel zuständig ist, wird Root-Knoten des Kanals. Im abgebildeten Falle ist dies Knoten A.\\
Weiterhin hält jeder Knoten eine Liste bei ihm angemeldeter Knoten. In der Abbildung wird diese Liste durch geschweifte Klammern nach der Knotenbezeichnung dargestellt.

\paragraph*{Subscribe}
Knoten F sendet eine \emph{subscribe}-Nachricht an A. Diese Nachrichten sind in der Grafik durch gebogene gestrichelte schwarze Verbindungslinien mit Pfeil dargestellt. Das Netzwerk würde diese Nachricht über Knoten D und C an A routen. Knoten D lässt die Nachricht terminieren und trägt F in die Liste der Subscriber ein. Knoten D sendet nun selbst eine subscribe-Nachricht an A. C, über den die Nachricht geroutet wird, terminiert diese, trägt D in die Liste ein und sendet selbst eine subscribe-Nachricht an A. A erhält nun diese Nachricht und trägt C in die Liste ein. Damit sind nun insgesamt drei Nachrichten verschickt worden.\\
Wenn sich Knoten E für den Kanal einschreibt, wird die subscribe-Nachricht an A über den Knoten C geleitet. Dieser terminiert die Nachricht und fügt E der Liste hinzu. Da C selbst angemeldet ist, muss keine weitere Nachricht versendet werden.

Scribe fordert periodische Anmeldungen zur Erhöhung der Fehlertoleranz. Ist ein Knoten ausgefallen, routet das Netzwerk die Nachrichten über andere Knoten. Damit kann der Multicast-Tree wieder aufgebaut werden.

\paragraph*{Unsubscribe}
Der Austritt aus einem Kanal erfolgt ähnlich zur Anmeldung. Die Nachricht läuft nur bis zum nächsten Knoten und terminiert dort. Der Knoten entfernt den Sender der Nachricht aus seiner Liste und sendet selbst nur eine \emph{unsubscribe}-Nachricht, wenn die Liste leer ist und er selbst nicht angemeldet ist.

\paragraph*{Publish}
In \Fref{fig:multicast_tree} möchte Knoten B eine Nachricht im Kanal publizieren. B sendet darauf eine Nachricht an den Root-Knoten A, da dieser für diesen Kanal zuständig ist (gebogene türkise Linie). Nun sendet A diese Nachricht an alle Knoten in seiner Liste (gerader türkise Linie mit Pfeil). Dies ist in der Abbildung nur Knoten C. Dieser sendet sie weiter an D und E. E gibt diese Nachricht direkt an die Applikation weiter, während D die Nachricht an F schicken muss.


Hierbei ist klar ersichtlich, dass zusätzliche Nachrichten verteilt werden müssen, wenn Knoten F eine Nachricht im Kanal publizieren möchte. Diese Nachricht muss erst von Knoten F zu Knoten A wandern, damit A diese Nachricht wieder über die anderen Knoten zurücksendet. Optimierte Versionen dieses Algorithmus können hier ansetzen und zu publizierende Nachrichten nicht mehr an den Knoten senden, der ihnen diese Nachricht geschickt hat. So würde C die Nachricht nur noch an E weiterleiten.

Bayeux \cite{Zhuang2001} ist ein ähnliches System, jedoch auf Basis des Overlay-Netzwerkes Tapestry \cite{Zhao2004Tapestry}. Tapestry entspricht auch der generischen API, somit stellt dies keinen Unterschied zu Pastry dar. Im Gegensatz zu Scribe, wird bei Bayeux der Multicast-Tree vom Root-Knoten aus aufgebaut. Aufgrund der unterliegenden Routingstruktur des genutzten Overlay-Netzwerkes können sich diese Pfade unterscheiden.


Nach diesem Einblick in eine mögliche Umsetzung eines kanalbasierten Publish/Sub\-scribe-Systems gibt der kommende Abschnitt am Beispiel von Mercury eine Vorstellung davon, wie filterbasierte Systeme\index{Publish/Subscribe!filterbasiert} in einem dezentralen Netzwerk implementiert sein können.

\subsection*{Umsetzung eines filterbasierten Systemes am Beispiel von Mercury}
\label{chap:related:mercury}
Zur besseren Vorstellung einer Umsetzung für filterbasierte Publish/Subscribe-Systeme\index{Publish/Subscribe!filterbasiert} wird im folgenden Kapitel Mercury \cite{Bharambe2004Mercury} vorgestellt. Obwohl \ac{m2etis} ein kanalbasiertes Publish/Subscribe-System darstellt \cite{Fischer2010a}, ist es sinnvoll eine möglich Umsetzung eines filterbasierten Systems zu beschreiben um die grundlegenden Unterschiede der Systeme genauer auszuarbeiten. 

\paragraph*{Arbeitsweise}
Im System gibt es eine Menge an Attributen, die ihrerseits einen definierten Wertebereich haben. Jedes Attribut wird durch einen eigenen Verbund aus Knoten, den sogenannten \emph{Hub}, bearbeitet. Der Wertebereich ist dabei nicht zwingend symmetrisch auf die Knoten verteilt.

\paragraph*{Anmelden}
Eine Subscription $S$ ist ein Tupel aus Filterbedingungen über die Attribute (z.B. $S := (5 < x <= 20; y = 15)$) sowie Kontaktinformationen des Knotens. $S$ wird an einen beliebigen Knoten eines Hubs gesendet, der für das Attribut aus der Filterbedingung mit der größten Selektivität zuständig ist. Im Beispiel ist dies Attribut $y$. Im Hub wird $S$ nun zu dem Knoten weitergereicht, der den Wertebereich der Filterung abdeckt. Dort wird $S$ in einer Liste gespeichert.

\paragraph*{Publizieren}
Eine Publikation $P$ ist ebenfalls ein Tupel mit bestimmten Werten der Attribute (z.B. $P := (x = 10; y = 0)$). $P$ wird an \emph{alle} Hubs gesendet und dort zum zuständigen Knoten weitergereicht. Dieser prüft nun die Liste der gespeicherten Subscriptions gegen die neue Publikation. Stimmen beide überein, so wird $P$ an den eingeschriebenen Knoten weitergeleitet.

%\paragraph*{Offene Punkte}
%\begin{itemize*}
%\item Änderung der Attribute zur Laufzeit?
%\item Auswahl der Knoten für einen Hub?
%\item Aufteilung der Wertemenge auf die Knoten?
%\end{itemize*}

\paragraph*{Ähnliche Algorithmen}
Mirinae ist ebenfalls ein filterbasiertes Publish/Subscribe-System, stellt den Wertebereich eines Attributes jedoch als Hyperwürfel dar. Eine automatische Anpassung dieser Aufteilung ermöglicht eine schnelle Anpassung der Routingtabelle und damit einen kurzen Weg für die Nachrichten \cite{Choi2005Mirinae}.


\subsection{VON}
\label{chap:related:von}
\ac{von} ist in seinen Grundzügen stark unterschiedlich zu den bisher vorgestellten Umsetzungen. \ac{von} nutzt das \ac{p2p}-Netzwerk nicht nur als Kommunikationsmedium, sondern auch dessen Aufbau als Verteilungsstruktur des Publish/Subscribe-Systems \cite{Hu2006VON}. VON zielt auf die Verteilungsoptimierung von Events zur Positionsänderung, muss allerdings über Applikationswissen verfügen: die Position des Spielers. \ac{vast} \cite{Backhaus2007Voronoibased} greift das Konzept von \ac{von} auf und testet eine Implementierung auf OpenSIM \cite{Baumgart2007OverSim}.

\begin{figure}[htbp]
\centering
\resizebox{\textwidth}{!}{%
\includegraphics{grafics/voronoi_von_backhaus.pdf}}
\caption{Struktur eines VON-Netzwerkes (aus \cite{Backhaus2007Voronoibased})}
\label{fig:von}
\end{figure}

\Fref{fig:von} zeigt einen Aufbau eines VON-Netzwerks. Die Spielwelt wird anhand der Position der einzelnen Knoten in Voronoi-Diagramme \cite{Aurenhammer1991Voronoi} unterteilt und jedem Knoten ein eigener Bereich zugeteilt. Ein Knoten hält Verbindungen zu seinen Nachbarn und kennt angrenzende Nachbarn im Bereich seiner \ac{aoi}. \emph{Enclosing neighbors} sind angrenzende Nachbarn, deren gesamter zugehöriger Bereich innerhalb der \ac{aoi} des Knotens liegt. \emph{Boundary neighbors} sind nicht angrenzende Knoten, deren eigener Bereich nicht vollkommen innerhalb der \ac{aoi} liegt.

\textbf{Anmeldungen} im Publish/Subscribe-System sind implizit, denn \textbf{Publikationen}, also Positionsänderungen, werden von einem Knoten an alle direkt angrenzenden Nachbarn (\emph{enclosing neighbor} in \Fref{fig:von}) gesendet. Damit das System konsistent bleibt, werden dabei auch Informationen über andere Nachbarn ausgetauscht. Mit jeder Positionsänderung verändert sich die Aufteilung des Voronoi-Diagrammes und damit auch die Nachbarschaften.

VON lässt sich nicht eindeutig als kanalbasiertes oder filterbasiertes System klassifizieren. Die Fixierung auf die Position und Auswertung der \ac{aoi} kann einerseits als \emph{filterbasiertes} System mit einem einzigen Attribut und andererseits als \emph{hybrides} System mit einem  Kanal und entsprechender Filterung anhand der AOI angesehen werden.

%\missing{Beschreiben!}
%Ebenfalls mit Delauny-Triangulierung arbeitet \cite{Liebeherr2002Applicationlayer}.


Nach den bisher dargestellten Grundlagen von \ac{p2p}-Netzwerken, Publish/Subscribe-Systemen und Einblicken in verschiedene Umsetzungen beschäftigt sich das nächste Kapitel mit der Evaluation dreier \ac{p2p}-Netzwerke m ein geeignetes System als Netzwerk für \ac{m2etis} auswählen.


\subsection{Betrügerisches Verhalten in p2p-Netzwerken}
\label{chap:grundlagen:cheating}
Betrügerisches Verhalten\index{Betrügerisches Verhalten} (Cheating) ist in p2p-Systemen einfacher als in Client/Server-Systemen, da die Kommunikation nicht zwingend über einen (vom Betreiber kontrollierten) Server läuft. Mit gefälschten Nachrichten für Statusmeldungen wie Positionsänderungen oder zur Entscheidungsfindung über die Reihenfolge von Aktionen können andere Knoten massiv benachteiligt werden. Das Wissen über das zugrunde liegende System wird genutzt um gezielt wichtige Aspekte des Spieles zu manipulieren.

Webb \cite{Webb2007Cheating} gibt eine Übersicht über Betrug in Netzwerkspielen und geht hier auch auf die Unterschiede zwischen Client/Server-Systemen und p2p-Systemen ein. Cheating wird in vier grundlegende Bereiche eingeteilt: \emph{Game level cheats}, \emph{Application level cheats}, \emph{Protocol level cheats} und \emph{Infrastructure level cheats}. Mögliche Verfahren wie \emph{Lockstep} \cite{Baughman2007}, das die Spielzeit in Runden einteilt, werden vorgestellt und ausgewertet.

Als Beispiel für einen \emph{Protocol level cheat} wird das Ausnutzen des \emph{Dead-Reckoning}-Verfahrens\index{Dead-Reckoning} \cite{Pantel2002} beschrieben. Dead-Reckoning kaschiert ausgefallene Nachrichten und die Latenz des Netzwerks und bietet durch vorberechnete Aktionen der anderen Spieler einen konstanten Spielfluss. Beispielsweise wird bei einer ausbleibenden Positionsnachricht eines Mitspielers angenommen, dessen Avatar bewegt sich ohne Richtungsänderung weiter. Sendet ein Spieler innerhalb eines bestimmten Intervalles keine Nachrichten mehr, muss das System in Aktion treten und beispielsweise die Welt in einen -- auf allen Rechnern -- konsistenten Zustand bringen. Sendet ein Spieler nur noch die minimal benötigen Nachrichten, können in der Zeit bis zum geforderten zu versendenden Update die Nachrichten der anderen Spieler ausgewertet werden und das Spiel somit zu eigenen Gunsten beeinflusst werden. Algorithmen, die diese Problematik angehen, werden entwickelt \cite{Aggarwal2005}.

Kabus \cite{Kabus2007Design, Kabus2009} beschreibt grundsätzliche Techniken, die in p2p-Systemen genutzt werden, um Betrug aufzudecken beziehungsweise zu verhindern. Beispielsweise können für eine Entscheidung über eine Spieleraktion zufällige Knoten gewählt werden. Diese validieren die Aktionen und können effektiv Betrug verhindern. Zum Beispiel ist die Aufnahme eines Gegenstand in der virtuellen Welt nur erlaubt, wenn andere Mitspieler bestätigen, dass sich der entsprechende Spieler an der Position des Gegenstands aufhält. Solche Verfahren gehen jedoch mit einer erhöhten Anzahl an Nachrichten einher.

In weiteren Arbeiten beschäftigt sich Castro mit sicherem Routing in p2p Overlay-Netzwerken \cite{Castro2002Secure}, Kabus \cite{Kabus2005Addressing} und Dautermann \cite{Dautermann2007} im Speziellen mit verteilten \acp{mmog}. Ferretti geht genauer auf den Aspekt des Betrugs mit \enquote{Zeitspielereien} ein \cite{Ferretti2008Cheating}.

In diesem Abschnitt wurde ein Überblick möglicher Betrugsarten gegeben und auch Möglichkeiten zur Verhinderung dargestellt. Betrügerisches Verhalten\index{Betrügerisches Verhalten} und dessen Verhinderung beziehungsweise Vermeidung sind nicht Thema dieser Arbeit. Das zu entwickelnde Framework bietet genug Freiheiten, betrugsresistente Publish/Subscribe-Algorithmen einzusetzen oder entsprechende andere Ansätze an das Framework anzubinden.


\section{Aufbau von M$^2$etis}
\label{chap:grundlagen:aufbau_metis}

Diese Arbeit entsteht im Rahmen des \ac{m2etis}-Projektes Cite Cite Cite und bla.

Aufbaubildchen

\cite{Fischer2010a, Fischer2010Event}



\chapter{Evaluation p2p Overlay-Netzwerke}
\label{chap:evaluation_p2p}

Dieses Kapitel bietet einen Überblick über einige p2p-Netzwerke und evaluiert diese anhand gestellter Anforderungen. Diese Evaluation beeinflusst die Entscheidung für ein Overlay-Netzwerk, auf das schließlich, das in dieser Arbeit zu entwickelnde, generische Publish/Subscribe-System gesetzt wird. Die Evaluation bedient sich zahlreicher Arbeiten, die sich alleine dem Vergleich dieser Netzwerke widmen \cite{Lua2005Survey, Goetz2005, Li2004Comparing, Darlagiannis2006Peertopeer, Castro2002Secure, Bo2003PeertoPeer} und geht auch auf ihre Nutzbarkeit als Basis für \emph{Application level multicast} sprich ein Publish/Subscribe-System ein \cite{Hosseini2007Survey, Fahmy2007, Castro2003Evaluation, Ratnasamy2001}.

Zuvor müssen jedoch die eigenen Anforderungen an solche Systeme identifiziert werden. Zu den offensichtlichen Anforderungen wie beispielsweise \emph{Skalierbarkeit} gesellen sich jedoch auch spezielle Anforderungen aus Spielsicht hinzu. Diese sind beispielsweise das Vorhandensein eines Masterservers oder das Übertragen von Applikationswissen auf das Netzwerk um damit dessen Entscheidungen bezüglich Nachbarschaften oder Versand von Nachrichten (Routing) zu beeinflussen.

\section{Anforderungen}

\paragraph{Geringe Latenz} Schnelle Reaktionszeiten und Nachrichtenübermittlung sind bei \ac{mmog} unverzichtbpar. Ebenfalls müssen größere Nachrichten (beispielsweise Update der Welt) schnell übertragen werden damit der Spielfluss nicht behindert wird. Dies lässt sich anhand der Anzahl der Hops beim Nachrichtenversand messen.

\paragraph{Skalierbar} Selbst bei einer großen Anzahl an Knoten soll das Netz nicht kollabieren. Hierbei ist es auch wichtig, dass Knoten nicht unbedingt lange im Spiel sein müssen. Zwar kann davon ausgegangen werden, dass ein durchschnittlicher Spieler längere Zeit im Spiel verbringt, aber durch Netzausfälle oder sonstigen Unbill kann dies stark variieren. Insofern ist es wichtig wie sich die Netzwerke bei großen Fluktationen verhalten \cite{Li2004Comparing}.

\paragraph{Fehlertoleranz bei Knotenausfall} Fallen Knoten aus, muss sich das Netz ohne großen Kommunikationsaufwand selbst reparieren. Ebenfalls mögliche Netzwerkpartitionierungen sind in dieser Arbeit jedoch kein Hindernis, denn der Hauptserver kann über eine gesonderte Verbindung immer erreicht werden. Damit kann das Netz wieder verbunden werden.\\
Interessant hierbei ist auch die eingebaute Redundanz einiger Systeme, die Daten auf mehrere Knoten verteilen. Wie sich dies im Vergleich von statischen Daten zu sich häufig verändernden Objekten der Spielewelt verhält ist zu untersuchen. 

\paragraph{Kommunikation über das Netzwerk} Das Netzwerk soll nicht nur das schnelle Auffinden von Peers ermöglichen, sondern auch einen Transport der Nachricht (Routing) durch das Netzwerk selbst bereitstellen. Die Alternative Direktverbindungen soll nur genutzt werden, wenn eine Datenübertragung im Netzwerk nicht performant genug ist, weil z.B. die Bandbreite der zwischengeschalteten Peers zu gering ist.

\paragraph{Bestimmung der Nachbarschaft} Eine dynamische Bestimmung der Nachbarschaftsgröße kann von Vorteil sein. So könnten Knoten mit mehr Bandbreite (bzw. entsprechenden anderen Metriken) mehr direkte Verbindungen halten als Knoten mit geringer Bandbreite (oder geringer Spieldauer).

\paragraph{Eingriff in Routingentscheidungen} Applikationswissen hilft auch beim Eingriff in das Routing des Netzes. So können Knoten bevorzugt zur Weiterleitung einer Nachricht ausgewählt werden. Diese Knoten zeichnen sich beispielsweise durch eine große Bandbreite oder spezielle Applikationsmetriken\footnote{Bsp: Spieler befindet sich in der selben Stadt} aus.

\paragraph{Verfügbarkeit als C/C++-Bibliothek} Da der Prototyp dieser Arbeit sowie das Umfeld in C++ entwickelt wird, gibt es damit eine weitere Anforderung an das Netzwerk: Die Verfügbarkeit als C/C++-Bibliothek.\\
Damit ist das Netzwerk einfach aus bestehendem Code nutzbar ohne dass kostenintensive Brücken zwischen beispielsweise Java und C++ geschlagen werden müssen. Da zudem betriebssystemübergreifend\footnote{In unserem Fall auf Windows und Linux} entwickelt und getestet wird, ist außerdem ein zur Verfügung stehender Quellcode vorteilhaft. Nur dadurch ist es möglich eventuellen betriebssystembezogen Netzwerk- oder Threadcode auf  Boost\footnote{Zahlreiche Bibliotheken für C++: http://www.boost.org} zu portieren um somit eine betriebssystemübergreifende Benutzung zu ermöglichen.

\paragraph{Unterbau eines Publish/Subscribe-Systems} Das Overlay-Netzwerk muss so gestaltet sein, dass es dem Anwendungsfall \emph{Application level multicast} genügt und dessen besondere Anforderungen (auch von Spielseite)\footnote{siehe \Fref{chap:konzeption_pubsub}} her unterstützt.

Anhand dieser Anforderungen sind unstrukturierte Netzwerke nicht als Netzwerksystem für dieser Arbeit geeignet. Eine Suche, bzw. Datenübertragung durch Flooding widerspricht klar der Anforderung nach geringer Latenz. Ebenso sind diese Netzwerke auf eine Übertragung via Direktverbindung ausgelegt.

\cite{Dabek2003Towards} stellt eine generische \ac{api} für \ac{kbr} Systeme vor, wie es viele der strukturierten Netzwerke anbieten vor und zeigt wie darauf aufbauen verschiedene Systeme wie \ac{dht}, \ac{dolr} und \ac{cast}\footnote{entspricht Application level multicast} implementiert werden können. Ebenfalls wird gezeigt wie einige bekannte Systeme (CAN, Chord, Pastry und Tapestry) diese die Anforderungen erfüllen.\\
Daraus ergibt sich eine weitere spezielle Anforderung an das Overlay-Netzwerk:

\paragraph{Anpassbarkeit an generische \ac{api}} Lässt sich das Netzwerk der generischen \ac{api} anpassen, gewinnt das an Flexibilität, da die Netzwerkschicht ohne große Änderungen an den restlichen Systemschichten ausgetauscht und verändert werden kann.


\subsection*{Generische KBR-API}
\label{chap:evaluation_p2p:generic_api}
Dabek moniert die unterschiedlichen Schnittstellen der verschiedenen strukturierten p2p-Netzwerke. Dies mache es aus Entwicklersicht schwer, vom Netzwerk zu abstrahieren und dieses gegebenenfalls zu wechseln \cite{Dabek2003Towards}.

\lstinputlisting[caption={Upcalls der generischen API}, label=lst:towards_upcall]{listings/towards_upcall.cpp}



\section{Evaluation}
In diesem Kapitel werden die vier bekannten Systeme Chord \cite{Stoica2003}, Pastry \cite{Rowstron2001}, Tapestry \cite{Zhao2001Tapestry,Zhao2004Tapestry} und CAN \cite{Ratnasamy2001Scalable} miteinander verglichen. Die ersten drei sind in ihrem Aufbau ähnlich (Schlüsselraum wird auf einen Ring verteilt) und unterscheiden sich in der Art des Routings. CAN hingegen bildet den Schlüsselraum auf ein d-dimensionalen kartesisches Koordinatensystem ab. Alle vier Systeme sind laut \cite{Dabek2003Towards} im Hinblick auf die generische \ac{api} nutzbar.

Zur Entscheidungsfindung werden die Netzwerke anhand folgender Gesichtspunkte verglichen:
\begin{itemize*}
\item Aufbau und Struktur
\item Routing
\item Nachbarschaft
\item Eintritt und Austritt (Fehlerfall) von Knoten
\item Nutzbarkeit als Basis für \ac{cast}
\end{itemize*}

\subsection{Aufbau und Struktur}
\paragraph{Chord}
Chord \cite{Stoica2003} legt die l-bit wertigen Schlüssel (meist Zahlen im Bereich $[0,2^l-1]$) auf einem eindimensionalen Ring modulo $2^l$ im Uhrzeigersinn an. Jedem Knoten und jedem Datum ist ein eindeutiger Schlüssel zugewiesen, diese sind \emph{ID} und \emph{key} benannt. Ein Datum $X$ ist dem Knoten zugewiesen, dessen ID größer gleich dem key ist. Dieser Knoten wird Nachfolger von X, \emph{SUCC(X)}, genannt. Analog dazu gibt es auch einen Vorgänger von X, \emph{PRED(X)}.

Damit ist ein Knoten für alle Daten zuständig, die - bildlich gesehen - im Ring gegen den Uhrzeigersinn vor ihm liegen. In \Fref{fig:chord_key_space} ist dies mit $l=6$ für sechs Knoten und fünf Datenpunkten gezeigt. Knoten 14 (N14) ist für das Datum mit Schlüssel 10 (K10) zuständig. Knoten 32 ist für K16 und K25. K51 ist bei N51 zu finden. Aufgrund der Ringstruktur ist N1 für K55 zuständig.

\begin{figure}[htbp]
\centering
\includegraphics{grafics/chord_key_space.pdf}
\caption{Schlüsselraum ($l=6$) für Chord mit sechs Knoten ($Nx$) und fünf Daten ($Kx$). Die gestrichelten Pfeile stellen die Einträge der Fingertabelle für Knoten $N1$ dar.}
\label{fig:chord_key_space}
\end{figure}


\paragraph{Pastry / Tapestry}
Pastry \cite{Rowstron2001} und Tapestry \cite{Zhao2001Tapestry,Zhao2004Tapestry} sind sich sehr ähnlich, da beiden auf Plaxtons Arbeit \cite{Plaxton1997Accessing} aufbauen. Auf Unterschiede wird explizit hingewiesen.

Pastry besitzt ebenfalls einen l-bit wertigen Schlüsselraum, dabei werden Schlüssel als Zahlen zur Basis $2^b$ dargestellt\footnote{l meist 128; b meist 4}, wobei die Wahl von $b$ einen Einfluss auf das Routing hat. Ein Datum ist dem Knoten zugewiesen, dessen ID am nähesten zum Schlüsselwert des Datums liegt.  \Fref{fig:pastry_key_space} zeigt dies beispielhaft für sechs Knoten und fünf Datensätzen. Im Unterschied zu Chord ist hier Knoten $N14$ für $K16$ und Knoten $N54$ für $K55$ zuständig.\\
\missing{Noch nicht so sehr schön beschrieben!}

Tapestry erzeugt automatische Redundanz, da hier die Daten auf mehrere Knoten verteilt werden.

\begin{figure}[htbp]
\centering
\includegraphics{grafics/pastry_key_space.pdf}
\caption{Schlüsselraum ($l=4$) für Pastry mit sechs Knoten ($Nx$) und fünf Daten ($Kx$).}
\label{fig:pastry_key_space}
\end{figure}


\paragraph{CAN}
Der Schlüsselraum bei CAN \cite{Ratnasamy2001Scalable} ist ein d-dimensionaler Torus. Die Schlüssel werden als d-Tupel (zum Beispiel $(x,y)$ für $d=2$) dargestellt. Wie bei Pastry ist der numerisch näheste Knoten für ein Datum zuständig. Der Schlüsselraum ist in nicht überlappende Zonen eingeteilt und hat eine feste Größe. Jeder Knoten \emph{besitzt} eine solche Zone, über deren Ausmaß er definiert ist, und ist damit für alle Daten zuständig, die in dieser Zone liegen. Ein Schlüssel wird wie bei \ac{dht} üblich über eine segmentierte Hashfunktion berechnet. Jedes Segment bildet dabei eine Dimension ab.

\Fref{fig:can_key_space} zeigt einen zweidimensionalen Schlüsselraum mit den drei Knoten A,B und C und fünf Daten. Der Schlüsselraum ist komplett auf die drei Knoten aufgeteilt, wobei A für den Bereich $(0, .5)-(1, 1)$, B für $(0, 0)-(.5, .5)$ und C für $(.5, 0)-(1, .5)$ zuständig ist.

\begin{figure}[htbp]
\centering
\includegraphics{grafics/can_key_space.pdf}
\caption{2-dimensionaler Schlüsselraum für CAN mit drei Knoten (A, B, C) und fünf Daten (schwarze Punkte).}
\label{fig:can_key_space}
\end{figure}


\subsection{Routing}
\paragraph{Chord}
Bei Chord besitzt jeder Knoten eine Verbindung zu seinem direkten Vorgänger und seinem direkten Nachfolger. Eine Nachricht wird an den Nachfolger geschickt, bis sie zum zuständigen Knoten gelangt. Bei einer \emph{LOOKUP(x)}-Nachricht\footnote{Suche für key x Knoten N, so dass gilt: $N = SUCC(x)$.} prüft jeder involvierte Knoten A, ob sein Nachfolger für den Schlüssel zuständig ist, d.h. $ID_A < x \le SUCC(A)$. Ist dies der Fall, so sendet Knoten A die Antwort SUCC(A) rückwärts den Pfad der Nachricht zurück. Bei normalem Nachrichtenaustausch wird die Nachricht an den entsprechenden Knoten weitergeleitet.

Da dies eine sehr ineffizientes  Routing darstellt, pflegt jeder Knoten eine sogenannte \emph{finger table}. Die maximal $l$ Einträge in dieser Tabelle zeigen auf andere Knoten im Ring, so dass der Eintrag in Zeile $i$ von Knoten $n$ denjenigen Knoten enthält der $n$ mit mindestens $2^{i-1}$ folgt.\\
\Fref{fig:chord_key_space} stellt die Fingertabelle von Knoten $N1$ dar. Die ersten Einträge SUCC(2), SUCC(3) und SUCC(5) verweisen auf Knoten $N5$. Der dritte Eintrag verweist auf $SUCC(9) = N14$. Analog dazu ergeben sich die restlichen Einträge.

Über diese Fingertabelle können Nachrichten eine weiter Strecke auf dem Ring überbrücken und die Routingzeit wird stark verkürzt. Da die IDs in der Tabelle exponentiell zur Basis zwei ansteigen, halbiert sich die Distanz zum Ziel. Damit hat das Routing eine Komplexität von $O(log N)$.

\paragraph{Pastry / Tapestry}
Jeder Knoten verwaltet neben dem ihm zugeteilten Daten drei Strukturen die dem Routing dienen. Diese sind das \emph{leaf set} mit Einträgen zu Knoten die im Schlüsselraum naheliegen, das \emph{neighborhood set} mit Einträgen zu Knoten die aus Netzwerksicht nahe liegen und die Routingtabelle.

\begin{figure}[htbp]
\centering
\includegraphics{grafics/pastry_routing_table.pdf}
\caption{Routing table, leaf set und neighborhood set (nach \cite{Goetz2005}) des Knoten 103220 für $l=12, b=2$. Einträge in Zeile $i$ haben einen Präfix der Länge $i$ mit dem Knoten $103220$ gemein. Die Übereinstimmungen sind zur Verdeutlichung fett hervorgehoben.}
\label{fig:pastry_routing_table}
\end{figure}

Die Routingtabelle besteht aus $\frac{l}{b}$ Reihen mit je $2^b -1$ Einträgen. \Fref{fig:pastry_routing_table} zeigt dies beispielhaft für Knoten $103220$ mit $l=12, b=2$. Einträge in Zeile $i$ haben einen Präfix der Länge $i$ mit dem Knoten $103220$ gemein. Die Übereinstimmungen sind in der Abbildung fett hervorgehoben. Ist kein passender Knoten bekannt, wird das entsprechende Feld nicht ausgefüllt. Damit hat die Routingtabelle Ähnlichkeiten zur Fingertabelle bei Chord. Ein Knoten hat ungenaues Wissen über entfernte Knoten. Der Detailgrad an Routinginformationen erhöht sich pro Zeile in der Routingtabelle. Wenn im System nur sehr wenig Knoten vorhanden sind, dann sind die letzten Reihen der Routingtabelle ebenfalls nur spärlich belegt. Im Durchschnitt sind bei $n$ Knoten im System nur $log_{2^b} n$ Einträge belegt.\\
Bei der Belegung der Routingtabelle werden bei gleichem Präfix diejenigen Knoten gewählt, die aus Netzwerksicht näher sind.

Das leaf set enthält die $l$ numerisch nähesten Knoten, $\frac{l}{2}$ davon sind kleiner und $\frac{l}{2}$ größer als der aktuelle Knoten. Neben Informationen zu Routingentscheidungen wird es zur Reparatur genutzt, wenn nahe gelegene Knoten ausfallen.

Das eigentlich Routing unterscheidet zwei Fälle: Zuerst prüft der Knoten ob der Schlüssel $k$ im Bereich seines leaf set ist. Ist dies der Fall, wird die Nachricht an den entsprechenden Knoten gesendet. Ist dieser Knoten für den Schlüssel zuständig, endet das Routing. Fällt $k$ nicht in den Bereich des leaf set, wird die Nachricht via Routingtabelle an einen entfernteren Knoten gesendet. Hierzu wird ein Eintrag gewählt, der eine größere beziehungsweise die größte Prefixübereinstimmung mit $k$ hat. Existiert kein solcher Eintrag, wird die Nachricht an den numerisch nähesten Knoten (zu $k$) mit gleicher Präfixübereinstimmung geschickt.\\
Da Nachrichten immer an Knoten mit einer größeren Übereinstimmung oder an nähere Knoten mit gleicher Präfixübereinstimmung gesendet werden, können keine Zyklen auftreten.

Dadurch verringert sich die Anzahl der Knoten mit längeren Präfixübereinstimmungen in jedem Schritt um mindestens den Faktor $2^b$. Somit hat das Routing eine Komplexität von $O(log_{2^b} N)$.

Das Routing von Tapestry ist sehr ähnlich zu dem hier vorgestellten Routing, allerdings nutzt Tapestry ein suffix-basiertes System. Ebenso speichert Tapestry in einem Eintrag der Routingtabelle mehrer mögliche Peers. So kann bei einem Ausfall schneller ein Ersatz gefunden werden.

Pastry genügt den Anforderungen der generischen \ac{api} und ruft in jedem Routingschritt die upcall-Funktion route\_message auf. Durch diese kann gezielt in Routingentscheidungen eingegriffen werden und durch Applikationswissen besser gelegene Knoten als Routinghinweis genutzt werden.

\paragraph{CAN} 
Die gespeichter Routinginformation ist bei CAN am geringsten: Jeder Knoten speichert lediglich seine Nachbarn, das sind Knoten deren Zonen die Zone des Kontens berühren, ab. Über die Zoneninformation jedes Nachbarn wird nun das Routing bestimmt. Eine Nachricht wird nun an den Knoten geschickt, der aufgrund seiner Zoneninformation näher am Ziel ist.

In \Fref{fig:can_routing} hat Knoten N1 die vier Knoten N2, N3, N4 und N5 als Nachbarn. Knoten N4 hingegen ist nur mit N1 und N5 benachbart.\\
Beim Routing kann Applikationswissen genutzt werden, so dass beispielsweise Knoten N2 statt N5 zum Routen der Nachricht von N1 genutzt wird. 

\begin{figure}[htbp]
\centering
\includegraphics{grafics/can_routing.pdf}
\caption{Routing und Nachbarschaft bei CAN. Nachbarschaft von $N1 = \{N2, N3, N4, N5\}$ Routing der Nachricht von $N1$ zu $K$ via $N5$. Eine alternative Route via $N2$ ist gestrichelt dargestellt.}
\label{fig:can_routing}
\end{figure}

Die durchschnittliche Länge eines Pfades bei $d$ Dimensionen und $n$ Knoten ist $O(d\cdot n^\frac{1}{d})$. Sind die Zonen gleich aufgeteilt, so besitzt jeder Knoten max. $2d$ Nachbarn und die Anzahl an Hops verringert sich auf $O(\frac{4}{d}\cdot n^\frac{1}{d})$


\subsection{Nachbarschaft}
\paragraph{Chord}
Die Nachbarschaft ist bei Chord begrenzt. Jeder Knoten hat eine Verbindung zu seinem Vorgänger sowie Nachfolger auf dem Ring und hält Einträge in der Fingertabelle vor. In die Routingentscheidungen kann somit nicht direkt eingegriffen werden.

\missing{routing-upcalls? Eingriff in Entscheidungen?}

\paragraph{Pastry / Tapestry}
Das neighborhood set (siehe \Fref{fig:pastry_routing_table}) enthält die $|m|$ nähesten Knoten aus Netzwerksicht. Obwohl es im Routing keine Rolle spielt kann es dazu genutzt werden in späteren Entscheidungen geeignete Knoten zu finden.\\
Da die Größe des leaf set ebenfalls wählbar ist, können hier ebenfalls vermehrt nahe Knoten platziert werden.

\missing{da geht noch was … börp}

\paragraph{CAN}
Die Nachbarschaft von CAN wurde bereits im vorigen Abschnitt behandelt. Dank deren Einfachheit ist der Ein- und Austritt von Knoten zwar besonders einfach und tangiert nur wenige Knoten im Netz, jedoch kann auf die Nähe aus Netzwerksicht nur beim Eintritt eines Knotens bei der Wahl seiner Koordinate Rücksicht genommen werden.

Die Erhöhung der Dimension bedingt eine größere Nachbarschaft und damit, neben einem kürzerem Routing, auch die Möglichkeit mehr nahe Knoten zusammenzubringen. Als weiterer Punkt, können in CAN sogenante \emph{Realitäten} genutzt werden. Hierbei werden alle Knoten und Daten mehreren CAN-Netzwerken dergestalt zugeordnet, dass sie andere Koordinaten besitzen. In einem System mit $r$ Realitäten muss ein Knoten demnach $z$ verschiedene Zonen und Nachbarschaften verwalten. Jedoch ist damit jedes Datum redundant im System gespeichert und eine Nachricht kann in der Realität übertragen werden, die die kürzeste Route verspricht.

\missing{Soll ich reinbringen, dass Knoten pro Realität anders gruppiert werden könnten? Nähe, virtueller Ort, Team}

\subsection{Eintritt und Austritt (Fehlerfall) von Knoten}
\paragraph{Chord}
Bei Chord kann die ID für einen neuen Knoten $n$ frei gewählt werden, es muss lediglich ein Knoten $b$ im System bekannt sein. n routet eine LOOKUP(n)-Nachricht via b und erfährt somit seinen Nachfolger auf dem Ring. In gleicher Weise vervollständigt $n$ seine Fingertabelle. Weiterhin teilt $n$ seinem Nachfolger mit, dass $n$ sein neuer Vorgänger ist.

Zur Stabilisierung und Vervollständigung der Routinginformationen arbeitet jeder Knoten im Ring periodisch die Funktion \emph{stabilize} ab. Jeder Knoten $a$ fragt seinen Nachfolger $n$ nach dessen Vorgänger $s$. Wenn $a != s$, so ist Knoten $s$ neu in den Ring eingetreten. $a$ informiert den neuen Knoten $s$, dass er sein Vorgänger ist und ändert selbst den eigenen Nachfolger auf $s$ ab. $a$ kennt nun den Bereich seiner Zuständigkeit $[ID_a, PRED(a)[$ und kopiert diese Daten von seinem Nachfolger und weißt diesen auf die Zuständigkeitsänderung hin.\\
Die Aktualisierung der Fingertabelle \emph{fix\_fingers} wird ebenfalls auf jedem Knoten periodisch angestoßen. Für einen zufällig gewählten Eintrag wird überprüft ob dieser noch aktuell ist.

Die Verzögerte Aktualisierung hat keinen großen Einfluss auf die Korrektheit oder Geschwindigkeit des Routings, da Nachrichten an $a$ über die Nachfolger- beziehungsweise Vorgängerverbindungen der Knoten um $a$ weitergeleitet werden und somit lediglich ein weiterer Knoten involviert ist.\\
Bei einer LOOKUP(n)-Nachricht gibt der Vorgänger von n den jeweils richtigen Knoten zurück oder leitet diese Nachricht noch einmal an seinen Nachfolger weiter. Bei einer normalen Nachricht, die zum Nachfolger geroutet werden würde, kann dieser feststellen, dass er nicht mehr für das Datum zuständig ist und die Nachricht an seinen Vorgänger weiterleiten beziehungsweise selbst antworten, wenn die Daten noch nicht umkopiert sind. 

Dies bedeutet allerdings, dass der neue Knoten $n$ erst später von seinem Vorgänger erfährt und damit auch erst spät alle ihm zugeteilten Daten kennt und zu sich übertragen kann.

\begin{figure}[htbp]
\centering
\includegraphics{grafics/chord_new_node.pdf}
\caption{Schlüsselraum von Chord nach Ankunft von Knoten $N20$. Die Zuständigkeit für $K16$ sowie die Fingertabelle von $N1$ (gestrichelte Linien) wurden angepasst.}
\label{fig:chord_new_node}
\end{figure}

\Fref{fig:chord_new_node} verdeutlicht den Neueintritt von Knoten $N20$. Die Änderungen sind in grün dargestellt. $N1$ passt seine Fingertabelle an und $N20$ ist für $K16$ zuständig.

Der Ausfall von Knoten wird über \emph{timeouts} ermittelt. Tritt ein timeout auf, so wird die Nachricht einfach an den besten bekannten Vorgänger des ausgefallenen Knotens weitergeleitet. Im schlimmsten Falle ist dies der Nachfolger des sendenden Knotens. Daraus wird ersichtlich, dass ein valider Nachfolger notwendig ist. Somit hält jeder Knoten eine Liste von möglichen Nachfolgern vor, die während \emph{stabilize} erstellt werden kann. Für fehlerhafte Knoten in der Fingertabelle kann \emph{fix\_fingers} explizit aufgerufen werden.

Knotenausfall bedeutet nicht nur einen Ausfall des Knotens, sondern bedingt, dass die dort gespeicherten Daten nicht mehr erreichbar sind. Da bei Chord immer SUCC(key) für das jeweilige Datum zuständig ist, empfiehlt es sich auf Applikationsebene die Daten auf Knoten $n$ und $SUCC(n)$ zu replizieren.

Verlässt ein Knoten das Netz, so beeinflusst dies das System nicht. Jedoch ist es effizienter wenn ein verlassender Knoten seinem Vorgänger und Nachfolger dies mitteilt, die Verbindungen angepasst werden und die Daten explizit übertragen werden.

\paragraph{Pastry / Tapestry}
Einem neuen Knoten $n$ wird von Applikationsseite ein freiwählbarer Schlüssel gegeben. Meist berechnet sich dieser Schlüssel anhand dem Hashwert der IP oder seines öffentlichen Namens. Weiterhin geht das System davon aus, dass $n$ aus einer Liste bekannter Knoten denjenigen Knoten $x$ wählen kann, der aus Netzwerksicht am nähesten ist. Von diesem Knoten kann das neighborhood set kopiert werden. Zum Aufbau der Routingtabelle und des leaf set lässt $n$ via $x$ eine \emph{JOIN}-Nachricht an einen numerisch nahen Schlüssel zu $n$ routen. Diese Nachricht gelangt schließlich zu Knoten $c$, von dem das leaf set kopiert werden kann, da $c$ und $n$ sich nahe sind. Alle Knoten die diese \emph{JOIN}-Nachricht weiterleiten senden ihre Routingtabelle an $n$. Für jeden Routinghop kopiert sich $n$ die entsprechende Zeile aus der Routingtabelle, da ausgehend von keiner Präfixübereinstimmung mit dem nahen Knoten $x$, jeder weitere Hop eine größere Präfixübereinstimmung bringt.\\
Im Gegensatz zur nachträglichen Aktualisierung bei Chord, wird nun die gesamte Routinginformation an alle bekannte Knoten gesendet. Der neue Knoten ist nun im Netzwerk bekannt und erreichbar.

Ausgefallene Knoten werden ebenfalls anhand von timeouts beim Routing entdeckt. Da die Einträge des neighborhood set nicht im Routing involviert sind, müssen diese periodisch geprüft werden. Fehlerhafte Einträge in der Routingtablle können über einen anderen Eintrag mit gleicher Präfixübereinstimmung kompensiert werden, müssen aber entfernt werden um ein stabiles und sicheres Routing zu ermöglichen. Hierzu können von benachbarten Einträgen Routinginformationen angefordert werden um die entstandene Lücke zu füllen. Ein fehlerhafter Eintrag im leaf set (neighborhood set)Damit hat das Routing eine Komplexität von $O(log N)$. kann auf ähnliche Weise repariert werden, hier werden Informationen von den anderen Einträgen im leaf set  (neighborhood set) angefordert.

Ein verlassender Knoten kann vom System wie ein ausgefallener Knoten behandelt werden, aber wie auch bei Chord ist es nützlich das Verlassen eines Knoten speziell zu behandeln, um die Datenintegrität zu gewährleisten und um unnötigen Nachrichtenversand im System zu vermeiden.

\paragraph{CAN}
\begin{figure}[htbp]
\centering
\includegraphics{grafics/can_new_node.pdf}
\caption{Eintritt und Fehlerfall bei CAN. Nach Eintritt von Knoten $N7$ und Aufteilung der Zone von $N4$ ist die Nachbarschaft von $N1 = \{N2, N3, N4, N5, N7\}$. Nach dem Ausfall von $N2$ (rot) vereinigt $N3$ die Zonen.}
\label{fig:can_new_node}
\end{figure}

Für den Eintritt eines neuen Knoten n muss wieder ein Knoten b aus dem Netz bekannt sein. Eine Koordinate im Schlüsselraum wird n zugewiesen und eine spezielle \emph{JOIN}-Nachricht via d an die gewählte Koordinate gesendet. Ist diese Nachricht über das normale Routing bei dem für diese Koordinate zuständigem Knoten d angekommen, halbiert dieser seine Zone und weist eine Hälfte dem neuen Knoten n zu. Die Aufteilung der Zonen erfolgt dabei anhand einer Reihenfolge der Dimensionen. Dies vereinfacht die Aufteilungs- sowie die Zusammenführungsprozedur. Letzlich kopiert d alle Daten aus dieser neuen Zone zu n. Knoten d schickt n seine Nachbarschaftsinformationen und trägt diesen selbst als neuen Nachbarn ein. Knoten n teilt seine Anwesenheit sofort seinen neuen Nachbarn mit. Der Neueintritt eines Knotens ist damit auf ein paar Nachrichten zwischen der Nachbarn begrenzt und beineträchtigt das übrige Netzwerk nicht. Über periodische \emph{UPDATE}-Nachrichten halten sich Nachbarn stets aktuell und senden ihre eigene Nachbarschaftsinformationen an ihre Nachbarn.\\
In \Fref{fig:can_new_node} ist solch ein Eintritt für Knoten N7 aufgezeigt. Nach Teilung der Zone von N4 verändert sich die Nachbarschaft von N1: Knoten N7 kommt neu hinzu.

Ausstehende \emph{UPDATE}-Nachrichten oder Timeouts weisen auf ausgefallene Knoten hin. Werden diese zum Routen einer Nachricht benutzt stellt dies keine direkte Beeinträchtigung des Netzes dar, da die Nachricht über einen anderen Nachbarn verschickt werden kann. So könnte N1 auch N2 zum Nachrichtenversand nutzen, wenn N5 ausgefallen wäre (vergleiche \Fref{fig:can_routing}).

Ein Knoten der einen ausfallenden Peer entdeckt hat, startet einen Timer, dessen Dauer proportial zur Zonengröße ist. Nach Ablauf sender der Knoten eine spezielle \emph{TAKEOVER}-Nachricht an alle Nachbarn des ausgefallen Knotens\footnote{Dieses Wissen ist durch vorhergegangenge \emph{UPDATE}-Nachrichten bekannt.}. Erreicht eine \emph{TAKEOVER}-Nachricht einen Knoten, stoppt dieser seinen eigenen Timer falls seine Zone größer als die des Sender ist. Andernfalls antwortet er selbst mit seiner \emph{TAKEOVER}-Nachricht. Auf diese Weise wird der Nachbar mit der kleinsten Zone gefunden. Dieser ist nun der neue Besitzer der Zone und fügt seine beiden Zonen zusammen wie es in \Fref{fig:can_new_node} am Beispiel von N2 und N3 ersichtlich ist. N2 (rot dargestellt) ist ausgefallen und N3 vergrößert seine Zone. N2 ist nun nicht mehr in der Nachbarschaft von N1 enthalten. Es ist möglich, dass ein Knoten Besitzer zweier Zonen wird, die nicht zusammenfügbar sind. (Ein Beispiel hierfür ist Knoten N4 der beim Ausfall von N5 dessen Zone übernimmt.) Ein Hintergrundprozess defragmentiert solche Zonen und weist Knoten gegebenenfalls neue Koordinaten zu.\\
Damit es zu keinem Ausfall von Daten kommt, wird eine periodische Auffrischung der Daten vorgeschlagen.

Möchte ein Knoten l das System verlassen, so sucht er einen Nachbarn mit der kleinsten Zone und sendet diesem seine Daten. Dieser Nachbar informiert nun die alten Nachbarn von l über die geänderte Nachbarschaft.

\subsection{Nutzbarkeit als Basis für \ac{cast}}
\paragraph{Chord}

\paragraph{Pastry / Tapestry}
We also show that multicast trees built using Pastry provide higher performance than ones built using CAN \cite{Castro2003Evaluation, KostasKatrinis2005}.

\paragraph{CAN}
\cite{Ratnasamy2001}

\subsection{Fazit}
Eine generelle Übersicht der Systeme stellt \Fref{tab:evaluation_fazit} dar. Jedes System hat eigene Schwächen und Stärken. So bezahlt CAN einen günstigen Ein- und Austritt der Knoten dank der kleinen Nachbarschaft mit mehr Routing Hops als beispielsweise Chord. Dieses hat durch seine Fingertabelle wiederum mehr Einfluss auf das Routing - aber aktualisiert diese Tabellen teilsweise nachträglich im Hintergrund während dies bei Pastry aktiv geschieht.

\Fref{tab:evaluation_fazit} (Auszug aus \cite{Goetz2005}) listet die durchschnittlich anfallenden Kosten (Routing Hops, Größe der Routinginformation, Nachrichtenanzahl beim Ein- und Austritt) für die drei getesten System auf.

\begin{table}[htbp]
\centering
\begin{tabular}{l|c|c|c|c}
 & Routing Hops & Routinginformation & Eintritt & Austritt\\ \hline  
Chord & $O(\frac{1}{2}log~n)$ & $O(2log~n) $ & $ O(log^2 n) $ & $ O(log^2 n) $ \\
Pastry & $O(\frac{1}{b}log~n)$ & $O(\frac{1}{b} (2^b-1) log~n) $ & $ O(log_{2^b}~n) $ & $ O(mlog_b~n) $ \\
CAN & $O(\frac{d}{2}n^\frac{1}{d})	$ & $O(2 d) $ & $ O(\frac{d}{2}N^\frac{1}{d}) $ & $ O(2 d) $
\end{tabular}
\caption{Vergleich der Systeme Chord, Pastry und CAN anhand einiger Gesichtspunkte. Wenn nicht anders angegeben bezeichnet $log$ $log_2$. $n$ ist die Anzahl der Knoten, $b$ die Anzahl der darstellenden Basis bei Pastry und $d$ die Anzahl der Dimensionen bei CAN. (aus \cite{Goetz2005})}
\label{tab:evaluation_fazit}
\end{table}

Xu untersucht den Zusammenhang zwischen Größe der Routinginformationen und Anzahl der Routinghops. Für uniformes Routing (d.h. das Routing ist bei allen Peers gleich) sind $O(log~n)$ beziehungsweise $O(n\frac{1}{d})$ Hops für Routingtabellen der Größe $O(log~n)$ beziehungsweise $O(d)$ die asymtothischen Grenze \cite{Xu2004Fundamental}. Die vorgestellen \ac{dht}-basierten Netzwerke nutzen ein uniformes Routing.


Ein einfaches Ranking der untersuchten Systeme ordnet diese anhand der eingangs gestellten Anforderungen.

\paragraph{Geringe Latenz und Kommunikation über das Netzwerk}
\begin{enumerate*}
\item Pastry/Tapestry
\item Chord
\item CAN
\end{enumerate*}
Bei Pastry/Tapestry sind die kürzesten Routen und damit auch oftmals geringste Latenz im Nachrichtenversand zu erwarten, da die Routingtabelle im Vergleich zu Chord und CAN mehr Einträge enthält und diese einfacher mit - aus Netzwerksicht - nahen Peers belegt werden kann. Bei CAN hingegen wird das langsamste Routing erwartet, da Nachrichten nur zwischen benachbarten Knoten ausgetauscht werden kann. Sprünge (via Fingertabellen) sind nicht vorgesehen. Die Erwartungen decken sich mit den Werten in \Fref{tab:evaluation_fazit}.

Li stellt für verschiedene Parametereinstellungen der Netzwerke deren Bandbreite und Latenz gegenüber und untersucht dabei das Verhalten
bei gehäuften Ein- und Austritten von Knoten. Hier ist Chord leicht im Vorteil, da lediglich der Verweis auf den Nachfolgeknoten für das korrekte Routing erforderlich ist. Bei allen Netzwerken pendelt sich die Latenz im \emph{worst cast} auf 250ms ein \cite{Li2004Comparing}.

Die Vorteile der Kommunikation bei Pastry/Tapestry überwiegen für unseren Anwendungsfall.

\missing{Oh, das ist schlecht}


\paragraph{Skalierbarkeit}
\begin{enumerate*}
\item CAN
\item Pastry/Tapestry
\item Chord
\end{enumerate*}
CAN steht hier an erster Stelle, da Ein- und Austritte von Peers nur wenige Knoten im Netz betreffen. Auch bei vielen Ein- und Austritten leidet das Netz unter keiner großen Nachrichtenlast. Allerdings leidet die Kommunikation in großen Netzen (siehe obigen Punkt). Chord liegt auf dem letzten Platz, da das Netzwerk erst durch später aufgerufene Methoden vollkommen funktionsfähig wird (siehe \emph{Eintritt und Austritt (Fehlerfall) von Knoten}). Hier stellt sich nun die Frage, warum ein neuer Knoten $n$ nicht beim Eintritt seinen Nachfolger $p$ nach dessen altem Vorgänger fragt und somit die Zuständigkeit der Daten gleich beim Eintritt klärt? Eine mögliche Antwort ist, dass dadurch viel Overhead durch das Umkopieren von Daten entstehen würden, ohne dass dadurch das Routing bzw. Auffinden der Daten merklich verbessert würde. Im Falle von häufigen Eintritt und Austritt von Knoten würde das Netz lahm gelegt. Pastry und Tapestry liegen auf zweitem Platz; der Aufbau der Routingtabelle erfolgt in vielen kleinen Schritten - dafür ist ein Knoten danach ein vollwertiger Peer im System.

\paragraph{Fehlertoleranz bei Knotenausfall}
\begin{enumerate*}
\item Pastry/Tapestry, CAN
\item Chord
\end{enumerate*}


\paragraph{Bestimmung der Nachbarschaft}
\begin{enumerate*}
\item Pastry/Tapestry
\item Chord
\item CAN
\end{enumerate*}
Allein die Größe der Routingtabelle bedingt, dass bei Pastry und Tapestry mehr Einfluss auf die Zusammenstellung genommen werden kann. Bei CAN gibt es faktisch nur eine Entscheidung bei Eintritt in das Netz, Chord bietet bietet über die Fingertabelle minimaleren Einfluss, währen bei Pastry explizit das Neighborhood Set eingesetzt wird um eventuelle Lücken in der Routingtabelle geschickt zu besetzen.

\paragraph{Eingriff in Routingentscheidungen}
\begin{enumerate*}
\item Pastry/Tapestry
\item Chord
\item CAN
\end{enumerate*}
Das Routing bei CAN kann nur bedingt beeinflusst werden: An welchen Nachbarn soll die Nachricht geschickt werden. Chord hingegen bietet mit seiner Fingertabelle mehr Variationsmöglichkeiten - allerdings nur für die zu überbrückende Distanz im Schlüsselraum. Pastry und Tapestry verbinden diese Variationen mit vielfältigen Nachbarschaftsoptionen (siehe oben).

\paragraph{Unterbau für \ac{cast}}
\begin{enumerate*}
\item Pastry/Tapestry
\item CAN
\item Chord
\end{enumerate*}
Pastry und Tapestry zeigen durch die Implementierung von Scribe und Bayeux, dass sie als Unterbau für \ac{cast} sehr wohl geeignet sind.



Aufgrund der mageren Ergebnisse in Castros Untersuchung \cite{Castro2003Evaluation} bleibt CAN trotz interessanter Ideen außen vor. Ebenso kann Chord nicht in Betracht gezogen werden, da es kurzzeitige Inkonsistenzen in der Nachbarschaft (und damit dem Routing) geben kann. Weiterhin wird eine größere Latenz im Nachrichtenversand erwartet, da Nachrichten nur in einer Richtung auf dem Ring weitergeleitet werden.

 Pastry ist jedoch nur als Javabibliothek\footnote{http://www.freepastry.org} verfügbar und die Entwicklung von Tapestry (ebenfalls in Java implementiert) wurde mit Version 2.01 eingestellt.

Chimera ist der Nachfolger von Tapestry und vereint laut Homepage\footnote{http://current.cs.ucsb.edu/projects/chimera/index.html} das Beste von Pastry und Tapestry in sich: 
\missing{PFUI! Lieber schreiben, dass es keine Publikationen gibt, weil es keine neue Forschung ist, sondern einfach nur ne Neuimplementierung die Sachen aufgreift!!}

\begin{quote}
Chimera is a light-weight C implementation of a \emph{next-generation} structured overlay that provides similar functionality as prefix-routing protocols Tapestry and Pastry.  Chimera gains simplicity and robustness from its use of Pastry's leafsets, and efficient routing from Tapestry's locality algorithms.  In addition to these properties, Chimera also provides efficient detection of node and network failures, and reroutes messages around them to maintain connectivity and throughput.  
\end{quote}

Der frei verfügbare Code (veröffentlicht unter GPL), die Anpassbarkeit und die Unterstützung der Zielplattformen Linux und Windows sprechen stark für Chimera als Netzwerkunterbau für das in dieser Arbeit zu entwickelnde generische Publish/Subscribe-System.\\
Da Chimera der generischen \ac{api} entspricht, könnte es bei gravierenden Problemen durch ein anderes System ausgetauscht werden, ohne das restliche System zu beeinflussen.

\chapter{Konzeption des Frameworks}
\label{chap:konzeption_pubsub}
Diese Arbeit entwickelt ein Framework auf Basis eines strukturierten P2P-Overlay-Netzwerkes. Die im System nutzbaren Publish/Subscribe-Systeme müssen ebenfalls verteilt arbeiten können. Dabei spielen die Fehlertoleranz und auch die Fähigkeit mit \emph{churn}, d.h. schnellem Wechsel der Mitgliedschaften, umgehen zu können, eine große Rolle.

\emph{Nicht vergessen}: Auf Knoten die Nachrichten weiterleiten wird \emph{forward} aufgerufen. Auf Knoten die eine Nachricht empfangen wird zuerst \emph{forward} und dann \emph{deliver} aufgerufen. In \emph{forward} kann der Knoten den Nachrichtenversand beenden. Vgl. \Fref{chap:evaluation_p2p:generic_api}!

Die Dimension \emph{Routing} entscheidet darüber ob und welche Nachrichten Typen in \emph{forward} oder \emph{deliver} behandelt werden sollen.\\
\emph{publish}-Nachrichten werden nicht in \emph{forward} sondern nur in \emph{deliver} behandelt.

Die Dimension \emph{Filter} muss sicherstellen, dass eine \emph{publish}-Nachricht nur an diejenigen Knoten geht, die diese Nachricht auch empfangen wollen. Muss eine solche Nachricht erst zum RootKnoten wandern (bsp. bei Direct oder Multicast) so, darf sie natürlich nicht gefiltert werden!\\
Dies bedeutet auch, dass bei der Auslieferung einer \emph{publish}-Nachricht nicht mehr gefiltert werden muss. Muss eine solche Nachricht jedoch weiter verteilt werden (bsp: Multicast), dann muss wie oben erwähnt gefiltert werden!

\begin{table}[!h]
\label{tab:konzeption_pubsub:verbindungsmatrix}
\resizebox{\textwidth}{!}{%
\begin{tabular}{llccccccc}
\toprule
Nachrichten- & Upcall	& \multicolumn{7}{c}{Dimension} \\
\cmidrule{3-9}
typ				&		& Routing & Filter & Deliver & Order & Persistence & Security & Validity \\
\midrule
publish	    & deliver & + & + & + & + & + & + & + \\
\cmidrule{2-9}
					  & forward & + & + & + &   &   & + & + \\
\midrule
subscribe	  & deliver & + & + &   &   &   & + & \\
\cmidrule{2-9}
			      & forward & + & + &   &   &   & + & \\
\midrule
unsubscribe & deliver & + & + &   &   &   & + & \\
\cmidrule{2-9}
      & forward & + & + &   &   &   & + & \\
\bottomrule
\end{tabular}}
\caption{Verbindungsmatrix}
\end{table}






\cite{BeFiMu2006PubSubQoS}


\cite{KostasKatrinis2005}

\chapter{Related Work}
\label{chap:related}

IPMulticast ist das Beste, Application Multicast ist toll, aber OverlayMulticast ist besser. OverlayMulticast: explizites Erstellen von RoutingKnoten. Widerspricht aber unserem Ansatz, so wenig \enquote{zentrale Server} wie möglich zu haben. 

\cite{Lao2005Comparative} % Multicast Protocols: Top, Bottom, or In the Middle

\section{NTree}
\label{chap:related:ntree}



\section{minTCO}
\label{chap:related:mintco}
\cite{citeulike:4069017}


\section{Vivaldi}
\label{chap:related:vivaldi}
Vivaldi \cite{citeulike:162250} ist ein dezentrales System, in dem sich Knoten des Netzwerkes Koordinaten so wählen, dass der Abstand zwischen zwei Koordinaten der Latenz zwischen diesen beiden Knoten entspricht.

\subsection{Arbeitsweise}
Neue Knoten im Netz wählen sich zufällig Koordinaten. Zu allen Nachbarn werden nun Ping-Nachrichten zur Messung der Laufzeit verschickt. Die Koordinaten der einzelnen Knoten werden dabei übertragen. So kann nun die eigene Koordinate anhand des Abstandes angepasst werden.

Das Paper verwendet hier eine Analogie zu gespannten Federn. Alle Knoten sind über Federn miteinander verbunden und suchen nun das gemeinsame Optimum der Federspannung.


\section{VON}
\label{chap:related:von}
\ac{von} ist ein integriertes System aus Overlaynetzwerk mit Publish/Subscribe-System und zielt direkt auf virtuelle Umgebungen wie \ac{mmog} ab \cite{Hu2006VON}. Jeder Knoten hat demnach eine Koordinate im Raum\footnote{z.B. Spielewelt} und eine \ac{aoi}. Die virtuelle Welt wird mittels Voronoi-Diagrammen in $n$ (Anzahl der Knoten im System) disjunkte Regionen eingeteilt und dadurch auf die Knoten verteilt. Ein Knoten hat direkte und angrenzende Nachbarn. Bewegt sich der Knoten im Raum, so verändert sich die Aufteilung der Regionen.

\ac{vast} \cite{Backhaus2007Voronoibased} greift das Konzept von \ac{von} auf und testet eine Implementierung auf OpenSIM \cite{Baumgart2007OverSim}.


\chapter{Zusammenfassung und Ausblick} 
\label{chap:zus}


\appendix

\chapter{Anhang}
\label{chap:anhang}
\chapter{Moderne Entwicklung mit C++}
\label{chap:impl_tmp}
\acf{tmp}\index{Template Meta-Programming} und policy-based Design\index{policy-based Design} \cite{Alexandrescu2001Modern} sind Paradigmen der Software-Entwicklung in C++ und bedienen sich der dort verfügbaren \emph{Templates}. Policy-based Design baut in der hier erklärten Variante auf dem turingvollständigen \ac{tmp} auf.

\section{Templates}
Templates dienen der Generalisierung von Code und werden zur Übersetzungszeit durch den Compiler ausgewertet. Templates sind ein integraler Bestandteil der \ac{stl}\footnote{Siehe auch \url{http://www.cppreference.com/wiki/} (26.\,11.\,2010)} und dienen dort vor allem zum Erstellen abstrakter Containerklassen wie zum Beispiel \texttt{std::vector}, \texttt{std::map} oder \texttt{std::set}. Basierend auf dem Konzept des Zugriffs über \texttt{Iteratoren} sind viele aus der Funktionalen Programmierung bekannte Funktionen implementiert, die ebenfalls durch Templates generisch gehalten sind.

Einzelne Funktionen oder komplette Klassen könnten als Templates realisiert werden. Zusätzlich zu der generischen Implementierung können auch spezielle Implementierungen, sog. Spezialisierungen, angegeben werden. Funktionstemplates lassen sich nur total, Klassentemplates auch partiell spezialisieren. Eine partielle Spezialisierung kann bei mehreren Templateparametern erfolgen. Ungenutzte Templates werden vom Compiler nicht instantiiert, dies bedeutet, dass kein ungenutzter Code im fertigen Kompilat vorhanden ist. Dies wiederum verringert die Größe ausführbarer Dateien oder Bibliotheken. Jedoch muss erwähnt werden, dass jede Instanz eines Templates eigenen Code generiert. Dieser lässt sich -- dank des Wissens zur Übersetzungszeit -- besser optimieren als zur Laufzeit.

In diesem und den folgenden Beispielen soll ein Einblick in die neue Methodik gezeigt werden, daher wird nicht auf eine optimierte Parameterübergabe oder Ähnliches geachtet. \Fref{lst:tmp_easy_spez} zeigt ein einfaches Template. Der Funktion \texttt{min} wird als Templateparameter der Typ der Parameter übergeben. Es wird erwartet, dass die übergebenen Typen den Operator $<$ überladen haben. Für den Typ \texttt{Pair} ist das Template total spezialisiert und eine separate Implementierung wurde angegeben. Zeilen 18 und 19 zeigen die einzelnen Funktionsaufrufe.

\lstinputlisting[caption={Einfaches Funktionstemplate}, label=lst:tmp_easy_spez, float=t]{listings/tmp_easy_spez.cpp}

Ebenso können Templates geschachtelt übergeben werden, wie es \Fref{lst:tmp_tmp} zeigt. Hier wird dem \texttt{WidgetFactory} ein Template übergeben mit dessen Hilfe er Objekte des Typs \texttt{Widget} erstellen kann. Beim Aufruf in Zeile 19 muss der Entwickler dennoch den Typ angeben, obwohl eine \texttt{WidgetFactory} nur Objekte des Typs \texttt{Widget} erzeugt, wie es der Name impliziert.

\lstinputlisting[caption={Einfaches Klassentemplate}, label=lst:tmp_tmp, float=t]{listings/tmp_tmp.cpp}

\Fref{lst:tmp_tmp_2} zeigt die Lösung mit \enquote{Template Template}-Parametern. Hier wird der \texttt{WidgetFactory} ein Template übergeben. Dieses Template ist noch nicht spezifiziert und benötigt weiterhin den Typ des zu erstellenden Objektes. Der Unterschied zum vorherigen Listing ist in Zeile 5 und am Aufruf in Zeile 10 zu sehen. \texttt{WidgetFactory} übergibt den Typ an das Template. Der Entwickler ist nicht mehr für die Auswahl des korrekten Typs zuständig.

\lstinputlisting[caption={Verbessertes Klassentemplate mit Template Template}, label=lst:tmp_tmp_2, float=!t]{listings/tmp_tmp_2.cpp}

Mit Templates lassen sich auch Entscheidungen anhand der übergebenen Templateparameter treffen. In \Fref{lst:tmp_cond} wird anhand des Parameters \texttt{isSingleton} entschieden, ob das neue Objekt mit \texttt{new} oder per Aufruf der Methode \texttt{T::getInstance()} erzeugt werden kann.

\lstinputlisting[caption={Entscheidung via Templates}, label=lst:tmp_cond, float=t]{listings/tmp_cond.cpp}

Der Compiler prüft dabei alle Templates syntaktisch. Obwohl \texttt{isSingleton} mit \texttt{false} instantiert wird, kann der Code nicht kompiliert werden, denn die Methode \texttt{getInstance} wird von \texttt{Foo} nicht angeboten. Um dieses Problem zu umgehen, beschreibt Alexandrescu in \cite{Alexandrescu2001Modern} das Hilfskonstrukt \texttt{Int2Type}, das im nächsten Kapitel eingeführt wird.

\section{Template Meta-Programming}
Template Meta-Programming arbeitet mit der vorgestellten Spezialisierung von Templates und wird ebenfalls vom Compiler zum Übersetzungszeitpunkt ausgewählt und basiert rein auf Typinformation.

\ac{tmp} kann dazu genutzt werden, die Problematik im vorangegangenen Kapitel zu lösen. \Fref{lst:tmp_cond_2} zeigt die Definition von \texttt{Int2Type}. \texttt{Int2Type} speichert den übergebenen Integerwert und erzeugt daraus pro Wert einen eigenen Typ. Mit überladenen Funktionen kann nun eine Auswahl zur Übersetzungszeit erfolgen. Die statische Methode \texttt{create} (Zeile 8) ruft nun abhängig vom Templateparameter \texttt{isSingleton} die überladene Methode \texttt{create\_impl} auf. Im ersten Fall (Zeile 14) nimmt diese ein \texttt{Int2Type$<$true$>$}, im zweiten Falle (Zeile 19) nur \texttt{Int2Type$<$false$>$}. Da der Compiler nur instantiierte Templates prüfen muss, gibt es nun keine Übersetzungsfehler mehr. Der Entwickler muss seinen Aufruf nicht anpassen.

\lstinputlisting[caption={Entscheidung via Templates zur Übersetzungszeit}, label=lst:tmp_cond_2, float=t]{listings/tmp_cond_2.cpp}

Durch diese Technik können Entscheidungen zur Übersetzungszeit getroffen werden. So können zum Beispiel auch komplexe Berechnungen zur Übersetzungszeit ein- beziehungsweise ausgeschaltet werden. Der zusätzliche Code, hier Methodenaufrufe, kann durch den Compiler leicht optimiert werden.

\section{Policy-based Design}
Policy-based Design kombiniert Vererbung und Templates und ermöglicht es, generischen Code zu entwickeln. Hierbei lassen sich sehr gut orthogonale Aspekte ausnutzen, wie Alexandrescu am Beispiel eines SmartPointers erläutert.

In \Fref{lst:tmp_pbd} ist ein einfaches Beispiel mit einer Ausgabeklasse (\texttt{Ausgabe}) beschrieben, die mittels den Klassen \texttt{Decorator} und \texttt{Printer} parametrisiert wird. \texttt{Ausgabe} leitet von diesen beiden Typen ab und nutzt deren Funktionalität. Dadurch erzeugt der Compiler implizit Anforderungen, die diese Parameter zu erfüllen haben. Jeder Decorator muss eine Methode \texttt{decorate} anbieten, welche einen \texttt{const string\&} als Parameter nimmt. Jeder \texttt{Printer} muss eine Methode \texttt{print} anbieten, welche den Rückgabetyp von \texttt{Decorator::decorate} als Parameter nimmt. Diese Kontrakte werden vom Compiler zur Übersetzungszeit in Zeile 24 geprüft. Weitere Varianten für \texttt{Decorator} und \texttt{Printer} sind trivial.

\texttt{Ausgabe} wird in diesem Zusammenhang als \enquote{Host-Klasse}, \texttt{Dec\_A} und \texttt{Printer} als \enquote{Policy-Klasse} bezeichnet. Da die Host-Klasse von ihren Policy-Klassen ableitet, kann die Host-Klasse in eine Policy-Klasse gecastet werden. So können nützliche Informationen abgefragt oder speziell angepasste Methoden der Policy-Klasse aufgerufen werden. Der Nutzer einer solchen mit Policies angereicherten Klasse bekommt dadurch sehr viele Freiheiten. Mögliche Sonderfälle spricht Alexandrescu an und bietet angepasste Lösungen.

\lstinputlisting[caption={Beispiel für policy-based Design}, label=lst:tmp_pbd, float=t]{listings/tmp_pbd.cpp}

Diese Möglichkeit der Parametrisierung von Klassen gibt es prinzipell auch zur Laufzeit. Hier wird jede Policy durch einen Pointer auf eine abstrakte Basisklasse abgebildet. Jedoch hat dies einige Nachteile bei der Laufzeit. Zum einen sind die Interfaces der Policy-Klassen festgeschrieben und die Host-Klasse leitet nicht von diesen ab. Zusätzlich sind virtuelle Funktionsaufrufe \enquote{teuer}, da diese in der \enquote{VTable} zur Laufzeit abgeprüft werden müssen. Diese Überprüfung kann durch den Compiler nicht entfernt oder optimiert werden. 


In dieser Arbeit kommt \ac{tmp} im Zusammenspiel mit policy-based Design bei der Entwicklung des einzelnen Kanals zur Anwendung. Durch die Optimierungen zur Übersetzungszeit wird schlanker Code ohne unnötigen Ballast zur Laufzeit erzeugt.


\chapter{Distributed Hashtable (DHT)}
\label{chap:dht}

Das System der \acf{dht} ist nach \cite{Wehrle2005} eine Lösung des \emph{lookup}-Problemes\footnote{``Wo speichere beziehungsweise finde ich Datensatz X?''} in verteilen Systemen ist und besticht dabei durch ihre Einfachheit, Robustheit und Skalierbarkeit. \acp{dht} können einerseits zur Speicherung von Datensätzen nach dem \emph{(key,value)}-Ansatz genutzt werden oder zum Routen von Nachrichten durch das Netzwerk. Beide Varianten schließen sich nicht gegenseitig aus. Für das System ist eine Hashfunktion $h$ definiert, so dass alle Eingaben eindeutig auf einen Wert aus dem \emph{Schlüsselraum} $S$ abgebildet werden: $h(x) \rightarrow y; y \in S$. Eine \emph{segmentierte} Hashfunktion bildet den Eingabewert auf $n$ eindeutige Schlüssel ab. Dabei ist der Schlüsselraum in $n$ disjunkte Bereiche aufteilt: $h_2(x) \rightarrow y,w; y \in S_1, w \in S_2, S := S_1 \cup S_2$. Die Wertemenge des Schlüsselraumes besteht dabei meist aus einem Intervall aus Ganzzahlen, beispielsweise von $0$ bis $2^{160}-1$. Die Hashfunktion bildet dabei die Eingaben möglichst auf den ganzen Wertebereich ab.

Jeder Knoten im Netzwerk bekommt über diese Hashfunktion einen eindeutigen Schlüssel zugeordnet. Als Eingabe der Hashfunktion kann zum Bespiel eine Kombination aus Hostnamen und Port sein. Einem Knoten wird -- je nach Netzwerk -- ein gewisser Bereich aus dem Schlüsselraum zugeordnet. Weiterhin hat jeder Knoten eine gewisse Kenntnis von andern Knoten im Netzwerk -- seine Nachbarschaft. Diese erlangt er durch netzwerkabhängige Verfahren beim Eintritt in das System\footnote{Beispiele sind in \Fref{chap:evaluation_p2p} bei der Vorstellung der verschiedenen Netzwerke Chord, Pastry und CAN gegeben.}. CAN benötigt eine an die Dimensionen angepasste segmentierte Hashfunktion zur Berechnung der Kooridinaten.

Wird das System zur Übertragung von Nachrichten genutzt, so bestet dessen minimale Schnittstelle aus den Methoden \texttt{route(key, message)} zum Senden und \texttt{receive(message)} zum Empfang von Nachrichten\footnote{Siehe \Fref{chap:grundlagen:api}}. Das System kann die Nachricht in jedem Schritt zu einem Knoten senden, der näher -- meist numerisch gesehen -- am Empfänger ist.\\
Wird das System als verteilter Datenspeicher genutzt, so wird für jeden Datensatz die selbe Hashfunktion genutzt. Beispielweise wird der Dateiname oder Dateiinhalt an die Hashfunktion übergeben. Damit kann für jeden Datensatz ein eindeutiger Schlüssel berechnet und damit der zuständige Knoten bestimmt werden. Dieser speichert den Datensatz oder hält Informationen bereit um auf den Datensatz zugreifen zu können. Die minimale Schnittstelle eines solchen Systems beschränkt sich auf die Methoden \texttt{put(key, value)} zum Speichern eines Datensatzes und \texttt{get(key)} zur Abfrage.



% CONTENT END

\cleardoublepage

\automark[chapter]{section}

% Index soll Stichwortverzeichnis heissen
\newpage
\renewcommand{\indexname}{Stichwortverzeichnis}
	 
% Stichwortverzeichnis soll im Inhaltsverzeichnis auftauchen
\phantomsection
\addcontentsline{toc}{chapter}{\indexname}

\markleft{\indexname}
\markright{\indexname}
\printindex

\automark[chapter]{section}

\bibliographystyle{alphadin}
\bibliography{bib/literatur}
%%% Wenn folgende Zeile nicht auskommentiert wird, steht alles im Litverzeichnis
%\nocite{*}
\end{document}
