\chapter*{Abstract}

\selectlanguage{english}


\selectlanguage{ngerman} 
\clearpage{\pagestyle{empty}\cleardoublepage}
\chapter*{Kurzfassung}
Massively Multiplayer Online Games (MMOG) oder allgemeiner Massive Multiuser Virtual Environments (MMVE) sind häufig nach dem Client-Server-Prinzip aufgebaut. Die Clients, Teilnehmer der virtuellen Welt, verbinden sich mit einem oder mehreren Servern des Anbieters, auf denen die Welt verwaltet wird. Verschiedene Entwicklungen versuchen die Kommunikation über ein p2p-Netzwerk abzuwickeln und somit die virtuelle Welt durch die einzelnen Teilnehmer verwalten zu lassen. \textbf{M}assive \textbf{M}ultiuser \textbf{E}ven\textbf{t} \textbf{I}nfra\textbf{S}tructure (M$^2$etis) will diesen Ansatz durch ein optimiertes kanalbasiertes Publish/Subscribe-System erweitern. Die vorkommenden Eventtypen des MMVE bestimmen die Kanäle des Systems. Anhand der semantischen Beschreibung der Events werden die Kanäle optimiert und bieten dadurch eine auf die virtuelle Welt optimal zugeschnittene Eventverteilung. Dieser Arbeit behandelt die Anbindung des Netzwerkes und die Konzeption der zur Übersetzungzeit optimierbareren Publish/Subscribe-Komponente. Hierzu werden die Grundlagen von Publish/Subscribe-Systemen und p2p-Netzwerken beschrieben. Es wurden verschiedene strukturierte p2p-Netzwerke vorgestellt und evaluiert. Pastry wurde als geeignetes Netzwerk für M$^2$etis ausgewählt. Zur Umsetzung der prototypischen Implementierung der Publish/Subscribe-Komponente sind Methoden der modernen C++-Entwicklung wie Template Meta-Programmierung (TMP) eingesetzt worden, die es erlauben das System anhand des zur Übersetzungzeit vorhandenen Wissens zu optimieren.

