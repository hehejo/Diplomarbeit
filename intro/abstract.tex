\chapter*{Abstract}
\selectlanguage{english}
\missing{Bitte die deutsche Kurzfassung lesen. Übersetzung ``in progress''.}


\selectlanguage{ngerman} 
\clearpage{\pagestyle{empty}\cleardoublepage}
\chapter*{Kurzfassung}
Viele Computerspiele werden als \acfp{mmog} über das Internet gespielt und bringen die Spieler dabei in einer virtuellen Welt zusammen. Die Kommunikation innerhalb dieses \acfp{mmve} geschieht meist über eine Client/Server-Architektur, die jedoch ob der benötigten Bandbreite pro Client an ihre Grenzen gerät. Ein Lösungsansatz ist das Aufteilen der Welt in unabhängige \emph{Shards}, das sind Instanzen des Spiels die auf verschiedenen Servern beziehungsweise Servercluster laufen. Dies schmälert möglicherweise das Spielvergnügen, wenn beispielsweise befreundete Spieler in verschiedenen \emph{Shards} spielen und trotz virtueller Welt nicht miteinander interagieren können.\\
\ac{m2etis} nimmt sich dieser Problematik an und strebt nach neuen Optimierungsmöglichkeiten der Kommunikation mittels \ac{p2p}-Netzwerke. Statt -- wie einige andere Ansätze -- die Nachrichtenverteilung lediglich auf die Behandlung eines Eventtypen hin zu optimieren, berücksichtigt \ac{m2etis} alle vorkommenden Eventtypen und optimiert deren Verteilung anhand ihrer semantische Klassifizierung in einem kanalbasierten, verteilten Publish/Subscribe-System. Die Verbindung aus \ac{p2p}-Netzwerk, kanalbasiertem Publish/Subscribe und der semantischen Beschreibung von Eventtypen ermöglicht neue Optimierungsdimensionen der Eventverteilung in \acp{mmve}.

Diese Arbeit entwickelt die erste Komponente von \ac{m2etis}: Das grundlegende Framework des optimierbaren Publish/Subscribe-Systems sowie dessen Verbindung mit dem \ac{p2p}-Netzwerk.
