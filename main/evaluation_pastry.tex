\subsection{Pastry/Tapestry}
\label{chap:evaluation_pastry}

\subsubsection{Aufbau und Struktur}
Pastry \cite{Rowstron2001} und Tapestry \cite{Zhao2001Tapestry,Zhao2004Tapestry} sind sich sehr ähnlich, da beide auf Plaxtons Arbeit \cite{Plaxton1997Accessing} aufbauen. Daher wird auf Tapestry nicht näher eingegangen, da die Unterschiede für das Konzept dieser Systeme nicht relevant sind.

Pastry besitzt ebenfalls einen $l$ bit-wertigen Schlüsselraum, dabei werden Schlüssel als Zahlen zur Basis $2^b$ dargestellt, wobei die Wahl von $b$ einen Einfluss auf das Routing hat, da dieser die Größe der Routingtabelle beeinflusst. Ein Datensatz ist dem Knoten zugewiesen, dessen ID den kleinsten Abstand zum Schlüsselwert des Datensatzes hat.\\
\Fref{fig:pastry_key_space} zeigt dies beispielhaft für die gleichen sechs Knoten und fünf Datensätzen wie in \Fref[plain]{fig:chord_key_space}. Im Unterschied zu Chord ist jedoch Knoten $N14$ für $K16$ und Knoten $N54$ für $K55$ zuständig.


\begin{figure}[htbp]
\centering
\includegraphics{grafics/pastry_key_space.pdf}
\caption{Schlüsselraum für Pastry mit sechs Knoten ($Nx$) und fünf Daten ($Kx$)}
\label{fig:pastry_key_space}
\end{figure}

\subsubsection{Routing}
Jeder Knoten verwaltet neben den ihm zugeteilten Daten drei Strukturen die dem Routing dienen. Diese sind das \emph{leaf set} mit Einträgen zu Knoten, die im Schlüsselraum benachbart sind, das \emph{neighborhood set} mit Einträgen zu Knoten, die aus Netzwerksicht nahe liegen, und die Routingtabelle selbst. In \Fref{fig:pastry_routing_table} sind nur die Schlüssel, nicht aber Kontaktinformation, wie IP-Adresse und Port dargestellt. Über diese kann zum Nachrichtenversand eine Direktverbindung genutzt werden.

\begin{figure}[htbp]
\centering
\includegraphics{grafics/pastry_routing_table.pdf}
\caption{Routing table, leaf set und neighborhood set bei Pastry}
\label{fig:pastry_routing_table}
\end{figure}

Die Routingtabelle besteht aus $\frac{l}{b}$ Reihen mit je $2^b -1$ Einträgen. \Fref{fig:pastry_routing_table} zeigt dies beispielhaft für Knoten $103220$ mit $l=12, b=2$ (nach \cite{Goetz2005}). Einträge in Zeile $i$ haben einen Präfix der Länge $i$ mit dem Knoten $103220$ gemein. Die Übereinstimmungen sind in der Abbildung fett hervorgehoben. Ist kein passender Knoten bekannt, wird das entsprechende Feld nicht ausgefüllt. Damit hat die Routingtabelle Ähnlichkeiten zur Fingertabelle bei Chord. Ein Knoten hat ungenaues Wissen über entfernte Knoten. Der Detailgrad an Routinginformationen erhöht sich pro Zeile in der Routingtabelle. Sind im System wenig Knoten vorhanden, sind die letzten Reihen der Routingtabelle spärlich belegt. Im Durchschnitt sind bei $n$ Knoten im System nur $log_{2^b} n$ Einträge ausgefüllt.\\
Bei der Belegung der Routingtabelle werden bei gleichem Präfix diejenigen Knoten gewählt, die aus Netzwerksicht näher sind.

Das leaf set enthält die $l$ numerisch nahen Knoten, $\frac{l}{2}$ davon sind kleiner und $\frac{l}{2}$ größer als der aktuelle Knoten. Neben Informationen zu Routingentscheidungen wird es zur Reparatur genutzt, sollten nahe gelegene Knoten ausfallen.

Das eigentliche Routing unterscheidet zwei Fälle: Zuerst prüft der Knoten ob der Schlüssel $k$ im Bereich seines leaf sets ist. Ist dies der Fall, wird die Nachricht an den entsprechenden Knoten gesendet. Ist dieser Knoten für den Schlüssel zuständig, endet das Routing. Fällt $k$ nicht in den Bereich des leaf sets, wird die Nachricht via Routingtabelle an einen entfernteren Knoten gesendet. Hierzu wird ein Eintrag gewählt, der eine größere beziehungsweise die größte Präfixübereinstimmung mit $k$ hat. Existiert kein solcher Eintrag, wird die Nachricht an den numerisch nächsten Knoten (zu $k$) mit gleicher Präfixübereinstimmung geschickt.\\
Da Nachrichten immer an Knoten mit einer größeren Übereinstimmung oder an nähere Knoten mit gleicher Präfixübereinstimmung gesendet werden, können keine Zyklen auftreten.

Dadurch verringert sich die Anzahl der Knoten mit längeren Präfixübereinstimmungen in jedem Schritt um mindestens den Faktor $2^b$. Somit hat das Routing eine Komplexität von $O(log_{2^b} N)$.

\subsubsection{Nachbarschaft}
Das neighborhood set (siehe \Fref{fig:pastry_routing_table}) enthält $|m|$ Knoten die aus Netzwerksicht nahe sind. Obwohl es im Routing keine Rolle spielt, kann es dazu genutzt werden um geeignete Knoten zu finden. Da die Größe des leaf sets ebenfalls wählbar ist, können hier auch vermehrt nahe Knoten platziert werden.

\subsubsection{Eintritt und Austritt (Fehlerfall) von Knoten}
Einem neuen Knoten $n$ wird von Applikationsseite ein frei wählbarer Schlüssel gegeben. Meist berechnet sich dieser Schlüssel anhand des Hashwerts über die IP oder seines öffentlichen Namens. Weiterhin geht das System davon aus, dass $n$ aus einer Liste bekannter Knoten denjenigen Knoten $x$ wählen kann, der aus Netzwerksicht nahe ist. Von diesem Knoten kann das neighborhood set kopiert werden. Zum Aufbau der Routingtabelle und des leaf sets lässt $n$ via $x$ eine \emph{JOIN}-Nachricht an einen numerisch nahen Schlüssel zu $n$ routen. Diese Nachricht gelangt schließlich zu Knoten $c$, von dem das leaf set (die Liste der $l$ numerisch nahen Knoten) kopiert werden kann. Alle Knoten,, die diese \emph{JOIN}-Nachricht weiterleiten, senden ihre Routingtabelle an $n$. Für jeden Routinghop kopiert sich $n$ die entsprechende Zeile aus der Routingtabelle, da ausgehend von keiner Präfixübereinstimmung mit dem nahen Knoten $x$, jeder weitere Hop eine größere Präfixübereinstimmung bringt.\\
Im Gegensatz zur nachträglichen Aktualisierung bei Chord, wird nun die gesamte Routinginformation an alle bekannten Knoten gesendet. Der neue Knoten ist nun im Netzwerk bekannt und erreichbar.

Ausgefallene Knoten werden ebenfalls anhand von Timeouts beim Routing entdeckt. Da die Einträge des neighborhood sets nicht im Routing involviert sind, müssen diese periodisch geprüft werden. Fehlerhafte Einträge in der Routingtabelle können über einen anderen Eintrag mit gleicher Präfixübereinstimmung kompensiert werden, müssen aber entfernt werden, um ein stabiles und sicheres Routing zu gewährleisten. Hierzu können von benachbarten Einträgen Routinginformationen angefordert werden, um die entstandene Lücke zu füllen. Ein fehlerhafter Eintrag im leaf set oder neighborhood set kann auf ähnliche Weise repariert werden: Hier werden Informationen von den anderen Einträgen im leaf set oder neighborhood set angefordert.

Der Austritt eines Knotens wird vom System wie ein Fehlerfall behandelt. Um die Datenintegrität zu gewährleisten und um unnötigen Nachrichtenversand im System zu vermeiden, sollte die Applikation den Austritt eines Knotens speziell behandeln.

