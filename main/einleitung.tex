\chapter{Einleitung}
\label{chap:einleitung}

Die Spielewelt verändert sich von Einzelspieler bzw. rundenbasierten Gruppenspielen an einem PC dank der fortschreitenden Übertragungstechniken zu im Netzwerk gespielten \ac{mmog}.

Um was geht's hier:\\
Framework auf Basis eines strukturierten p2p-overlay Netzwerkes zur Verteilungsoptimierung von Events in Publish/Subscribe-Systemen.

\section{Motivation}
\begin{itemize*}
\item dezidierte Server
\item Latenz
\item Ausfälle
\item Spielfluss
\end{itemize*}

\begin{itemize*}
\item vielfältige Ansätze \cite{Bharambe2008Donnybrook} %donnybrook, VAST
\item dezentral $\rightarrow$ Overlay Netzwerk
\item Pub/Sub Systeme \cite{Knutsson2004Peertopeer, Triebel2008Peertopeer} %peer2peer support
\end{itemize*}

Also im Grunde ein paar \emph{Motivationen} zusammenschreiben. :-)

\cite{Fischer2010Event, Fischer2010a} % Toms Paper

\section{Aufbau der Arbeit}
Im ersten Kapitel werden nach einer Einführung von \ac{m2etis} die Grundlagen zu Overlay- und p2p-Netzwerken gelegt. Ein Abschnitt über betrügerisches Verhalten und dessen Verhinderung rundet die Einführung ab. Das nächste Kapitel dieser Arbeit widmet sich der Evaluation der drei Overlay-Netzwerke Chord, Pastry/Tapestry und CAN. Das Kapitel über die Konzeption des Frameworks handelt von der Umsetzung der in \cite{Fischer2010Event} identifizierten orthogonalen Dimensionen durch Policies und deren Reihenfolge und Zusammenspiel im Verarbeitungsmodell. Eine Einführung in Template Meta-Programming stellt einige die zur Reduzierung von Overhead zur Laufzeit genutzten Techniken vor und Einblicke in die Implementierung zeigen die Abstraktionsebenen des Frameworks auf.
