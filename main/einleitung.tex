\chapter{Motivation}
\label{chap:einleitung}
Die Computerspiele veränderten sich von Einzelspieler-Spielen beziehungsweise rundenbasierten Gruppenspielen an einem PC dank der fortschreitenden Übertragungs\-techniken und der günstigen Zugangspreise zu im Netzwerk gespielten \ac{mmog}. Diese Spielwelten sind meist nach dem Client-Server Prinzip aufgebaut, bei dem die virtuellen Welten auf Servern des Anbieters gehostet und betrieben werden und die einzelnen Spieler ihre Aktionen in der Welt mit diesen Servern abgleichen. Bei beliebten Spielen steigt die Anzahl der benötigten Server stark an und dennoch wird das Spiel meist in verschiedene Instanzen (sogenannte Shards) aufgeteilt. Dies kann das Spielerlebnis einschränken, da die verschiedenen Instanzen untereinander nicht verbunden sind.

Um diese Trennung in verschiedene Welten aufzubrechen und um die Anzahl der vorzuhaltenden Server zu verringern, gibt es Konzepte dezentrale Techniken wie \ac{p2p}-Netzwerke zu nutzen \cite{Knutsson2004Peertopeer, Triebel2008Peertopeer}. Einige Ansätze wie Donnybrook \cite{Bharambe2008Donnybrook} oder VAST \cite{Backhaus2007Voronoibased}, die Optimierungen auf Basis eines \ac{p2p}-Netzwerkes durchführen, existieren. Jedoch bieten diese nicht die Bandbreite an Optimierung die \ac{m2etis} ermöglichen soll \cite{Fischer2010Event}. Statt das System auf die Behandlung eines Eventtypen -- bei VAST ist dies beispielsweise die Position -- hin zu optimieren, strebt \ac{m2etis} eine Optimierung aller im Spiel genutzten Eventtypen auf Basis eines kanalbasierten Publish/Subscribe-Systems an \cite{Fischer2010a}. Diese Arbeit entwickelt das grundlegende Framework zur Kommunikation mit dem \ac{p2p}-Netzwerk sowie das zur Übersetzungszeit optimierbare Publish/Subscribe-System.

\section*{Aufbau der Arbeit}
Ein einfaches Szenario zeigt typische XXX und verdeutlicht die Ansätze von \ac{m2etis}. Danach wird in die Grundlagen zu Overlay- und p2p-Netzwerken sowie verteilten Publish/Subscribe-Systeme eingeführt. Das nächste Kapitel dieser Arbeit widmet sich der Beschreibung der drei Overlay-Netzwerke Chord, Pastry/Tapestry und CAN und der Auswahl eines für \ac{m2etis} geeigneten Netzwerks. Das Kapitel über die Konzeption des Frameworks zeigt die Umsetzung der in \cite{Fischer2010Event} identifizierten Dimensionen zur Klassifizierung von Eventtypen und deren Reihenfolge und Zusammenspiel im Verarbeitungsmodell. Durch Einblicke in die prototypische Implementierung werden die Abstraktionsebenen des Frameworks und dessen einfache Handhabbarkeit ersichtlich. Die Einführung in \ac{tmp} stellt zudem die Techniken vor die den Overhead zur Laufzeit reduzieren und findet sich im Anhang dieser Arbeit.
