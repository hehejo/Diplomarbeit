\chapter{Einleitung}
\label{chap:einleitung}

Die Spielewelt verändert sich von Einzelspieler bzw. rundenbasierten Gruppenspielen an einem PC dank der fortschreitenden Übertragungstechniken zu im Netzwerk gespielten \ac{mmog}.

Um was geht's hier:\\
Framework auf Basis eines strukurierten p2p-overlay Netwerkes zur Verteilungsoprimierung von Events in Publish/Subscribe-Systemen.

\section{Motivation}
\begin{itemize*}
\item dezidierte Server
\item Latenz
\item Ausfälle
\item Spielfluss
\end{itemize*}

\begin{itemize*}
\item vielfältige Ansätze \cite{Bharambe2008Donnybrook} %donnybrook, VAST
\item dezentral $\rightarrow$ Overlay Netzwerk
\item Pub/Sub Systeme \cite{Knutsson2004Peertopeer, Triebel2008Peertopeer} %peer2peer support
\end{itemize*}

Also im Grunde ein paar \emph{Motivationen} zusammenschreiben. :-)

\section{Aufbau der Arbeit}
Zuerst Grundlagen mit kurzer Begriffserklärung, Einführung in \ac{m2etis},  p2p-Netzwerke und Eventsystem anhand Publish/Subscribe und dessen mögliche Umsetzung in p2p-Netzwerken. In \Fref{chap:evaluation_p2p} werden verschiedene p2p-overlay Netzwerke vorgestellt und ihn ihren Arbeitsweisen einander gegenübergestellt.
