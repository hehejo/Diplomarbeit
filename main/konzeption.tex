\chapter[Konzeption der Publish/Subscribe-Komponente]{Konzeption der Publish/Subscribe-Komponente von M$^2$etis}
\label{chap:konzeption_pubsub}
In diesem Kapitel wird die Konzeption der ersten Komponente zur Verteilungsoptimierung in ihren Einzelheiten erläutert.\\
Ausgehend von den in \Fref[plain]{chap:grundlagen:event} identifizierten, orthogonalen Dimensionen zur Optimierung einzelner Eventtypen wird die, im nächsten Abschnitt beschriebene, Problemstellung greifbar. Das Framework muss die vielfältige Anpassung einzelner Kanäle ermöglichen ohne dies mit Einbußen zur Laufzeit zu erkaufen. Ein weiterer wichtiger Fokus dieser Arbeit ist es, \ac{m2etis} als ein einfach zu benutzendes kanalbasiertes Publish/Subscribe-System zu präsentieren.


\section{Aufbau von M$^2$etis}
\label{chap:aufbau_metis}

Dieser Abschnitt erläutert den Aufbau von \ac{m2etis} als ein Framework für \acp{mmve}. Das Ziel ist die Verteilung der verschiedenen Events zu optimieren. Dabei wird nicht der klassische Ansatz einer Client-Server-Kommunikation, sondern ein dezentraler Ansatz mit einem Publish/Subscribe-System auf einem \ac{p2p}-Netzwerk, gewählt \cite{Fischer2010a}. Hierbei soll durch die unterschiedliche Behandlung einzelner Eventtypen die Verteilung dieser optimiert werden \cite{Fischer2010Event}.

\begin{figure}[htbp]
\centering
\resizebox{\textwidth}{!}{%
\includegraphics{grafics/metis_aufbau.pdf}}
\caption{Architekturübersicht von M$^2$etis}
\label{fig:metis_aufbau}
\end{figure}

\Fref{fig:metis_aufbau} zeigt die zeitliche Aufteilung des Systems in \emph{Konzeption} und \emph{Laufzeit} sowie die Schnittpunkte von \ac{m2etis} mit dem \ac{mmve}. In der Grafik wird dieses durch ein \emph{Spiel} dargestellt.

Im ersten Schritt werden die im Spiel genutzten Eventtypen durch den Spielentwickler identifiziert und semantisch beschrieben. Im zweiten Schritt verarbeitet der \ac{m2etis} \emph{Optimierer} diese Beschreibungen unter Zuhilfenahme des \emph{semantischen Modells} und des \emph{Kostenmodells}. Für jeden Typ wird dabei ein optimierter \emph{Kanal} erzeugt, in dem die unterschiedlichen Implementierungen der in \cite{Fischer2010a} beschriebenen Dimensionen Anwendung finden.\\
Die optimierten Kanäle lassen sich zur Laufzeit (im 3. Schritt) über die Publish/Subscribe-API ansprechen und kommunizieren über das \ac{p2p}-Netzwerk. Zur Laufzeit soll das Spiel jedoch keine weiteren Implementierungsdetails der optimierten Kanäle kennen müssen, um das System nutzen zu können.

Diese Arbeit beschränkt sich auf die Anbindung des gewählten \ac{p2p}-Netzwerk und die Konzeption des Publish/Subscribe-Systems. Damit das Framework frühzeitig einsetzbar ist und weitere Arbeiten darauf aufbauen können, wird zudem eine prototypische Implementierung des Publish/Subscribe-Systems erstellt. Die Konzeption des Kostenmodells oder die Entwicklung des Optimierers würden den Rahmen dieser Diplomarbeit sprengen und bleiben nachfolgenden Arbeiten überlassen.


\section{Umsetzung der Dimensionen}
Vor der Beschreibung des Verarbeitungsmodells müssen die semantischen Dimensionen in dort greifbare Elemente umgesetzt werden. Im Hinblick auf die Implementierung mit \emph{policy-based Design} wird, für die logische Umsetzung einer Dimension der Begriff \emph{Policy} eingeführt. Eine Policy definiert die Schnittstelle für verschiedene konkrete Implementierungen (genannt Strategie) und deren Auswirkung auf die Nachrichtenverarbeitung im Publish/Subscribe-System. Die folgenden sieben Policies entsprechen nahezu 1:1 den in \Fref[plain]{chap:grundlagen:event} identifizierten Dimensionen:

\begin{description}
\item[Verteilung] bestimmt die Verteilungsart der einzelnen Events und den Aufbau des logischen Multicast-Trees, mittels dessen die Nachrichten versandt werden \cite{KostasKatrinis2005}.
\item[Filterung] erlaubt es Anmeldungen, Prädikate mitzugeben. Implementierungen dieser Policy müssen sicherstellen, dass diese Prädikate nach oben im Multicast-Tree zusammengeführt und Nachrichten frühzeitig gefiltert werden können. Dies bedeutet, dass Nachrichten jeweils beim Versand durch den logischen Kopf des Multicast-Trees gefiltert werden.
\item[Zustellung] bestimmt das Kommunikationsparadigma des Nachrichtenversands und leitet beispielsweise den Versand von Bestätigungen über eingegangene Nachrichten an den sendenden Knoten ein.
\item[Reihenfolge] definiert das Synchronisationskonzept eines Kanals.
\item[Persistenz] bietet die Möglichkeit der Speicherung eines Events beziehungsweise der daraus erfolgenden Zustandsänderung der virtuellen Welt.
\item[Sicherheit] gibt eine Schnittstelle zur Nachrichtenverschlüsselung vor.
\item[Validität] prüft die ankommenden Nachrichten auf ihre Validität. Frühzeitig verworfene Nachrichten vermindern das Nachrichtenaufkommen im System stark.
\end{description}

Im späteren Optimierungsschritt wird \ac{m2etis} für jeden semantischen Typen einen optimierten Kanal erzeugen, der über das Publish/Subscribe-System mit den bekannten Methoden \texttt{subscribe, unsubscribe} und \texttt{publish} ansprechbar ist. Jeder Kanal ist entlang den sieben Dimensionen derart optimiert, dass Nachrichten bestmöglich verarbeitet und über das Netzwerk verteilt werden können. Das Netzwerk selbst wird über die bereits erwähnte \ac{kbr}-API\footnote{siehe \Fref[plain]{chap:grundlagen:api}} angesprochen. Die Verbindung der Publish/Subscribe-Methoden mit dieser \ac{kbr}-API ist trivial: Jede verschickte Nachricht muss auf den Knoten durch die Methoden \texttt{forward} und \texttt{deliver} des Netzwerkes verarbeitet werden. Einzelne Strategien können jedoch auf weitere Methoden der KBR-API zugreifen und zum Beispiel die Nachbarschaft des aktuellen Knotens abfragen.\\
Die Anwendung der Policies auf die verschiedenen Nachrichtentypen hingegen ist interessanter und in \Fref{tab:verbindungsmatrix} dargestellt. Hierbei ist zu beachten, dass \emph{Publish}-Nachrichten den eigentlichen Events entsprechen, deren Verteilung optimiert wird. \emph{Subscribe}- und \emph{Unsubscribe}-Nachrichten dienen nur zum Aufbau und zur Verwaltung des logischen Verteilungssystems. Hierzu sind nur die Policies Verteilung, Filterung und Sicherheit nötig, während bei Publish-Nachrichten alle Policies involviert sind, um die gewünschte Optimierung zu ermöglichen.

\begin{table}[!h]
\resizebox{\textwidth}{!}{%
\begin{tabular}{llccccccc}
\toprule
\multirow{2}{2cm}{Nachrichten\-typ} & \multirow{2}{2cm}{KBR-API Methode}	& \multicolumn{7}{c}{Policy pro Kanal} \\
\cmidrule{3-9}
			&	& Verteilung & Filterung & Zustellung & Reihenfolge & Persistenz & Sicherheit & Validität \\
\toprule 
publish	    & \texttt{deliver} & + & + & + & + & + & + & + \\
\midrule
\multirow{2}{*}{subscribe}	& \texttt{deliver} & + & + &   &   &   & + & \\
\cmidrule{2-9}
			& \texttt{forward} & + & + &   &   &   & + & \\
\midrule
\multirow{2}{*}{unsubscribe} & \texttt{deliver} & + & + &   &   &   & + & \\
\cmidrule{2-9}
			& \texttt{forward} & + & + &   &   &   & + & \\
\bottomrule
\end{tabular}}
\caption{Verbindungsmatrix}
\label{tab:verbindungsmatrix}
\end{table}

Die eingesetzte Verteilungsstrategie entscheidet darüber, ob \emph{Subscribe}- und \emph{Unsubscribe}-Nachrichten in \texttt{forward} und/oder \texttt{deliver} behandelt werden. Nur in \texttt{forward} ist die Nachricht durch Verteilung und Filter veränderbar und kann auch durch den Verteilungsalgorithmus terminiert werden. Durch die Behandlung aller Publish-Nachrichten -- also Events --, ausschließlich in \texttt{deliver}, wird sichergestellt, dass diese bei allen Empfängern ankommen und unterwegs nicht verändert oder unterbrochen werden (wie es \texttt{forward} erlauben würde).

Eine Ausweitung der anderen Policies auf die anderen Nachrichtentypen ist nicht nötig. Anmeldungen und Abmeldungen benötigen keine spezielle Reihenfolge und müssen zudem auch nicht gespeichert werden. Eine spezielle Zustellungsgarantie für diese Nachrichten ist  kontraproduktiv:  Anmeldungen müssen -- systembedingt\footnote{Viele Multicast-Systeme, zum Beispiel Scribe, verlangen eine periodische Auffrischung der Anmeldung um eventuell ausgefallene Knoten im Multicast-Tree ersetzen zu können.} -- periodisch wiederholt werden. Eine Zustellbestätigung würde die Anzahl der Verwaltungsnachrichten stark erhöhen.

Nachdem die Verbindung des Netzwerkes mit der Publish/Subscribe-API dargelegt ist, wird die Reihenfolge der Policies -- bestimmend für eine effiziente Bearbeitung jeder einzelnen Nachricht -- im nächsten Kapitel über das Verarbeitungsmodell beschrieben.

\section{Verarbeitungsmodell}
Das Verarbeitungsmodell für den \emph{Versand} und der \emph{Verarbeitung} einer Nachricht in \texttt{forward} und \texttt{deliver}, muss die Verarbeitung der verschiedenen Policies koordinieren und auf eine effiziente Bearbeitung einer Nachricht ausgelegt sein. Beispielsweise ist eine frühe Validitätsprüfung einer Nachricht sinnvoll. Jedoch ist \enquote{Verteilung} ist neben \enquote{Filterung} die wichtigste Policy, denn sie bestimmt das Routing der Nachrichten. Das von der Verteilungspolicy erzeugte System wird als ein logischer Multicast-Tree betrachtet. Die restlichen Policies ermöglichen zusätzliche Verarbeitungsmöglichkeiten einer Nachricht an einem Knoten, haben aber keinen Einfluss auf die Verteilung der Nachrichten und fügen sich ohne Probleme in das entwickelte Modell ein. 

Zur Modellentwicklung wurden verschiedene Verteilungsalgorithmen verglichen und deren Gemeinsamkeiten und Unterschiede untersucht, um ein allgemein gültiges Verteilungsmodell zu erstellen. Am Beispiel von Scribe\footnote{siehe \Fref{chap:related:scribe}} wurde ein generischer Multicast-Tree und ein Multicast-Tree der Höhe 1 -- im Folgenden als \emph{DirectSend} bezeichnet -- untersucht. VON\footnote{siehe \Fref{chap:related:von}} stand Pate für einen Verteilungsalgorithmus, der nachbarschaftszentriert arbeitet. Gemein haben diese Algorithmen, dass die Empfänger einer Nachricht sowohl vom Typ einer Nachricht als auch von der logischen Position ihres Knotens im Multicast-Tree bestimmt werden. Unterschiede gibt es zum Beispiel bei der unterschiedlichen Verarbeitung von \emph{Subscribe}-Nachrichten\footnote{\emph{Unsubscribe}-Nachrichten werden ähnlich behandelt}. Im Falle des Multicast-Trees müssen diese in \texttt{forward} verändert oder terminiert werden, während es bei DirectSend ausreichend ist, diese in \texttt{deliver} zu bearbeiten. Auch bei der Verarbeitung von \emph{Publish}-Nachrichten finden sich Unterschiede: Multicast-Tree-Algorithmen senden diese zuerst an die Rootknoten, das sind die Knoten die als logische Wurzel des Baumes darstellen und für die eigentliche Verteilung des Events zuständig sind. Auf den Rootknoten wird die eigentliche Verteilung der Nachricht eingeleitet. Bei VON hingegen können \emph{Publish}-Nachrichten direkt an alle Nachbarn (aus Spielsicht) gesendet werden. Dies zeigt, dass die Empfänger einer Nachricht nicht generisch ermittelt werden können, sondern von der gewählten Verteilungsstrategie abhängig sind.\\
Nach der nun folgenden Beschreibung der einzelnen Verarbeitungsabschnitte, werden die oben genannten Algorithmen beispielhaft auf das Verarbeitungsmodell angewandt und dessen Flexibiliät bezüglich verschiedenster Verteilungsstrategien gezeigt. Zuerst wird die Behandlung beim Erstellen und Versenden von Nachrichten erklärt.

\begin{figure}[htbp]
\centering
\resizebox{\textwidth}{!}{%
\includegraphics{grafics/processing_send.pdf}}
\caption{Versand von Nachrichten}
\label{fig:processing_send}
\end{figure}

Beim Erstellen einer Nachricht werden Verwaltungsinformationen der einzelnen Policies abgefragt und zusammen mit der Nachricht verschickt. Anhand dieser Informationen kann die Nachricht auf einem anderen Knoten entsprechend behandelt werden. \emph{Subscribe}- und \emph{Unsubscribe}-Nachrichten bestehen nur aus Verwaltungsinformationen, da sie zum Aufbau und der Verwaltung des Multicast-Trees dienen und keine Events transportieren.

Der Versand einer Nachricht ist in \Fref{fig:processing_send} dargestellt. Nachdem die Nachricht erstellt ist, wird die Verteilungspolicy nach einer Liste von Empfängern befragt. Der konkreten Verteilungsstrategie stehen verschiedene Informationen zur Verfügung um die korrekten Empfänger zu ermitteln: Einerseites der Nachrichtentyp (\emph{Subscribe}, \emph{Unsubscribe} oder \emph{Publish}) und für \emph{Publish}-Nachrichten die Information, ob diese von einem normalen Knoten an die dedizierten Rootknoten gesendet oder von solch einem Rootknoten weiter verteilt werden. Im erstern Fall wird die Nachricht mit \emph{to root}, im zweiten Falle mit \emph{from root} gekennzeichnet. Diese Kennzeichnung wird bei der Erstellung einer Nachricht angegeben und ebenfalls von der gewählten Verteilungsstrategie abgefragt. Für eine \emph{Publish}-Nachrichten mit der Markierung \emph{from root} entfernt die Filterpolicy alle Knoten aus der Empfängerliste, deren Prädikat nicht auf die Nachricht passt. Dadruch wird eine früher Filterung weit oben im logischen Multicast-Tree ermöglicht.\\
Schließlich wird die Nachricht durch die anhand der Sicherheitspolicy vorgegebene Verschlüsselung kodiert und über das Netzwerk an alle Empfänger gesandt.

Zur Behandlung von Nachrichten in \texttt{forward} werden diese erst dekodiert, wie es in \Fref{fig:processing_forward} aufgezeigt ist. Die Verteilungs- und Filterpolicies können anhand der mitgeschickten Verwaltungsinformationen ihren Zustand anpassen und wenn nötig die Nachricht ändern oder gar stoppen.

\begin{figure}[htbp]
\centering
\resizebox{\textwidth}{!}{%
\includegraphics{grafics/processing_forward.pdf}}
\caption{Verarbeitung von Nachrichten in forward}
\label{fig:processing_forward}
\end{figure}

Die Abarbeitung der Nachrichten in \texttt{deliver} ist komplexer als die beiden oben genannten Fälle, da hier einerseits alle Policies zusammenarbeiten und die Verteilung von \emph{Publish}-Nachrichten getriggert wird. \Fref[plain]{fig:processing_deliver} stellt die einzelnen Vearbeitungsschritte dar. Nach der Entschlüsselung werden \emph{Subscribe}- und \emph{Unsubscribe}-Nachrichten ähnlich wie bei der Behandlung in \texttt{forward} verarbeitet. Die Policies können ihren Zustand aktualisieren, jedoch die Nachricht nicht verändern.\\
\emph{Publish}-Nachrichten müssen eine Validitätsprüfung bestehen, bevor entschieden wird, ob sie eine Nachricht \emph{to root}, also an die Rootknoten sind oder nicht. Jeder Rootknoten und alle anderen Knoten auf dem Verteilungsweg, die selbst Verteilungsaufgaben übernehmen müssen (abhängig von der gewählten Verteilungspolicy), leiten nun das Verteilen der Nachricht ein: Dazu werden in einem ersten Schritt eine neue \emph{Publish}-Nachricht mit der Markierung \emph{from root} erstellt und die Nachricht mit dem obig beschriebenen Verfahren gesendet. Nun wird an diesen Knoten geprüft, ob sie selbst an diesem Kanal angemeldet sind und, falls dies zutrifft, ob sie an der Nachricht interessiert sind. Wenn nicht, endet die Bearbeitung der Nachricht. Ansonsten trifft sich der Ablaufpfad an dieser Stelle mit dem Ablaufpfad einfacher Knoten, die lediglich Subscriber sind.

\begin{figure}[htbp]
\centering
\resizebox{\textwidth}{!}{%
\includegraphics{grafics/processing_deliver.pdf}}
\caption{Verarbeitung von Nachrichten in deliver}
\label{fig:processing_deliver}
\end{figure}


Die Synchronisierungspolicy ermöglicht eine Wohlgeordnetheit aller Nachrichten des Eventtyps. Sie kann die zu bearbeitende Nachricht sowohl komplett zurückhalten als auch mehrere Nachrichten zurückgeben, falls durch die aktuelle Nachricht weitere Nachrichten \enquote{freigeschaltet} werden. Alle Nachrichten werden nun nochmals auf ihre Validität geprüft, da zurückgehaltene Nachrichten inzwischen veraltet sein können. Für jede valide Nachricht wird eine Signalisierung der Zustellung entsprechend der gewählten Strategie, zum Beispiel eine Bestätigungsnachricht zurück an den Sender, ermöglicht. Bevor die Nachrichten schließlich an die Applikation übergeben werden, können sie -- oder ihre Auswirkung auf den Zustand der virtuellen Welt -- abgespeichert werden.


Der nächste Abschnitt prüft das eingeführte Verarbeitungsmodell anhand einiger Strategien auf seine Tauglichkeit. Dazu werden hauptsächlich Verteilungsstrategien genutzt, da diese die Hauptarbeit im System tragen.

\subsection*{Beispielhafte Strategien}
Für jede einzelne Policy bietet \ac{m2etis} mehrere konkrete Implementierungen an und lässt sich mit benutzerdefinierten Strategien weiter auf die Anforderungen der Applikation anpassen\footnote{siehe \Fref{fig:metis_aufbau}}. In diesem Kapitel wird das Verarbeitungsmodell anhand zweier Verteilungsstrategien getestet. Weitere konkrete Strategien werden im Folgenden nur kurz angerissen, da sie keinen Einfluss auf die Verteilung eines Events haben, sondern lediglich der Nachrichtenverarbeitung auf einem Knoten dienen. Die Verteilung der einzelnen Nachrichten hat einen höheren Stellenwert, da sie eine Auswirkung auf die Anzahl der zu verschickenden Nachrichten im System haben.\\
Beispielsweise fügt eine zeitstempelbasierte Strategie zur Validitäsprüfung bei der Erstellung einer \emph{Publish}-Nachricht dieser den aktuellen Zeitstempel hinzu. Anhand diesem Zeitstempel kann bei der Prüfung einer Nachricht entschieden werden, ob ein entsprechendes Zeitintervall seit dem Versand abgelaufen ist und die Nachricht damit üngültig ist. Ein einfacher Ansatz zur Nachrichtensynchronisation ist ebenfalls mit Zeitstempeln möglich. In \Fref[plain]{chap:impl} wird die Schnittstelle für diese Strategien aufgezeigt.

Die näher betrachtete Verteilungsalgorithmen orientieren sich an Scribe, das in \Fref{chap:related:scribe} vorgestellt wurde, und an VON, das auch bereits in \Fref[plain]{chap:related:von} beschrieben ist. Scribe erzeugt einen Multicast-Tree, während VON seinen Fokus auf die \enquote{in-game}-Nachbarschaft legt. Dies ist vor allem bei Events mit einer hohen Lokalität sinnvoll. Der Einfluss solcher Events, wie zum Beispiel Positionsänderungen, wirkt sich nur in einem begrenzten Gebiet aus und muss daher nur benachbarten Knoten -- im Spiel -- mitgeteilt werden.

Zuerst wird das Verarbeitungsmodell zum Nachrichtenversand und danach der Empfang und die Verteilung von Nachrichten betrachtet. Bei Scribe richten sich im Gegensatz zu VON die Zielknoten nach der Nachrichtenart und der Position des jeweiligen Knotens im logischen Verteilungsbaum. \emph{Subscribe}- und \emph{Unsubscribe}-Nachrichten wie auch \emph{Publish}-Nachrichten normaler Knoten werden immer an den Rootknoten des Kanals gesendet. Hierzu kann der Hashwert des Kanalnamens berechnet und auf einen Schlüssel im Netz abgebildet werden. Verteilen der Rootknoten oder weitere Knoten auf dem Verteilungsweg eine \emph{Publish}-Nachricht, so gibt Scribe die Liste der an diesem Knoten eingeschriebenen Subscriber zurück.\\
Ein Algorithmus wie VON könnte auf Applikationswissen zugreifen und liefert die sichtbaren Nachbarn im Spiel als Empfänger einer \emph{Subscribe}-Nachricht. Diese Nachbarschaftsmetrik muss sich nicht alleine auf die Position im Spiel beziehen, sondern kann auch von der Mitgliedschaft in Gruppen bestimmt werden; sie ist aber in jedem Falle spezifisch für den gewählten Eventtyp. \emph{Unsubscribe}-Nachrichten richten sich prinzipiell an die gleichen Nachbarn jedoch müssen diese mit den Nachbarn zur Anmeldung abgeglichen werden. \emph{Publish}-Nachrichten werden zuerst immer an den Knoten selbst geschickt, damit dieser in der Abarbeitung in \texttt{deliver} den Event an alle eingetragenen Empfänger senden kann.

Scribe verarbeitet \emph{Subscribe}- und \emph{Unsubscribe}-Nachrichten in \texttt{forward}. Jeder Knoten auf dem Routingpfad vom Sender zum Rootknoten fügt den Sender der Nachricht seiner Liste der Empfänger hinzu und ändert die Anmeldenachricht: Er trägt sich als Absender ein. Der Multicast-Tree wird also Knoten für Knoten beim Routing der Nachrichten aufgebaut. Sollte der Knoten schon angemeldet sein, kann er die Nachricht terminieren und damit effektiv die Anzahl an Nachrichten begrenzen\footnote{siehe Beschreibung von Scribe in \Fref[plain]{chap:related:scribe}}.\\
Die von Scribe geforderte periodische Auffrischung der Anmeldung erfolgt für normal angemeldete Knoten außerhalb der Verteilungsstrategie: Die Anmeldung wird nach einem, von der Strategie bestimmten Zeitintervall, wiederholt. Ist ein Knoten aufgrund seiner Lage im Multicast-Tree angemeldet, kann er diese Anmeldung auffrischen, wenn periodische Anmeldungen zu bearbeiten sind. Hierbei ist es wichtig, dass die Filterstrategie bei allen Auffrischungen involviert ist, damit die Prädikate weiterhin im Baum zusammengeführt werden. Durch die periodischen Anmeldungen werden auch ausgefallene Knoten ausgetauscht, denn die Nachrichten werden vom Netzwerk über andere Knoten geroutet und somit wird der Multicast-Tree wieder aufgebaut.

Eine an VON angelegte Strategie bearbeitet alle Nachrichten in \texttt{deliver}. Der Sender der \emph{Subscribe}-Nachricht wird in die Liste der Empfänger eingetragen. Eine \emph{Publish}-Nachricht wird an all diese weitergeleitet. Da ein Knoten nicht bei sich selbst angemeldet ist, kann die Bearbeitung der Nachricht beendet werden.

Die an VON angelehnte Verteilungsstratege kann auch gänzlich anders implementiert werden. Alle Anmeldungen sind lediglich implizit und werden nicht als Nachrichten auf das Netzwerk gelegt. Statt eine \emph{Publish}-Nachricht an den eigenen Knoten zu senden, wird diese sofort allen Nachbarn im Spiel zugestellt. Dies bedeutet, dass \emph{Publish}-Nachrichten vom Sender mit \emph{from root} statt \emph{to root} markiert werden. Mit diesem impliziten System entfällt auch die Abmeldung am Kanal.

Diese drei vorgestellten Implementierungsansätze verschiedener Verteilungsstrategien zeigen wie mächtig das Verarbeitungsmodell ist und welchen Spielraum es verschiedenen Strategien bietet, um Events anhand der angebotenen Dimensionen bestmöglich zu bearbeiten.
