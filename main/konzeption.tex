\chapter{Konzeption des Frameworks}
\label{chap:konzeption_pubsub}
In diesem Kapitel wird die Konzeption des Frameworks zur Verteilungsoptimierung in seinen Einzelheiten erläutert. Ausgehend von den in \cite{Fischer2010a} identifizierten Typen von Events und deren orthogonalen Dimensionen zur Optimierung wird die, im nächsten Abschnitt beschriebene, Problemstellung greifbar. Das Framework muss die vielfältige Anpassung einzelner Kanäle ermöglichen ohne dies mit Einbußen zur Laufzeit zu erkaufen \cite{NEW_PAPER} \missing{CITE: NEW PAPER!}. Ein weiterer wichtiger Fokus dieser Arbeit ist \ac{m2etis} als ein einfach nutzbares kanalbasiertes Publish/Subscribe-System zu präsentieren:\\
Es muss ohne Wissen über die verschiedenen Optimierungen verwendbar sein.

Im Optimierungsschritt erzeugt \ac{m2etis} für jeden semantisch Typen einen optimierten Kanal der über das Publish/Subscribe-System ansprechbar ist\footnote{siehe \Fref{chap:grundlagen:aufbau_metis}}. Dieses bieten die bekannten Methoden \emph{subscribe, unsubscribe} und \emph{publish} an. Jeder Kanal ist entlang den Dimensionen derart optimiert, dass Nachrichten bestmöglich verarbeitet und über das Netzwerk verteilt werden können.

Das Netzwerk selbst wird über die bereits erwähnte \ac{kbr}-API angesprochen. Drei Methoden, sogenannte ``upcalls"\footnote{Technisch gesehen eine Callbackmethode der Applikation, welche vom Netzwerk aufgerufen wird.},  informieren über Ein- und Austritte von Knoten sowie über zu routende und auch ankommende Nachrichten an einem Knoten. Andere Methoden ermöglichen die Abfrage von Netzwerkinformationen \cite{Dabek2003Towards}.

\section{Umsetzung der Dimensionen}
Für die logische Umsetzung der in \cite{Fischer2010Event} identifizierten semantischen Dimensionen eines Eventtyps wird der Begriff \emph{Policy} eingeführt.\\
Eine Policy definiert die Schnittstelle für verschiedene konkrete Implementierungen (genannt Strategie) und deren Auswirkung auf die Nachrichtenverarbeitung im Publish/Subscribe-System. Die folgenden sieben Policies decken die Dimensionen ab \cite{NEW_PAPER}.\missing{CITE: NEW PAPER!}

\begin{description}
\item[Verteilung] bestimmt die Verteilungsart der einzelnen Events und den Aufbau des logischen Multicast-Trees, mittels dem die Nachrichten versandt werden \cite{KostasKatrinis2005}.
\item[Filterung] erlaubt es Anmeldungen Prädikate mitzugeben. Diese Policy stellt sicher, dass diese Prädikate nach oben im Multicast-Tree zusammengeführt werden und Nachrichten frühzeitig gefiltert werden können. Dies bedeutet, dass Nachrichten jeweils beim Versand durch den logischen Kopf des Multicast-Trees gefiltert werden.
\item[Zustellung] bestimmt das Kommunikationspradigma des Nachrichtenversands und leitet z.B. den Versand von Bestätigungen über eingegangene Nachrichten an den sendenden Knoten ein.
\item[Reihenfolge] definiert das Synchronisationskonzept eines Kanals.
\item[Persistenz] bietet die Möglichkeit der Speicherung eines Event beim Empfänger.
\item[Sicherheit] gibt eine Schnittstelle zur Nachrichtenverschlüsselung vor.
\item[Validität] prüft die ankommenden Nachrichten auf ihre Validität. Frühzeitig verworfene Nachrichten vermindern das Nachrichtenaufkommen im System stark.
\end{description}

Die Verbindung der Publish/Subscribe Methoden und der genutzten \ac{kbr} Methoden ist trivial: Jede verschickte Nachricht wird durch die Methoden \emph{forward} und \emph{deliver} des Netzwerkes verarbeitet wird \cite{Dabek2003Towards}.\\
Die Verteilung der Policies auf die verschiedenen Nachrichtentypen ist hingegen interessanter und in \Fref{tab:verbindungsmatrix} aufgezeigt. Bei der Verarbeitung von \emph{Publish}-Nachrichten, den eigentlichen Events, ist jede Policy beteiligt. Die \emph{Subscribe}- und \emph{Unsubscribe}-Nachrichten dienen nur zum Aufbau und der Verwaltung des Multicast-Trees. Aus diesem Grund sind bei diesen Nachrichtentypen nur die Policies Verteilung, Filterung und Sicherheit involviert.

\begin{table}[!h]
\resizebox{\textwidth}{!}{%
\begin{tabular}{llccccccc}
\toprule
Nachrichten- & KBR	& \multicolumn{7}{c}{Policy pro Kanal} \\
\cmidrule{3-9}
typ			&	Methode	& Verteilung & Filterung & Zustellung & Reihenfolge & Persistenz & Sicherheit & Validität \\
\midrule
publish	    & deliver & + & + & + & + & + & + & + \\
\midrule
subscribe	& deliver & + & + &   &   &   & + & \\
\cmidrule{2-9}
			& forward & + & + &   &   &   & + & \\
\midrule
unsubscribe & deliver & + & + &   &   &   & + & \\
\cmidrule{2-9}
			& forward & + & + &   &   &   & + & \\
\bottomrule
\end{tabular}}
\caption{Verbindungsmatrix}
\label{tab:verbindungsmatrix}
\end{table}

Events, also Publish-Nachrichten, werden ausschließlich in deliver behandelt, so dass diese bei allen Empfängern ankommen und unterwegs nicht verändert oder terminiert werden können. Dies ist ein kleiner Schritt in die Richtung zur Betrugsverhinderung. Für Subscribe- und Unsubscribe-Nachrichten entscheidet die eingesetzte Verteilungsstrategie darüber, ob diese in forward und/oder deliver behandelt werden. Nur in foward ist die Nachricht durch Verteilung und Filter veränderbar. Weiterhin kann die Verteilung der Nachricht auch terminiert werden, wenn dies der Verteilungsalorithmus wünscht.

Die Reihenfolge der Policies wird im nächsten Abschnitt beschrieben. Sie ist bestimmend für eine effiziente Bearbeitung jeder einzelnen Nachricht.

\section{Verarbeitungsmodell}
Beim Erstellen einer Nachricht, werden Verwaltungsinformationen der einzelnen Policies abgefragt und mit der Nachricht verschickt. Anhand dieser Informationen kann die Nachricht auf einem anderen Knoten entsprechend behandelt werden. Subscribe- und Unsubscribenachrichten bestehen nur aus diesen Verwaltungsinformationen, da sie zum Aufbau und der Verwaltung des Multicast-Trees dienen und keine Events transportieren.

Das von der Verteilungspolicy erzeugte System ist ein logischer Multicast-Tree. Dies bedeutet, dass es einen oder mehere Knoten, im folgenden Rootknoten genannt, gibt, die die Wurzel des Multicast-Trees darstellen. In \Fref{chap:related:scribe} wurde ein möglicher Aufbau anhand von Scribe beschrieben. Nachbarschaftszentrierte Algorithmen, die z.B. an VON \cite{Hu2006VON} angelehnt sind, können mehere Knoten als Wurzel eines Multicast-Trees haben. 

Der Versand einer Nachricht ist in \Fref{fig:processing_send} dargestellt. Nachdem die Nachricht erstellt ist, wird die Verteilungspolicy nach einer Liste von Empfängern befragt. Diese ist vom Nachrichtentyp abhängig.\\
Für Subscribe- und Unsubscribenachrichten sind dies immer die Rootknoten des Multicast-Trees, da diese dessen Aufbau koordinieren, beziehungsweise dies durch die Knoten auf dem Routingpfad durch das Netzwerk geschieht.\\
Publish-Nachrichten werden an die Rootknoten geschickt, falls die Nachricht mit \emph{to root} gekennzeichnet ist. Dies ist der Fall, wenn ein Knoten einen Event in das System bringen möchte. Jeder Event muss an die Rootknoten geschickt werden, da die Verteilung selbst durch die Methode deliver ausgeführt wird. Hat die Nachricht die Rootknoten erreicht, wird sie mit \emph{from root} gekennzeichnet und die Verteilungsstrategie gibt als Empfänger die eingeschriebenen Knoten aus. Für jeden Eintrag dieser Liste wird nun von der Filterstrategie geprüft, ob die Nachricht auf das mit diesem Knoten verbundene Prädikat passt und an diesen verschickt werden soll. Schließlich wird die Nachricht durch die anhand der Sicherheitsstrategie vorgegebene Verschlüsselung kodiert und über das Netzwerk an alle Empfänger gesandt.

\begin{figure}[htbp]
\centering
\resizebox{\textwidth}{!}{%
\includegraphics{grafics/processing_send.pdf}}
\caption{Versand von Nachrichten}
\label{fig:processing_send}
\end{figure}

Sollten Nachrichten in forward behandelt werden, so werden diese zuerst dekodiert wie es in \Fref{fig:processing_forward} aufgezeigt ist. Die Verteilungs- und Filterpolicies können anhand der Verwaltungsinformationen ihren Zustand anpassen und wenn nötig die Nachricht ändern oder gar stoppen.

\begin{figure}[htbp]
\centering
\resizebox{\textwidth}{!}{%
\includegraphics{grafics/processing_forward.pdf}}
\caption{Verarbeitung von Nachrichten in forward}
\label{fig:processing_forward}
\end{figure}

Die Abarbeitung der Nachrichten in deliver ist komplexer als die beiden oben genannten Fälle, da hier alle Policies zusammenarbeiten wie es \Fref{fig:processing_deliver} aufzeigt. Nach der Entschlüsselung werden Subscribe- und Unsubscribenachrichten ähnlich wie bei der Behandlung in forward verarbeitet. Die Policies können ihren Zustand aktualisieren.\\
Publishnachrichten müssen eine Validitätsprüfung bestehen, bevor entschieden wird, ob sie eine Nachricht \emph{to root}, also an die Rootknoten sind, oder nicht. Jeder Rootknoten und alle anderen Knoten auf dem Verteilungsweg, die selbst Verteilungsaufgaben übernehmen müssen (abhängig von der gewählten Verteilungsstrategie), leiten nun das Verteilen der Nachricht ein. Dazu wird in einem ersten Schritt eine neue Publishnachricht erstellt, alle Verwaltungsinformationen abgefragt und schließlich die Nachricht mit dem obig beschriebenen Verfahren gesendet. Nun wird an diesen Knoten geprüft ob sie selbst an diesem Kanal angemeldet sind und ob sie in diesem Falle an der Nachricht interessiert sind. Wenn nicht, endet die Bearbeitung der Nachricht. Ansonsten trifft sich der Ablaufpfad an dieser Stelle mit dem Ablaufpfad einfacher Knoten, die lediglich Subscriber sind. Die Synchronisierungspolicy ermöglicht eine Wohlordnung der Nachricht und kann diese sowohl komplett zurückhalten als auch mehrere Nachrichten zurückgeben, falls durch die aktuelle Nachricht weitere Nachrichten ``freigeschaltet" werden. Diese Nachrichten werden nun nochmals auf ihre Validität geprüft, da zurückgehaltene Nachrichten inzwischen veraltet sein können. Für jede valide Nachricht wird eine Signalisierung der Zustellung laut Policy, z.B. eine Acknowledgementnachricht zurück an den Sender, ermöglicht. Bevor die Nachricht schließlich an die Applikation übergeben wird, kann sie persistiert werden.

\begin{figure}[htbp]
\centering
\resizebox{\textwidth}{!}{%
\includegraphics{grafics/processing_deliver.pdf}}
\caption{Verarbeitung von Nachrichten in deliver}
\label{fig:processing_deliver}
\end{figure}

Im nächsten Kapitel wird das Verarbeitungsmodell anhand einiger Strategieen getestet.

\subsection{Beispielhafte Strategien}
For every policy \ac{m2etis} provides one ore more implementations. For example \emph{Direct} or \emph{Multicast} for routing. The framwork permits inclusion of more user-defined strategies. Different types of channel use a different set of implementations for every policy, creating a variety of optimization options. Each implemented strategy must provide additional cost information that is used in the optimizing step and they may add information onto each message to enable the customized treatment.

Using the example of Scribe \cite{Castro2002Scribe} and VON \cite{Hu2006VON} some details and inner workings of the processing model showed in figure \ref{fig:deliver} are explained. Scribe creates a multicast-tree trying to minimize the amount of messages while an algorithm like VON is neighborhood centric.

Using a scribe-like algorithm for routing \emph{Get Targetnodes} returns either the calculated root node for the channel or the list of subscribed nodes to which a publication must be forwarded. On the other hand a routing strategy like VON returns the in-game neighbors, obtained through application-level knowledge, as each node subscribes at his neighbors in the virtual world. Publications are sent to self enabling the distribution in deliver.

Scribe processes unsubscribe and subscribe messages in forward. Each node on the routing path adds the sender to its list of subscribers and can changes the message to create the multicast-tree. The periodic resubscription is triggerd externally for subscribed nodes as the channel will call its own subscribe method again or internally for intermediate routing hops as the algorithm will resend its subscriptions if other nodes down in the multicast-tree will refresh their subscribtion. It is necessary that the filter strategy is always involved to ensure the correct merge of predicates upwards the logical mulitcast-tree.

VON on the other hand does not need automatic periodic subscriptions, because each node will unsubscribe and subscribe frequently. Using the example of position updates it is obvious that each node and its neighbors move often and need to alter their subscribtions each frame in the game.

Using TTL or a timestamp-based approach for the strategy implementing the validity policy, the TTL would be increased on every routing hop or the passed time since the message was created is tested against the interval set in the strategy.

This exemplary discussion of different strategies indicates the generic nature of our processing model in terms of a multidimensional optimization for publish/subscribe channels.
