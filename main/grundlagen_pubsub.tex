\section{Verteilte Publish/Subscribe-Systeme}
\label{chap:grundlagen:pubsub}
Konzeptionell lassen sich Publish/Subscribe-Systeme als eventbasierte Systeme betrachten. Auf Grund ihres Aufbaus und der Skalierung  in orthogonalen Dimensionen\index{Publish/Subscribe!orthogonale Dimensionen} \enquote{Raum}, \enquote{Zeit} sowie \enquote{Synchronisation} eignen sich diese gut zur Verteilung von Events in dezentralen Umgebungen \cite{PatrickTh2003Many, Cugola2002Using}.

\begin{figure}[htbp]
\centering
\includegraphics{grafics/pubsub_black_box.pdf}
\caption{Schema eines Publish/Subscribe-Systemes.}
\label{fig:pubsub_black_box}
\end{figure}

Publisher und Subscriber werden durch das Event-System voneinander getrennt, wie es in \Fref{fig:pubsub_black_box} dargestellt ist.  Publisher und Subscriber sind räumlich voneinander getrennt. Ein Publisher übergibt die Nachricht an das System und hält weder direkte Verbindung mit den Subscribern noch muss der Publisher alle Subscriber kennen. Diese Trennung bezieht sich nicht nur auf verschiedene Komponenten einer Applikation, sondern kann auch über Applikations- oder gar Rechnergrenzen gehen. Die Zeitliche Trennung beschreibt, dass sich ein Subscriber am System anmelden kann obwohl kein Publisher vorhanden ist, analog können Nachrichten publiziert werden ohne dass Empfänger eingeschrieben sind. Je nach Implementierung können Nachrichten zwischengespeichert werden um diese neuen Subscribern zu zustellen. Bei einem Fernaufrufsystem wie \emph{remote proceduce call (RPC)} \cite{Birrell1984Implementing} ist dies nicht möglich, da die Gegenseite existieren sein muss. Das Senden einer Nachricht ist für den Publisher nicht blockierend und Subscriber warten zudem nicht aktiv auf neue Nachrichten, sondern werden meist per Callback über neue Nachrichten informiert. Damit wird die Verarbeitung vom Event-System aus nebenläufig getriggert.

Diese Arbeit beschäftigt sich ausschließlich mit dezentralen Publish/Subscribe-Sys\-temen, denn \ac{m2etis} zielt darauf ab, die Rechner der Nutzer in einem p2p-Netzwerk zu verbinden und darauf aufbauend die Events zu verteilen. Viele der Grundlagen in diesem Kapitel gelten sowohl für klassische zentrale wie auch dezentrale Publish/Subscribe-Systeme, allerdings müssen im verteilten Fall die Verwaltungsinformationen dezentral auf allen Knoten gespeichert werden, beziehungsweise geeignete Verteilunsalgorithmen gefunden werden. Somit relativiert sich die Dimension der räumlichen Trennung, da Publisher wie Subscriber Teil des Eventsystems sind.

Banerjee vergleicht verschiedene Arten zum Aufbau solch eines Multicast-Systemes. \enquote{mesh-first} beschreibt den expliziten Aufbau des Netzwerkes. Die Peers untereinander verändern ihre Verbindungen aufgrund bestimmter Metriken und können auch Netzwerkpartitionen beheben sind somit selbst für das Netzwerk zuständig. Der \enquote{implizte} Ansatz beschreibt Publish/Subscribe-Systeme, die auf einem Overlaynetzwerk aufsetzen und dessen Routingalgorithmus indirekt die Verteilungsstruktur bestimmt \cite{Banerjee2001Comparative}. Ein Beispiel hierfür ist Scribe, das in \Fref{chap:related:scribe} beschrieben wird.

Fiege befasst sich näher auf den Aspekt der Sicherheit und des Vertrauens zwischen Sender, Empfänger und dem Verteilungsystem ein \cite{FiegeSecurity}. Behnel stellt verschiedene Aspekte von \enquote{Quality of Service} auf verschiedenen Ebenen eines Publish/Subscribe-Systems vor. Beispielsweise \enquote{Latenz}, \enquote{Bandbreite}, \enquote{Zustellgarantien} für Nachrichten auf Netzwerkebene oder \enquote{Reihenfolge}, \enquote{Validität} oder \enquote{Authentifizierung} von Nachrichten auf Verteilungsebene. Er beschreibt wie sich einige Publish/Subscribe-System hinsichtlich der beschriebenen Aspekte verhalten \cite{BeFiMu2006PubSubQoS}. 

Grundsätzlich lassen sich Publish/Subscribe-Systeme in zwei Varianten einteilen: \emph{kanalbasiert}\index{Publish/Subscribe!kanalbasiert} und \emph{filterbasiert}\index{Publish/Subscribe!filterbasiert} \cite{Liu2003Survey}. In kanalbasierten Systemen werden die Nachrichten einzelnen Kategorien zugeordnet. Subscriber können sich für Nachrichten dieser Kategorien anmelden und bekommen diese zugestellt. Filterbasierte Systeme haben diese Einteilung nicht, stattdessen sind Nachrichten typisiert (zum Beispiel nur einfache Datentypen und Zeichenketten) und mit einem Wertebereich versehen. Bei der Anmeldung kann ein Prädikat zur Filterung angegeben werden. Der Knoten empfängt nun nur gefilterte, auf das Prädikat passende Nachrichten.

Verbindet man die Filterung von Nachrichten mit einem kanalbasiertem Ansatz gelangt man zu einem \emph{hybriden} System\index{Publish/Subscribe!hybrid}: Einer Anmeldung an einem Kanal kann ein Prädikat übergeben werden. Beispielsweise wird eine Anmeldung am Kanal für Bewegungsnachrichten über ein Gebiet eingeschränkt. Die dezentrale Filterung ist jedoch nur möglich, wenn die Nutzdaten vom System lesbar sind oder mit filterbaren Metainformationen angereichert sind. Zudem müssen die Prädikate im logisch aufgebauten Verteilungssystem bekanntgemacht werden, damit Nachrichten frühzeitig bei der Verteilung gefiltert werden können. Das von \ac{m2etis} zur Verfügung gestellte kanalbasierte Publish/Subscribe-System kann pro Kanal mit einer eigenen Filterungskomponente versehen werden und somit als hybrides System genutzt werden; dies wird in \Fref{chap:konzeption_pubsub} beschrieben.

Ein prominenter Vertreter verteilter, kanalbasierter Systeme ist Scribe, dessen Funktionsweise im nächsten Abschnitt beschrieben wird.

\subsection*{Umsetzung eines kanalbasieren Systemes am Beispiel von Scribe}
\label{chap:related:scribe}
Eine Umsetzung von Publish/Subscribe-Systemen in verteilen Systemen, ist der Aufbau eines Multicast-Trees\index{Multicast-Tree}, d.h. eines durch die Knoten im Netz gebildeten Baumes in dem die Nachrichten verteilt werden. Hierbei wird pro Kanal ein eigener Multicast-Tree aufgebaut. Am Algorithmus von Scribe \cite{Castro2002Scribe}wird diese Struktur beschrieben.

Scribe basiert auf dem strukturierten Overlay-Netzwerk Pastry \cite{Rowstron2001} und erzeugt einen vom Subscriber zum Publischer aufgebauten Baum \emph{reverse path forwarding tree} \cite{Dalal1978}.

\begin{figure}[htbp]
\centering
\resizebox{\textwidth}{!}{%
\includegraphics{grafics/multicast_tree.pdf}}
\caption{Schema eines Multicast-Trees}
\label{fig:multicast_tree}
\end{figure}

\Fref{fig:multicast_tree} zeigt ein Netzwerk mit den sechs Knoten A-F. Die Verbindungen der Knoten werden durch dünne schwarze Linien dargestellt. Beispielsweise hat Knoten C Verbindungen zu A, B, D und F.\\
Der Multicast-Tree benötigt einen Knoten, der die Wurzel (im Folgenden \emph{Root-Knoten} genannt) darstellt. Aus Hashwert des Kanalnamens wird ein Schlüssel berechnet. Derjenige Knoten, der aufgrund der Netzwerkmetrik für diesen Schlüssel zuständig ist, wird Root-Knoten des Kanals. Im abgebildeten Falle ist dies Knoten A.\\
Weiterhin hält jeder Knoten eine Liste bei ihm angemeldeter Knoten. In der Abbildung wird diese Liste durch geschweifte Klammern nach der Knotenbezeichnung dargestellt.

\paragraph*{Subscribe}
Knoten F sendet eine \emph{subscribe}-Nachricht an A. Diese Nachrichten sind in der Grafik durch gebogene gestrichelte schwarze Verbindungslinien mit Pfeil dargestellt. Das Netzwerk würde diese Nachricht über Knoten D und C an A routen. Knoten D lässt die Nachricht terminieren und trägt F in die Liste der Subscriber ein. Knoten D sendet nun selbst eine subscribe-Nachricht an A. C, über den die Nachricht geroutet wird, terminiert diese, trägt D in die Liste ein und sendet selbst eine subscribe-Nachricht an A. A erhält nun diese Nachricht und trägt C in die Liste ein. Damit sind nun insgesamt drei Nachrichten verschickt worden.\\
Wenn sich Knoten E für den Kanal einschreibt, wird die subscribe-Nachricht an A über den Knoten C geleitet. Dieser terminiert die Nachricht und fügt E der Liste hinzu. Da C selbst angemeldet ist, muss keine weitere Nachricht versendet werden.

Scribe fordert periodische Anmeldungen zur Erhöhung der Fehlertoleranz. Ist ein Knoten ausgefallen, routet das Netzwerk die Nachrichten über andere Knoten. Damit kann der Multicast-Tree wieder aufgebaut werden.

\paragraph*{Unsubscribe}
Der Austritt aus einem Kanal erfolgt ähnlich zur Anmeldung. Die Nachricht läuft nur bis zum nächsten Knoten und terminiert dort. Der Knoten entfernt den Sender der Nachricht aus seiner Liste und sendet selbst nur eine \emph{unsubscribe}-Nachricht, wenn die Liste leer ist und er selbst nicht angemeldet ist.

\paragraph*{Publish}
In \Fref{fig:multicast_tree} möchte Knoten B eine Nachricht im Kanal publizieren. B sendet darauf eine Nachricht an den Root-Knoten A, da dieser für diesen Kanal zuständig ist (gebogene türkise Linie). Nun sendet A diese Nachricht an alle Knoten in seiner Liste (gerader türkise Linie mit Pfeil). Dies ist in der Abbildung nur Knoten C. Dieser sendet sie weiter an D und E. E gibt diese Nachricht direkt an die Applikation weiter, während D die Nachricht an F schicken muss.


Hierbei ist klar ersichtlich, dass zusätzliche Nachrichten verteilt werden müssen, wenn Knoten F eine Nachricht im Kanal publizieren möchte. Diese Nachricht muss erst von Knoten F zu Knoten A wandern, damit A diese Nachricht wieder über die anderen Knoten zurücksendet. Optimierte Versionen dieses Algorithmus können hier ansetzen und zu publizierende Nachrichten nicht mehr an den Knoten senden, der ihnen diese Nachricht geschickt hat. So würde C die Nachricht nur noch an E weiterleiten.

Bayeux \cite{Zhuang2001} ist ein ähnliches System, jedoch auf Basis des Overlay-Netzwerkes Tapestry \cite{Zhao2004Tapestry}. Tapestry entspricht auch der generischen API, somit stellt dies keinen Unterschied zu Pastry dar. Im Gegensatz zu Scribe, wird bei Bayeux der Multicast-Tree vom Root-Knoten aus aufgebaut. Aufgrund der unterliegenden Routingstruktur des genutzten Overlay-Netzwerkes können sich diese Pfade unterscheiden.


Nach diesem Einblick einer möglichen Umsetzung kanalbasiertem Publish/Subscribe, gibt der kommende Abschnitt am Beispiel von Mercury eine Vorstellung davon, wie filterbasierte Systeme\index{Publish/Subscribe!filterbasiert} in einem dezentralem Netzwerk implementiert sein können.

\subsection*{Umsetzung eines filterbasierten Systemes am Beispiel von Mercury}
\label{chap:related:mercury}
Zur besseren Vorstellung einer Umsetzung für filterbasierte Publish/Subscribe-Systeme\index{Publish/Subscribe!filterbasiert} wird im folgenden Kapitel Mercury \cite{Bharambe2004Mercury} vorgestellt. Obwohl \ac{m2etis} ein kanalbasiertes Publish/Subscribe-System darstellt \cite{Fischer2010a}, ist es sinnvoll eine möglich Umsetzung eines filterbasierten Systems zu beschreiben um die grundlegenden Unterschiede der Systeme genauer auszuarbeiten. 

\paragraph*{Arbeitsweise}
Im System gibt es eine Menge an Attributen, die ihrerseits einen definierten Wertebereich haben. Jedes Attribut wird durch einen eigenen Verbund aus Knoten, den sogenannten \emph{Hub}, bearbeitet. Der Wertebereich ist dabei nicht zwingend symmetrisch auf die Knoten verteilt.

\paragraph*{Anmelden}
Eine Subscription $S$ ist ein Tupel aus Filterbedingungen über die Attribute (z.B. $S := (5 < x <= 20; y = 15)$) sowie Kontaktinformationen des Knotens. $S$ wird an einen beliebigen Knoten eines Hubs gesendet, der für das Attribut aus der Filterbedingung mit der größten Selektivität zuständig ist. Im Beispiel ist dies Attribut $y$. Im Hub wird $S$ nun zu dem Knoten weitergereicht, der den Wertebereich der Filterung abdeckt. Dort wird $S$ in einer Liste gespeichert.

\paragraph*{Publizieren}
Eine Publikation $P$ ist ebenfalls ein Tupel mit bestimmten Werten der Attribute (z.B. $P := (x = 10; y = 0)$). $P$ wird an \emph{alle} Hubs gesendet und dort zum zuständigen Knoten weitergereicht. Dieser prüft nun die Liste der gespeicherten Subscriptions gegen die neue Publikation. Stimmen beide überein, so wird $P$ an den eingeschriebenen Knoten weitergeleitet.

%\paragraph*{Offene Punkte}
%\begin{itemize*}
%\item Änderung der Attribute zur Laufzeit?
%\item Auswahl der Knoten für einen Hub?
%\item Aufteilung der Wertemenge auf die Knoten?
%\end{itemize*}

\paragraph*{Ähnliche Algorithmen}
Mirinae ist ebenfalls ein filterbasiertes Publish/Subscribe-System, stellt den Wertebereich eines Attributes jedoch als Hyperwürfel dar. Eine automatische Anpassung dieser Aufteilung ermöglicht eine schnelle Anpassung der Routingtabelle und damit einen kurzen Weg für die Nachrichten \cite{Choi2005Mirinae}.


\subsection{VON}
\label{chap:related:von}
\ac{von} ist in seinen Grundzügen stark unterschiedlich zu den bisher vorgestellten Umsetzungen. \ac{von} nutzt das \ac{p2p}-Netzwerk nicht nur als Kommunikationsmedium, sondern auch dessen Aufbau als Verteilungsstruktur des Publish/Subscribe-Systems \cite{Hu2006VON}. VON zielt auf die Verteilungsoptimierung von Events zur Positionsänderung, muss allerdings über Applikationswissen verfügen: die Position des Spielers. \ac{vast} \cite{Backhaus2007Voronoibased} greift das Konzept von \ac{von} auf und testet eine Implementierung auf OpenSIM \cite{Baumgart2007OverSim}.

\begin{figure}[htbp]
\centering
\resizebox{\textwidth}{!}{%
\includegraphics{grafics/voronoi_von_backhaus.pdf}}
\caption{Struktur eines VON-Netzwerkes (aus \cite{Backhaus2007Voronoibased})}
\label{fig:von}
\end{figure}

\Fref{fig:von} zeigt einen Aufbau eines VON-Netzwerks. Die Spielwelt wird anhand der Position der einzelnen Knoten in Voronoi-Diagramme \cite{Aurenhammer1991Voronoi} unterteilt und jedem Knoten ein eigener Bereich zugeteilt. Ein Knoten hält Verbindungen zu seinen Nachbarn und kennt angrenzende Nachbarn im Bereich seiner \ac{aoi}. \emph{Enclosing neighbors} sind angrenzende Nachbarn, deren gesamter zugehöriger Bereich innerhalb der \ac{aoi} des Knotens liegt. \emph{Boundary neighbors} sind nicht angrenzende Knoten, deren eigener Bereich nicht vollkommen innerhalb der \ac{aoi} liegt.

\textbf{Anmeldungen} im Publish/Subscribe-System sind implizit, denn \textbf{Publikationen}, also Positionsänderungen, werden von einem Knoten an alle direkt angrenzenden Nachbarn (\emph{enclosing neighbor} in \Fref{fig:von}) gesendet. Damit das System konsistent bleibt, werden dabei auch Informationen über andere Nachbarn ausgetauscht. Mit jeder Positionsänderung verändert sich die Aufteilung des Voronoi-Diagrammes und damit auch die Nachbarschaften.

VON lässt sich nicht eindeutig als kanalbasiertes oder filterbasiertes System klassifizieren. Die Fixierung auf die Position und Auswertung der \ac{aoi} kann einerseits als \emph{filterbasiertes} System mit einem einzigen Attribut und andererseits als \emph{hybrides} System mit einem  Kanal und entsprechender Filterung anhand der AOI angesehen werden.

%\missing{Beschreiben!}
%Ebenfalls mit Delauny-Triangulierung arbeitet \cite{Liebeherr2002Applicationlayer}.


Nach den Grundlagen von \ac{p2p}-Netzwerken, Publish/Subscribe-Systemen und Einblicke in verschiedene Umsetzungen, wird sich das nächste Kapitel mit der Evaluation dreier \ac{p2p}-Netzwerke beschäftigen und ein geeignetes System als Netzwerk für \ac{m2etis} auswählen.
