\chapter{Related Work}
\label{chap:related}

IPMulticast ist das Beste, Application Multicast ist toll, aber OverlayMulticast ist besser. OverlayMulticast: explizites Erstellen von RoutingKnoten. Widerspricht aber unserem Ansatz, so wenig \enquote{zentrale Server} wie möglich zu haben. 

\cite{Lao2005Comparative} % Multicast Protocols: Top, Bottom, or In the Middle

\section{NTree}
\label{chap:related:ntree}



\section{minTCO}
\label{chap:related:mintco}
\cite{citeulike:4069017}


\section{Vivaldi}
\label{chap:related:vivaldi}
Vivaldi \cite{citeulike:162250} ist ein dezentrales System, in dem sich Knoten des Netzwerkes Koordinaten so wählen, dass der Abstand zwischen zwei Koordinaten der Latenz zwischen diesen beiden Knoten entspricht.

\subsection{Arbeitsweise}
Neue Knoten im Netz wählen sich zufällig Koordinaten. Zu allen Nachbarn werden nun Ping-Nachrichten zur Messung der Laufzeit verschickt. Die Koordinaten der einzelnen Knoten werden dabei übertragen. So kann nun die eigene Koordinate anhand des Abstandes angepasst werden.

Das Paper verwendet hier eine Analogie zu gespannten Federn. Alle Knoten sind über Federn miteinander verbunden und suchen nun das gemeinsame Optimum der Federspannung.


\section{VON}
\label{chap:related:von}
\ac{von} ist ein integriertes System aus Overlaynetzwerk mit Publish/Subscribe-System und zielt direkt auf virtuelle Umgebungen wie \ac{mmog} ab \cite{Hu2006VON}. Jeder Knoten hat demnach eine Koordinate im Raum\footnote{z.B. Spielewelt} und eine \ac{aoi}. Die virtuelle Welt wird mittels Voronoi-Diagrammen in $n$ (Anzahl der Knoten im System) disjunkte Regionen eingeteilt und dadurch auf die Knoten verteilt. Ein Knoten hat direkte und angrenzende Nachbarn. Bewegt sich der Knoten im Raum, so verändert sich die Aufteilung der Regionen.

\ac{vast} \cite{Backhaus2007Voronoibased} greift das Konzept von \ac{von} auf und testet eine Implementierung auf OpenSIM \cite{Baumgart2007OverSim}.

