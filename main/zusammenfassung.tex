\chapter{Zusammenfassung und Ausblick} 
\label{chap:zus}
Diese Arbeit befasste sich mit der Konzeption und prototypischen Implementierung eines Frameworks zur optimierten Verteilung von Events in Publish/Subscribe-System auf Basis eines strukturierten p2p-Overlay Netzwerks.\\
Die Evaluation der p2p-Netzwerke hat gezeigt, dass das Konzept von Pastry/Tapestry für \ac{m2etis} geeignet ist, da vielfältige Eingriffe in Routingentscheidungen möglich sind. Ausgefallene Knoten werden zügig ersetzt und behindern das Routing von Nachrichten nur wenig. Mit  Chimera hat sich zudem ein freies System gefunden, dass dieses Konzept umsetzt. Die beispielhafte Anwendung zweier verschiendener Verteilungsstrategien, namentlich Scribe und ein Algorithmus der an VON angelegt ist, zeigen die Mächtigkeit und Offenheit des Verarbeitungsmodells. Dieses garaniert damit eine Vielzahl von Möglichkeiten die Eventverteilung zu optimieren. Die Implementierung des Publish/Subscribe-Systems nutzt viele Techniken der Template Meta-Programmierung um den zusätzlichen Verwaltungsaufwand zur Laufzeit zu minimieren. Dabei wurde stets darauf geachtet, dass das Framework ohne Kenntnis der eingesetzten Optimierungen nutzbar bleibt.

Damit das gesamte System \ac{m2etis} zu seiner vollen Blüte reifen kann, sind weitere Komponenten erforderlich. Im nächsten Schritt muss ein Kostenmodell erstellt werden, damit die bisherigen Strategien und die weiteren zu implementierenden Strategien mit ihren Kosten ausgezeichnet werden können. Für die verschiedenen Policies gilt es entsprechende Strategien zu finden, die ihrer Aufgabe, unterschiedliche Ansätze zur Optimierung beizutragen, gerecht werden. Die im System vorhande Queue zur Abarbeitung der ankommenden Nachrichten kann mit einer Priorisierung erweitert werden um wichtige Nachrichten gezielt abzuarbeiten. Zeitgleich kann das semantische Modell zur Beschreibung von Eventtypen erstellt werden und mit einer domänenspezifischen Beschreibungssprache (DSL) abgerundet werden. Der \ac{m2etis}-Optimierer ist schießlich nötig um die semantischen Beschreibungen auszuwerten und anhand des Kostenmodells und weiterer Parameter optimierte Kanäle für jeden Eventtypen zu erzeugen. Diese weiteren Parameter sind beispielsweise Einstellungen für das Netzwerk oder Intervalle für periodische Neuanmeldungen. Diese und die genauen Werte für das Kostenmodell können im Zusammenspiel mit einem Netzwerksimulator wie OverSim \cite{Baumgart2007OverSim} ausgemessen werden, für dessen Anbindung des Simulators ein Netzwerkwrapper geschrieben werden muss. Sind diese Komponenten vorhanden, kann \ac{m2etis} als Ganzes evaluiert werden um es mit anderen Systeme, wie Donnybrook \cite{Bharambe2008Donnybrook}, VON \cite{Hu2006VON} oder NTree \cite{GauthierDickey2005Using}, zu vergleichen.
