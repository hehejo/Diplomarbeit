\chapter{Zusammenfassung und Ausblick} 
\label{chap:zus}
Diese Arbeit befasste sich mit der Konzeption und prototypischen Implementierung eines Frameworks zur optimierten Verteilung von Events in einem kanalbasierten Publish/Subscribe-System auf Basis eines strukturierten p2p-Overlay-Netzwerks.\\
Zuerst wurde die Eventklassifikation anhand eines anschaulichen Szenarios eines \acp{mmog} gezeigt und dabei die möglichen Dimensionen nach \cite{Fischer2010Event} aufgezeigt. Die daran anschließende Evaluation der p2p-Netzwerke hat gezeigt, dass das Konzept von Pastry/Tapestry für \ac{m2etis} geeignet ist, da vielfältige Eingriffe in Routingentscheidungen möglich sind. Auch werden ausgefallene Knoten zügig ersetzt und behindern das Routing von Nachrichten nur wenig. Mit Chimera hat sich zudem ein freies System gefunden, dass dieses Konzept umsetzt. Die beispielhafte Anwendung verschiedener Verteilungsstrategien, namentlich Scribe und ein an VON angelehnter Algorithmus, zeigen die Mächtigkeit und Offenheit des entwickelten Verarbeitungsmodells. In diesem werden die identifizierten Dimensionen zur Klassifikation von Eventtypen umgesetzt und miteinander zur optimalen Bearbeitung von Nachrichten verbunden. Es garantiert damit eine Vielzahl von Möglichkeiten, die Eventverteilung zu optimieren. Die Implementierung des Publish/Subscribe-Systems nutzt viele Techniken der \ac{tmp}, um den Verwaltungsaufwand zur Laufzeit zu minimieren. Dabei wurde stets darauf geachtet, dass das Framework ohne Kenntnis der eingesetzten Optimierungen nutzbar bleibt.

Damit das System \ac{m2etis} als Ganzes einsetzbar ist, sind weitere Komponenten erforderlich. Im nächsten Schritt muss ein Kostenmodell erstellt werden, damit die Kosten die bisherigen Strategien und noch zu implementierenden Strategien bestimmt werden können. Für die verschiedenen Policies gilt es entsprechende Strategien zu finden, die ihrer Aufgabe, unterschiedliche Ansätze zur Optimierung beizutragen, gerecht werden. Die im System vorhandene Queue zur Abarbeitung der ankommenden Nachrichten kann mit einer Priorisierung erweitert werden, um wichtige Nachrichten gezielter abzuarbeiten. Zeitgleich sollte das semantische Modell zur Beschreibung von Eventtypen erstellt und mit einer domänenspezifischen Beschreibungssprache abgerundet werden. Der \ac{m2etis}-Optimierer ist schließlich nötig, um die semantischen Beschreibungen auszuwerten und anhand des Kostenmodells optimierte Kanäle für jeden Eventtypen zu erzeugen. Weitere Parameter sind beispielsweise Einstellungen für das Netzwerk oder Intervalle für periodische Neuanmeldungen. Diese und die genauen Werte für das Kostenmodell können im Zusammenspiel mit einem Netzwerksimulator wie OverSim \cite{Baumgart2007OverSim} ausgemessen werden, für dessen Anbindung ein Netzwerkwrapper geschrieben werden muss. Sind diese Komponenten vorhanden, kann \ac{m2etis} als Ganzes evaluiert werden um es mit anderen Systeme, wie Donnybrook \cite{Bharambe2008Donnybrook}, VON \cite{Hu2006VON} oder NTree \cite{GauthierDickey2005Using}, zu vergleichen.
