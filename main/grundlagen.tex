\chapter{Grundlagen}
\label{chap:grundlagen}

In diesem Kapitel werden die Grundlagen dieser Arbeit, \ac{p2p}-Netzwerke und Publish/Subscribe-System, erläutert. Ein Abschnitt zu betrügerischem Verhalten, dessen Aufdeckung beziehungsweise Verhinderung, beendet dieses Kapitel.

\section{Begriffsklärung}

Overlay-Netzwerke und \ac{p2p}-Netzwerke sind verschiedene Netzwerkarten, werden jedoch häufig in einem Atemzug genannt, da p2p-Netzwerke häufig auf Overlay-Netzwerken basieren. In vielen zitierten Arbeiten wird ebenfalls von \emph{p2p overlay network} gesprochen; dies drückt die Anforderung an solch ein Netzwerk aus:\\
Ein p2p Overlay-Netzwerk ist ein vom physischen Netzwerk (räumlich) unabhängiger Verbund aus gleichberechtigten Knoten, die miteinander kommunizieren können.

\paragraph{Overlay-Netzwerk} Ein Overlay-Netzwerk\index{Netzwerk!Overlay} ist ein logischer Aufsatz auf einem bestehenden Netzwerk und abstrahiert von diesem durch einen eignen Adressraum. Knoten im Netzwerk können benachbart sein, ohne eine direkte physikalische Verbindung zu haben.\\
Ein Overlay-Netzwerk bietet neben dem eigenen Adressraum auch Funktionalität zum Versand, Empfang und Routing von Nachrichten \cite{Tannenbaum2003}.

Overlay-Netzwerke sind vielschichtiger Natur. So sind z.B. der EMail-Dienst mit seinem eigenen Namensraum, Twitter oder Facebook als Overlay-Netzwerke zu bezeichnen.

\paragraph{\ac{p2p}-Netzwerk} p2p-Netzwerke\index{Netzwerk!p2p} beschreiben den Verbund von Knoten, sogenannten Peers, die miteinander gleichberechtigt kommunizieren können \cite{Steinmetz2005}. p2p-Netzwerke werden meist durch Overlay-Netzwerke realisiert, damit ist das p2p-Netzwerk vom unterliegenden physischem Netzwerk unabhängig. Grundlagen dieser Systeme sowie deren unterschiedliche Ansätze werden in \Fref[plain]{chap:grundlagen:p2p} näher besprochen.\\
Die erste Generation dieser Netzwerke, wurde häufig zum Austausch von Dateien verwendet und wird daher vielfach mit dem \emph{Filesharing} assoziiert. Jedoch werden p2p-Netzwerke auch als Kommunikationsnetze genutzt, zum Beispiel in von Sensorknoten aufgebauten Netzwerken \cite{MuneebAliandKoenLangendoen2007Case, Darlagiannis2006Peertopeer}.

\section{p2p-Netzwerke}
\label{chap:grundlagen:p2p}

p2p-Netzwerke sind die Grundlage aktueller Filesharing-Systeme wie BitTorrent. Der einfache Netzaufbau, die Fehlertoleranz bei ausfallenden Knoten sowie der Wegfall eines Servers lässt diese Netzwerke auch für Computerspiele interessant werden \cite{Knutsson2004Peertopeer, Triebel2008Peertopeer}. Roussopoulos stellt in \cite{Roussopoulos20032} einen Entscheidungsbaum für und wider den Einsatz von p2p-Netzwerken zur Verfügung. Auch für Sensornetzwerke ist der Aufbau eines p2p-Netzwerkes von Vorteil. Es können wertvolle Ressourcen (z.B. Batterie) besser genutzt werden \cite{MuneebAliandKoenLangendoen2007Case, Sioutas2009Building}, da die Kommunikation zwischen den Knoten weniger Sendeleistung benötigt als es eine Verbindungen zu einem -- eventuell weit entfernten -- Hauptrechner benötigen würde. Solche Netzwerke können beispielsweise zur Erkennung von Waldbränden genutzt werden. Sensorknoten werden über dem zu überwachenden Gebiet abgeworfen und sammeln Daten wie Temperatur und Luftfeuchtigkeit. Verbunden in einem großen p2p-Netzwerk werden die Daten untereinander ausgetauscht. Zur dezentralen Verwaltung und Abfrage genügt es an einem Sensorknoten die Daten abzufragen.

p2p-Netzwerke lassen sich grundsätzlich als \emph{unstrukturiert} oder \emph{strukturiert} klassifizieren \cite{Steinmetz2005, Lua2005Survey} und in Generationen einteilen \cite{Bo2003PeertoPeer}:
\begin{itemize*}
	\item \emph{1G} unstrukturierte Netzwerke
	\item \emph{2G} strukturierte Netzwerke
	\item \emph{3G} strukturierte Netzwerke mit Fokus auf Anonymität, Authentifizierung, Schutz vor Zensur und verschiedenen Rollen der einzelnen Knoten
\end{itemize*}

Netzwerke der dritten Generation, wie z.B. GNUnet \cite{Bennett2002GNet} werden in dieser Arbeit nicht näher betrachtet. Die zusätzlich angebotene Funktionalität wird nicht benötigt und würde ob ihrer Komplexität das zu entwickelnde System aufblähen. Die Beschreibung widmet sich im Folgenden den Netzwerken der ersten und zweiten Generation.

\subsection{Unstrukturierte Netzwerke}
Unstrukturierte Netzwerke\index{Netzwerk!Overlay!unstrukturiert} zeichnen sich dadurch aus, dass alle Informationen und Dateien durch Suchalgorithmen \cite{Lv2002} gefunden werden müssen. \\
Ein Client tritt dem Netzwerk bei und stellt seine Suchanfrage in das Netz. Werden entsprechende Peers gefunden die diese Suchanfrage beantworten können, werden zum Transfer Direktverbindungen aufgebaut. Aufgrund der einfachen (meist textbasierten) Struktur der Suchanfragen können diese einen großen Wertebereich abdecken.

Im Wesentlichen lassen sich unstrukturierte Netzwerke in die Typen \emph{zentralisiert} und \emph{dezentralisiert} einteilen.

\paragraph{zentralisiert} Knoten im Netz melden eine Liste der verfügbaren Dateien an bekannte Hauptrechner. Suchanfragen werden ebenfalls an diese Rechner gerichtet. Der Suchende erhält eine Liste von potentiellen Peers, die Dateien seiner Suchanfrage entsprechend anbieten. Diese Dateien werden über direkte Verbindungen zwischen den Peers übertragen. Einige Systeme tauschen weitere potentielle Knoten über Peers aus. Die einzelnen Hauptrechner stellen bei dieser Art von System einen \emph{single point of failure} dar \cite{Eberspaecher2005}.

\paragraph{dezentralisiert} In dezentralen Netzen gibt es keine bekannten Hauptrechner. Damit sich ein neuer Knoten in das Netz einklinken kann, muss diesem mindestens ein bestehender Knoten im Netzwerk bekannt sein. Über diesen Peer tauscht der neue Knoten Informationen aus und baut eine Nachbarschaft auf. Damit Dateien gefunden werden können, wird die Suchanfrage an alle Nachbarn geschickt, sprich die Suchanfrage wird durch das Netzwerk geflutet. Diese senden sie weiter an ihre Nachbarn.\\
Die Suchanfrage kann z.B. mit einer Anzahl an Hops oder einer TTL\footnote{Time to live} in ihrer Reichweite eingegrenzt werden. Potentielle Zyklen müssen bei dieser Art von Suche aufgelöst werden \cite{Lv2002}. Sind Peers gefunden, wird ebenfalls eine direkte Verbindung zur Datenübertragung aufgebaut. 

Beispiele für solche Netzwerke sind Napster (zentralisiert), Gnutella oder BitTorrent\footnote{\url{http://www.bittorrent.com/}} (dezentralisiert).

\subsection{Strukturierte Netzwerke}
Strukturierte Netzwerke\index{Netzwerk!Overlay!strukturiert} der zweiten Generation sind oft dezentraler Natur. Ein Datensatz muss nicht gesucht werden, da anhand der Struktur des Netzes die zuständigen Knoten berechnet werden können. Daten können ebenfalls via Direktverbindung übertragen werden, meist wird jedoch das Netzwerk selbst zum Versand genutzt. Hierbei werden die in Nachrichten gepackte Datensätze über verschiedene Peers geroutet. Der dezentralen Art dieser Netze ist geschuldet, dass ein neuer Knoten mindestens einen Peer aus dem Netzwerk kennen muss. Viele Systeme gehen davon aus, dass der neue Knoten aus einer Liste von Peers, den ihm nächst gelegenem Knoten\footnote{Nähe im Sinne von Latenz beziehungsweise räumlicher Nähe} wählen kann und über diesen den Eintritt in das Netz anstößt.

Strukturierte Netzwerke nutzen die Technik der \ac{dht}, da diese als Grundlage einer Vielzahl von unterschiedlichen Anwendungen dienen kann \cite{Wehrle2005, Ghodsi2006AlgorithmsDHT}. Grundsätzlich ist jedem Knoten ein Schlüssel aus dem Schlüsselraum des Netzwerkes zugeordnet. Ein Knoten ist dabei - je nach genutzter Metrik - für einen Bereich aus dem Schlüsselraum zuständig. Für das zu suchende Datum wird nun der Hashwert berechnet - dieser entspricht einem Schüssel. Der suchende Knoten kennt damit den zuständigen Peer und nutzt das zugrunde liegende Overlay-Netzwerk zur Datenübertragung.\\
Hier wird deutlich, dass diese Netzwerke eher der Kommunikation diesen als dem Austausch von Dateien.

Darauf aufbauend gibt es unterschiedliche Systeme die sich hinsichtlich Organisation, Routing, Ein- und Austritt von Knoten und dem Verhalten im Fehlerfall unterscheiden \cite{Goetz2005, Lua2005Survey}.

Beispiele für solche Netzwerke sind Chord \cite{Hosseini2007Survey}, Pastry \cite{Rowstron2001}, Tapestry \cite{Zhao2001Tapestry,Zhao2004Tapestry} oder CAN \cite{Ratnasamy2001Scalable}. Auf diese wird im \Fref[plain]{chap:evaluation_p2p}, das sich der Evaluation widmet, näher eingegangen.


\subsection{Betrügerisches Verhalten in p2p-Netzwerken}
\label{chap:grundlagen:cheating}
Betrügerisches Verhalten\index{Betrügerisches Verhalten} (Cheating) ist in p2p-Systemen einfacher als in Client/Server-Systemen, da die Kommunikation nicht zwingend über einen (vom Betreiber kontrollierten) Server läuft. Mit gefälschten Nachrichten für Statusmeldungen wie Positionsänderungen oder zur Entscheidungsfindung über die Reihenfolge von Aktionen können andere Knoten massiv benachteiligt werden. Das Wissen über das zugrunde liegende System wird genutzt um gezielt wichtige Aspekte des Spieles zu manipulieren.

Webb \cite{Webb2007Cheating} gibt eine Übersicht über Betrug in Netzwerkspielen und geht hier auch auf die Unterschiede zwischen Client/Server-Systemen und p2p-Systemen ein. Cheating wird in vier grundlegende Bereiche eingeteilt: \emph{Game level cheats}, \emph{Application level cheats}, \emph{Protocol level cheats} und \emph{Infrastructure level cheats}. Mögliche Verfahren wie \emph{Lockstep} \cite{Baughman2007}, das die Spielzeit in Runden einteilt, werden vorgestellt und ausgewertet.

Als Beispiel für einen \emph{Protocol level cheat} wird das Ausnutzen des \emph{Dead-Reckoning}-Verfahrens\index{Dead-Reckoning} \cite{Pantel2002} beschrieben. Dead-Reckoning kaschiert ausgefallene Nachrichten und die Latenz des Netzwerks und bietet durch vorberechnete Aktionen der anderen Spieler einen konstanten Spielfluss. Beispielsweise wird bei einer ausbleibenden Positionsnachricht eines Mitspielers angenommen, dessen Avatar bewegt sich ohne Richtungsänderung weiter. Sendet ein Spieler innerhalb eines bestimmten Intervalles keine Nachrichten mehr, muss das System in Aktion treten und beispielsweise die Welt in einen -- auf allen Rechnern -- konsistenten Zustand bringen. Sendet ein Spieler nur noch die minimal benötigen Nachrichten, können in der Zeit bis zum geforderten zu versendenden Update die Nachrichten der anderen Spieler ausgewertet werden und das Spiel somit zu eigenen Gunsten beeinflusst werden. Algorithmen, die diese Problematik angehen, werden entwickelt \cite{Aggarwal2005}.

Kabus \cite{Kabus2007Design, Kabus2009} beschreibt grundsätzliche Techniken, die in p2p-Systemen genutzt werden, um Betrug aufzudecken beziehungsweise zu verhindern. Beispielsweise können für eine Entscheidung über eine Spieleraktion zufällige Knoten gewählt werden. Diese validieren die Aktionen und können effektiv Betrug verhindern. Zum Beispiel ist die Aufnahme eines Gegenstand in der virtuellen Welt nur erlaubt, wenn andere Mitspieler bestätigen, dass sich der entsprechende Spieler an der Position des Gegenstands aufhält. Solche Verfahren gehen jedoch mit einer erhöhten Anzahl an Nachrichten einher.

In weiteren Arbeiten beschäftigt sich Castro mit sicherem Routing in p2p Overlay-Netzwerken \cite{Castro2002Secure}, Kabus \cite{Kabus2005Addressing} und Dautermann \cite{Dautermann2007} im Speziellen mit verteilten \acp{mmog}. Ferretti geht genauer auf den Aspekt des Betrugs mit \enquote{Zeitspielereien} ein \cite{Ferretti2008Cheating}.

In diesem Abschnitt wurde ein Überblick möglicher Betrugsarten gegeben und auch Möglichkeiten zur Verhinderung dargestellt. Betrügerisches Verhalten\index{Betrügerisches Verhalten} und dessen Verhinderung beziehungsweise Vermeidung sind nicht Thema dieser Arbeit. Das zu entwickelnde Framework bietet genug Freiheiten, betrugsresistente Publish/Subscribe-Algorithmen einzusetzen oder entsprechende andere Ansätze an das Framework anzubinden.


\subsection{Generische key-based Routing API}
\label{chap:grundlagen:api}
Verschiedene Typen und Implementierungen von p2p-Netzwerke haben sehr unterschiedliche Schnittstellen zur Programmierung, obwohl die Anwendungsfälle meist gleicher Art sind. Dabek moniert diese unterschiedlichen Schnittstellen der verschiedenen strukturierten p2p-Netzwerke die häufig auf dem Prinzip des \ac{kbr} agieren \cite{Dabek2003Towards}. Dies mache es aus Entwicklersicht schwer, vom Netzwerk zu abstrahieren und dieses gegebenenfalls zu wechseln. Er untersucht verschiedene strukturierte p2p-Netzwerke und identifiziert ein minimales Set von Funktionen. Diese sind in zwei Zuständigkeitsbereiche aufgeteilt: \enquote{Routing messages} und \enquote{Routing state access}.\\
Erstere umfassen drei Methoden, von denen zwei \enquote{Upcalls} sind. Ein Upcall ist ein Callback der Applikation die vom Netzwerk aufgerufen wird. \texttt{route} ermöglicht das Senden einer Nachricht. Dabei kann ein Hinweis an das Netzwerk übergeben werden, über welchen Knoten die Nachricht als nächstes geroutet werden soll. Der Upcall \texttt{forward} wird auf jedem Knoten aufgerufen, der eine Nachricht weiterleitet. Als Parameter werden der Schlüssel, die Nachricht und der nächste Routingknoten übergeben. Alle Parameter können verändert werden und der Nachrichtenversand kann auch terminiert werden. Auf dem eigentlichen Empfänger der Nachricht wird zusätzlich vor \texttt{upcall} noch \texttt{forward} aufgerufen. Als Parameter werden der Schlüssel und die Nachricht übergeben.\\
\Fref{fig:routing_kbr} verdeutlicht die Aufrufreihenfolge. Angenommen eine Nachricht an Knoten \texttt{0x7b} wird von Knoten \texttt{0x3f} über \texttt{0x4a} und \texttt{0x64} geroutet. An Knoten \texttt{0x3f} wird die Nachricht mittels \texttt{route} an das Netzwerk übergeben. Die Applikation wird an den Knoten \texttt{0x4a} und \texttt{0x64} durch den Callback \texttt{forward} über die Nachricht informiert. Am Zielknoten \texttt{0x7b} wird diese jedoch nicht direkt mittels \texttt{deliver} an die Applikation übergeben, sondern erst einer Bearbeitung durch \texttt{forward} zugeführt. Jeder Knoten kann in \texttt{forward} anhand des übergebenen nächsten Routingknotens feststellen, ob er der zuständige Knoten ist. Dadurch wird dem Empfänger die Möglichkeit gegeben, die Nachricht zu verändern um diese bespielsweise an einen anderen Knoten weiterzuleiten.

\begin{figure}[htbp]
\centering
\resizebox{\textwidth}{!}{%
\includegraphics{grafics/routing_kbr.pdf}}
\caption{Routing einer Nachricht und Aufruf der Callbacks}
\label{fig:routing_kbr}
\end{figure}


\enquote{Routing state access} umfasst vier Methoden und einen Upcall. Dieser, \texttt{update}, informiert über Knoten, die das Netzwerk betreten oder es verlassen. Um eine Liste möglicher Knoten, die als möglicher nächster Routinghop in Frage kommen, wird die Methode \texttt{local\_lookup} genutzt. \texttt{neighborSet} liefert eine Liste der Nachbarn des aktuellen Knotens. Soll ein Datensatz auf mehreren Knoten gespeichert werden, ist die Methode \texttt{replicaSet} nützlich. Diese liefert für einen gegebenen Schlüssel eine Liste aller Knoten, die für den Datensatz geeignet sind. Um den Bereich zu ermitteln, für den der eigene Knoten zuständig ist, wird die Methode \texttt{range} angeboten.

Dabek zeigt, dass dieses \acf{api} ausreichend ist, um darauf aufbauend verschiedene Anwendungen wie Publish/Subscribe zu implementieren. Er beschreibt ebenfalls, wie einige Systeme -- namentlich CAN, Chord, Pastry und Tapestry -- durch diese \ac{api} anzupassen sind.


\section{Verteilte Publish/Subscribe-Systeme}
\label{chap:grundlagen:pubsub}
Konzeptionell lassen sich Publish/Subscribe-Systeme als eventbasierte Systeme betrachten. Auf Grund ihres Aufbaus und der Skalierung  in orthogonalen Dimensionen\index{Publish/Subscribe!orthogonale Dimensionen} \enquote{Raum}, \enquote{Zeit} sowie \enquote{Synchronisation} eignen sich diese gut zur Verteilung von Events in dezentralen Umgebungen \cite{PatrickTh2003Many, Cugola2002Using}.

\begin{figure}[htbp]
\centering
\includegraphics{grafics/pubsub_black_box.pdf}
\caption{Schema eines Publish/Subscribe-Systems}
\label{fig:pubsub_black_box}
\end{figure}

Publisher und Subscriber werden durch das Event-System voneinander getrennt, wie es in \Fref{fig:pubsub_black_box} dargestellt ist.  Publisher und Subscriber sind räumlich voneinander getrennt. Ein Publisher übergibt die Nachricht an das System und hält weder direkte Verbindung mit den Subscribern noch muss der Publisher alle Subscriber kennen. Diese Trennung bezieht sich nicht nur auf verschiedene Komponenten einer Applikation, sondern kann auch über Applikations- oder gar Rechnergrenzen gehen. Die zeitliche Trennung beschreibt, dass sich ein Subscriber am System anmelden kann, obwohl kein Publisher vorhanden ist, analog können Nachrichten publiziert werden, ohne dass Empfänger eingeschrieben sind. Je nach Implementierung können Nachrichten zwischengespeichert werden, um diese neuen Subscribern zuzustellen. Bei einem Fernaufrufsystem wie \emph{remote proceduce call (RPC)} \cite{Birrell1984Implementing} ist dies nicht möglich, da die Gegenseite existieren muss. Das Senden einer Nachricht ist für den Publisher nicht blockierend und Subscriber warten zudem nicht aktiv auf neue Nachrichten, sondern werden meist per Callback über neue Nachrichten informiert. Damit wird die Verarbeitung vom Event-System aus nebenläufig getriggert.

Diese Arbeit beschäftigt sich ausschließlich mit dezentralen Publish/Subscribe-Sys\-temen, denn \ac{m2etis} zielt darauf ab, die Rechner der Nutzer in einem p2p-Netzwerk zu verbinden und darauf aufbauend die Events zu verteilen. Viele der Grundlagen in diesem Kapitel gelten sowohl für klassische zentrale als auch dezentrale Publish/Subscribe-Systeme, allerdings müssen im verteilten Fall die Verwaltungsinformationen ebenfalls dezentral auf allen Knoten gespeichert, beziehungsweise geeignete Verteilungsalgorithmen gefunden werden. Somit relativiert sich die Dimension der räumlichen Trennung, da Publisher wie Subscriber Teil des Eventsystems sind.

Banerjee vergleicht verschiedene Arten zum Aufbau solch eines Multicast-Systemes. \enquote{mesh-first} beschreibt den expliziten Aufbau des Netzwerkes. Die Peers verändern ihre Verbindungen aufgrund bestimmter Metriken und können auch Netzwerkpartitionen beheben und sind somit selbst für das Netzwerk zuständig. Der \enquote{implizte} Ansatz beschreibt Publish/Subscribe-Systeme, die auf einem Overlaynetzwerk aufsetzen und dessen Routingalgorithmus indirekt die Verteilungsstruktur bestimmt \cite{Banerjee2001Comparative}. Ein Beispiel hierfür ist Scribe, das in \Fref{chap:related:scribe} beschrieben wird.

Fiege befasst sich näher mit dem Aspekt der Sicherheit und des Vertrauens zwischen Sender, Empfänger und dem Verteilungsystem \cite{FiegeSecurity}. Behnel stellt verschiedene Aspekte von \enquote{Quality of Service} auf verschiedenen Ebenen eines Publish/Sub\-scribe-Systems vor. Beispielsweise \enquote{Latenz}, \enquote{Bandbreite}, \enquote{Zustellgarantien} für Nachrichten auf Netzwerkebene oder \enquote{Reihenfolge}, \enquote{Validität} oder \enquote{Authentifizierung} von Nachrichten auf Verteilungsebene. Er beschreibt das Verhalten einiger Publish/Subscribe-Systeme hinsichtlich der beschriebenen Aspekte \cite{BeFiMu2006PubSubQoS}. 

Grundsätzlich lassen sich Publish/Subscribe-Systeme in zwei Varianten einteilen: \emph{kanalbasiert}\index{Publish/Subscribe!kanalbasiert} und \emph{filterbasiert}\index{Publish/Subscribe!filterbasiert} \cite{Liu2003Survey}. In kanalbasierten Systemen werden die Nachrichten einzelnen Kategorien zugeordnet. Subscriber können sich für Nachrichten dieser Kategorien anmelden und bekommen diese zugestellt. Filterbasierte Systeme haben diese Einteilung nicht, stattdessen sind Nachrichten typisiert (zum Beispiel nur einfache Datentypen und Zeichenketten) und mit einem Wertebereich versehen. Bei der Anmeldung kann ein Prädikat zur Filterung angegeben werden. Der Knoten empfängt nun nur gefilterte, auf das Prädikat passende Nachrichten.

Verbindet man die Filterung von Nachrichten mit einem kanalbasierten Ansatz, gelangt man zu einem \emph{hybriden} System\index{Publish/Subscribe!hybrid}: Einer Anmeldung an einem Kanal kann ein Prädikat übergeben werden. Beispielsweise wird eine Anmeldung am Kanal für Bewegungsnachrichten auf ein Gebiet eingeschränkt. Die dezentrale Filterung ist jedoch nur möglich, wenn die Nutzdaten vom System lesbar oder mit filterbaren Metainformationen angereichert sind. Zudem müssen die Prädikate im logisch aufgebauten Verteilungssystem bekanntgemacht werden, damit Nachrichten frühzeitig bei der Verteilung gefiltert werden können. Beispielsweise kann der Eventtyp \emph{Gildennachricht}\footnote{vergleiche \Fref[plain]{chap:grundlagen:szenario}} auf einem filterbaren Kanal abgebildet werden. Als Prädikat kann die Gildenzugehörigkeit des Avatars oder eine Liste der Gildenmitglieder, von denen Nachrichten erwünscht sind, angegeben werden. Hyper ist ein Beispiel eines solchen hybriden Systems \cite{Zhang}.\\
Das von \ac{m2etis} zur Verfügung gestellte kanalbasierte Publish/Subscribe-System kann pro Kanal mit einer eigenen Filterungskomponente versehen werden und somit als hybrides System genutzt werden; dies wird in \Fref{chap:konzeption_pubsub} beschrieben.

Ein prominenter Vertreter verteilter, kanalbasierter Systeme ist Scribe, dessen Funktionsweise im nächsten Abschnitt beschrieben wird.

\subsection*{Umsetzung eines kanalbasieren Systemes am Beispiel von Scribe}
\label{chap:related:scribe}
Eine Umsetzung von Publish/Subscribe-Systemen in verteilen Systemen, ist der Aufbau eines Multicast-Trees\index{Multicast-Tree}, d.h. eines durch die Knoten im Netz gebildeten Baumes in dem die Nachrichten verteilt werden. Hierbei wird pro Kanal ein eigener Multicast-Tree aufgebaut. Am Algorithmus von Scribe \cite{Castro2002Scribe}wird diese Struktur beschrieben.

Scribe basiert auf dem strukturierten Overlay-Netzwerk Pastry \cite{Rowstron2001} und erzeugt einen vom Subscriber zum Publischer aufgebauten Baum \emph{reverse path forwarding tree} \cite{Dalal1978}.

\begin{figure}[htbp]
\centering
\resizebox{\textwidth}{!}{%
\includegraphics{grafics/multicast_tree.pdf}}
\caption{Schema eines Multicast-Trees}
\label{fig:multicast_tree}
\end{figure}

\Fref{fig:multicast_tree} zeigt ein Netzwerk mit den sechs Knoten A-F. Die Verbindungen der Knoten werden durch dünne schwarze Linien dargestellt. Beispielsweise hat Knoten C Verbindungen zu A, B, D und F.\\
Der Multicast-Tree benötigt einen Knoten, der die Wurzel (im Folgenden \emph{Root-Knoten} genannt) darstellt. Aus Hashwert des Kanalnamens wird ein Schlüssel berechnet. Derjenige Knoten, der aufgrund der Netzwerkmetrik für diesen Schlüssel zuständig ist, wird Root-Knoten des Kanals. Im abgebildeten Falle ist dies Knoten A.\\
Weiterhin hält jeder Knoten eine Liste bei ihm angemeldeter Knoten. In der Abbildung wird diese Liste durch geschweifte Klammern nach der Knotenbezeichnung dargestellt.

\paragraph*{Subscribe}
Knoten F sendet eine \emph{subscribe}-Nachricht an A. Diese Nachrichten sind in der Grafik durch gebogene gestrichelte schwarze Verbindungslinien mit Pfeil dargestellt. Das Netzwerk würde diese Nachricht über Knoten D und C an A routen. Knoten D lässt die Nachricht terminieren und trägt F in die Liste der Subscriber ein. Knoten D sendet nun selbst eine subscribe-Nachricht an A. C, über den die Nachricht geroutet wird, terminiert diese, trägt D in die Liste ein und sendet selbst eine subscribe-Nachricht an A. A erhält nun diese Nachricht und trägt C in die Liste ein. Damit sind nun insgesamt drei Nachrichten verschickt worden.\\
Wenn sich Knoten E für den Kanal einschreibt, wird die subscribe-Nachricht an A über den Knoten C geleitet. Dieser terminiert die Nachricht und fügt E der Liste hinzu. Da C selbst angemeldet ist, muss keine weitere Nachricht versendet werden.

Scribe fordert periodische Anmeldungen zur Erhöhung der Fehlertoleranz. Ist ein Knoten ausgefallen, routet das Netzwerk die Nachrichten über andere Knoten. Damit kann der Multicast-Tree wieder aufgebaut werden.

\paragraph*{Unsubscribe}
Der Austritt aus einem Kanal erfolgt ähnlich zur Anmeldung. Die Nachricht läuft nur bis zum nächsten Knoten und terminiert dort. Der Knoten entfernt den Sender der Nachricht aus seiner Liste und sendet selbst nur eine \emph{unsubscribe}-Nachricht, wenn die Liste leer ist und er selbst nicht angemeldet ist.

\paragraph*{Publish}
In \Fref{fig:multicast_tree} möchte Knoten B eine Nachricht im Kanal publizieren. B sendet darauf eine Nachricht an den Root-Knoten A, da dieser für diesen Kanal zuständig ist (gebogene türkise Linie). Nun sendet A diese Nachricht an alle Knoten in seiner Liste (gerader türkise Linie mit Pfeil). Dies ist in der Abbildung nur Knoten C. Dieser sendet sie weiter an D und E. E gibt diese Nachricht direkt an die Applikation weiter, während D die Nachricht an F schicken muss.


Hierbei ist klar ersichtlich, dass zusätzliche Nachrichten verteilt werden müssen, wenn Knoten F eine Nachricht im Kanal publizieren möchte. Diese Nachricht muss erst von Knoten F zu Knoten A wandern, damit A diese Nachricht wieder über die anderen Knoten zurücksendet. Optimierte Versionen dieses Algorithmus können hier ansetzen und zu publizierende Nachrichten nicht mehr an den Knoten senden, der ihnen diese Nachricht geschickt hat. So würde C die Nachricht nur noch an E weiterleiten.

Bayeux \cite{Zhuang2001} ist ein ähnliches System, jedoch auf Basis des Overlay-Netzwerkes Tapestry \cite{Zhao2004Tapestry}. Tapestry entspricht auch der generischen API, somit stellt dies keinen Unterschied zu Pastry dar. Im Gegensatz zu Scribe, wird bei Bayeux der Multicast-Tree vom Root-Knoten aus aufgebaut. Aufgrund der unterliegenden Routingstruktur des genutzten Overlay-Netzwerkes können sich diese Pfade unterscheiden.


Nach diesem Einblick in eine mögliche Umsetzung eines kanalbasierten Publish/Sub\-scribe-Systems gibt der kommende Abschnitt am Beispiel von Mercury eine Vorstellung davon, wie filterbasierte Systeme\index{Publish/Subscribe!filterbasiert} in einem dezentralen Netzwerk implementiert sein können.

\subsection*{Umsetzung eines filterbasierten Systemes am Beispiel von Mercury}
\label{chap:related:mercury}
Zur besseren Vorstellung einer Umsetzung für filterbasierte Publish/Subscribe-Systeme\index{Publish/Subscribe!filterbasiert} wird im folgenden Kapitel Mercury \cite{Bharambe2004Mercury} vorgestellt. Obwohl \ac{m2etis} ein kanalbasiertes Publish/Subscribe-System darstellt \cite{Fischer2010a}, ist es sinnvoll eine möglich Umsetzung eines filterbasierten Systems zu beschreiben um die grundlegenden Unterschiede der Systeme genauer auszuarbeiten. 

\paragraph*{Arbeitsweise}
Im System gibt es eine Menge an Attributen, die ihrerseits einen definierten Wertebereich haben. Jedes Attribut wird durch einen eigenen Verbund aus Knoten, den sogenannten \emph{Hub}, bearbeitet. Der Wertebereich ist dabei nicht zwingend symmetrisch auf die Knoten verteilt.

\paragraph*{Anmelden}
Eine Subscription $S$ ist ein Tupel aus Filterbedingungen über die Attribute (z.B. $S := (5 < x <= 20; y = 15)$) sowie Kontaktinformationen des Knotens. $S$ wird an einen beliebigen Knoten eines Hubs gesendet, der für das Attribut aus der Filterbedingung mit der größten Selektivität zuständig ist. Im Beispiel ist dies Attribut $y$. Im Hub wird $S$ nun zu dem Knoten weitergereicht, der den Wertebereich der Filterung abdeckt. Dort wird $S$ in einer Liste gespeichert.

\paragraph*{Publizieren}
Eine Publikation $P$ ist ebenfalls ein Tupel mit bestimmten Werten der Attribute (z.B. $P := (x = 10; y = 0)$). $P$ wird an \emph{alle} Hubs gesendet und dort zum zuständigen Knoten weitergereicht. Dieser prüft nun die Liste der gespeicherten Subscriptions gegen die neue Publikation. Stimmen beide überein, so wird $P$ an den eingeschriebenen Knoten weitergeleitet.

%\paragraph*{Offene Punkte}
%\begin{itemize*}
%\item Änderung der Attribute zur Laufzeit?
%\item Auswahl der Knoten für einen Hub?
%\item Aufteilung der Wertemenge auf die Knoten?
%\end{itemize*}

\paragraph*{Ähnliche Algorithmen}
Mirinae ist ebenfalls ein filterbasiertes Publish/Subscribe-System, stellt den Wertebereich eines Attributes jedoch als Hyperwürfel dar. Eine automatische Anpassung dieser Aufteilung ermöglicht eine schnelle Anpassung der Routingtabelle und damit einen kurzen Weg für die Nachrichten \cite{Choi2005Mirinae}.


\subsection{VON}
\label{chap:related:von}
\ac{von} ist in seinen Grundzügen stark unterschiedlich zu den bisher vorgestellten Umsetzungen. \ac{von} nutzt das \ac{p2p}-Netzwerk nicht nur als Kommunikationsmedium, sondern auch dessen Aufbau als Verteilungsstruktur des Publish/Subscribe-Systems \cite{Hu2006VON}. VON zielt auf die Verteilungsoptimierung von Events zur Positionsänderung, muss allerdings über Applikationswissen verfügen: die Position des Spielers. \ac{vast} \cite{Backhaus2007Voronoibased} greift das Konzept von \ac{von} auf und testet eine Implementierung auf OpenSIM \cite{Baumgart2007OverSim}.

\begin{figure}[htbp]
\centering
\resizebox{\textwidth}{!}{%
\includegraphics{grafics/voronoi_von_backhaus.pdf}}
\caption{Struktur eines VON-Netzwerkes (aus \cite{Backhaus2007Voronoibased})}
\label{fig:von}
\end{figure}

\Fref{fig:von} zeigt einen Aufbau eines VON-Netzwerks. Die Spielwelt wird anhand der Position der einzelnen Knoten in Voronoi-Diagramme \cite{Aurenhammer1991Voronoi} unterteilt und jedem Knoten ein eigener Bereich zugeteilt. Ein Knoten hält Verbindungen zu seinen Nachbarn und kennt angrenzende Nachbarn im Bereich seiner \ac{aoi}. \emph{Enclosing neighbors} sind angrenzende Nachbarn, deren gesamter zugehöriger Bereich innerhalb der \ac{aoi} des Knotens liegt. \emph{Boundary neighbors} sind nicht angrenzende Knoten, deren eigener Bereich nicht vollkommen innerhalb der \ac{aoi} liegt.

\textbf{Anmeldungen} im Publish/Subscribe-System sind implizit, denn \textbf{Publikationen}, also Positionsänderungen, werden von einem Knoten an alle direkt angrenzenden Nachbarn (\emph{enclosing neighbor} in \Fref{fig:von}) gesendet. Damit das System konsistent bleibt, werden dabei auch Informationen über andere Nachbarn ausgetauscht. Mit jeder Positionsänderung verändert sich die Aufteilung des Voronoi-Diagrammes und damit auch die Nachbarschaften.

VON lässt sich nicht eindeutig als kanalbasiertes oder filterbasiertes System klassifizieren. Die Fixierung auf die Position und Auswertung der \ac{aoi} kann einerseits als \emph{filterbasiertes} System mit einem einzigen Attribut und andererseits als \emph{hybrides} System mit einem  Kanal und entsprechender Filterung anhand der AOI angesehen werden.

%\missing{Beschreiben!}
%Ebenfalls mit Delauny-Triangulierung arbeitet \cite{Liebeherr2002Applicationlayer}.


Nach den bisher dargestellten Grundlagen von \ac{p2p}-Netzwerken, Publish/Subscribe-Systemen und Einblicken in verschiedene Umsetzungen beschäftigt sich das nächste Kapitel mit der Evaluation dreier \ac{p2p}-Netzwerke m ein geeignetes System als Netzwerk für \ac{m2etis} auswählen.



