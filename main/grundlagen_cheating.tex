\subsection{Betrügerisches Verhalten in p2p-Netzwerken}
\label{chap:grundlagen:cheating}
Betrügerisches Verhalten\index{Betrügerisches Verhalten} (Cheating) ist in p2p-Systemen einfacher als in Client/Server-Systemen, da die Kommunikation nicht zwingend über einen (vom Betreiber kontrollierten) Server läuft. Mit gefälschten Nachrichten für Statusmeldungen wie Positionsänderungen oder zur Entscheidungsfindung über die Reihenfolge von Aktionen können andere Knoten massiv benachteiligt werden. Das Wissen über das zugrunde liegende System wird genutzt um gezielt wichtige Aspekte des Spieles zu manipulieren.

Webb \cite{Webb2007Cheating} gibt eine Übersicht über Betrug in Netzwerkspielen und geht hier auch auf die Unterschiede zwischen Client/Server-Systemen und p2p-Systemen ein. Cheating wird in vier grundlegende Bereiche eingeteilt: \emph{Game level cheats}, \emph{Application level cheats}, \emph{Protocol level cheats} und \emph{Infrastructure level cheats}. Mögliche Verfahren wie \emph{Lockstep} \cite{Baughman2007}, das die Spielzeit in Runden einteilt, werden vorgestellt und ausgewertet.

Als Beispiel für einen \emph{Protocol level cheat} wird das Ausnutzen des \emph{Dead-Reckoning}-Verfahrens\index{Dead-Reckoning} \cite{Pantel2002} beschrieben. Dead-Reckoning kaschiert ausgefallene Nachrichten und die Latenz des Netzwerks und bietet durch vorberechnete Aktionen der anderen Spieler einen konstanten Spielfluss. Beispielsweise wird bei einer ausbleibenden Positionsnachricht eines Mitspielers angenommen, dessen Avatar bewegt sich ohne Richtungsänderung weiter. Sendet ein Spieler innerhalb eines bestimmten Intervalles keine Nachrichten mehr, muss das System in Aktion treten und beispielsweise die Welt in einen -- auf allen Rechnern -- konsistenten Zustand bringen. Sendet ein Spieler nur noch die minimal benötigen Nachrichten, können in der Zeit bis zum geforderten zu versendenden Update die Nachrichten der anderen Spieler ausgewertet werden und das Spiel somit zu eigenen Gunsten beeinflusst werden. Algorithmen, die diese Problematik angehen, werden entwickelt \cite{Aggarwal2005}.

Kabus \cite{Kabus2007Design, Kabus2009} beschreibt grundsätzliche Techniken, die in p2p-Systemen genutzt werden, um Betrug aufzudecken beziehungsweise zu verhindern. Beispielsweise können für eine Entscheidung über eine Spieleraktion zufällige Knoten gewählt werden. Diese validieren die Aktionen und können effektiv Betrug verhindern. Zum Beispiel ist die Aufnahme eines Gegenstand in der virtuellen Welt nur erlaubt, wenn andere Mitspieler bestätigen, dass sich der entsprechende Spieler an der Position des Gegenstands aufhält. Solche Verfahren gehen jedoch mit einer erhöhten Anzahl an Nachrichten einher.

In weiteren Arbeiten beschäftigt sich Castro mit sicherem Routing in p2p Overlay-Netzwerken \cite{Castro2002Secure}, Kabus \cite{Kabus2005Addressing} und Dautermann \cite{Dautermann2007} im Speziellen mit verteilten \acp{mmog}. Ferretti geht genauer auf den Aspekt des Betrugs mit \enquote{Zeitspielereien} ein \cite{Ferretti2008Cheating}.

In diesem Abschnitt wurde ein Überblick möglicher Betrugsarten gegeben und auch Möglichkeiten zur Verhinderung dargestellt. Betrügerisches Verhalten\index{Betrügerisches Verhalten} und dessen Verhinderung beziehungsweise Vermeidung sind nicht Thema dieser Arbeit. Das zu entwickelnde Framework bietet genug Freiheiten, betrugsresistente Publish/Subscribe-Algorithmen einzusetzen oder entsprechende andere Ansätze an das Framework anzubinden.
