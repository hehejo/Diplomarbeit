\subsection{Betrügerisches Verhalten in p2p-Netzwerken}
\label{chap:grundlagen:cheating}
Betrügerisches Verhalten\index{Betrügerisches Verhalten} (Cheating) ist in p2p-Systemen einfacher als in Client/Server-Systemen, da die Kommunikation nicht zwingend über einen (vom Betreiber kontrollierten) Server läuft. Mit gefälschten Nachrichten für Statusmeldungen, wie Positionsänderungen, oder zur Entscheidungsfindung über die Reihenfolge von Aktionen können andere Knoten massiv benachteiligt werden. Das Wissen über das zugrunde liegende System wird genutzt um gezielt wichtige Aspekte des Spieles zu manipulieren.

Webb \cite{Webb2007Cheating} gibt eine Übersicht über Betrug in Netzwerkspielen und geht hier auch auf die Unterschiede zwischen Client/Server-Systemen und p2p-Systemen ein. Cheating wird in vier grundlegende Bereiche eingeteilt: \emph{Game level cheats}, \emph{Application level cheats}, \emph{Protocol level cheats} und \emph{Infrastructure level cheats}. Mögliche Verfahren wie \emph{Lockstep} \cite{Baughman2007}, das die Spielzeit in Runden einteilt, werden vorgestellt und ausgewertet.

Als Beispiel für einen \emph{Protocol level cheat} wird das Ausnutzen des \emph{Dead-Reckoning}-Verfahrens\index{Dead-Reckoning} \cite{Pantel2002} beschrieben. Dead-Reckoning kaschiert die Latenz im Netzwerk und bietet dem Spieler durch vorberechnete Aktionen der anderen Spieler einen konstanten Spielfluss. Hierbei akzeptiert das System eine gewisse Anzahl an ausgefallenen/verlorenen Updates eines Knotens bevor unterschiedliche Aktionen aus Spielsicht durchgeführt werden.\\
Hält ein Spieler nun eigene Updates zurück, können in der Zeit bis zum geforderten Update die Informationen der anderen Spieler ausgewertet werden und das Spiel kann somit zu eigenen Gunsten beeinflusst werden. Algorithmen die diese Problematik angehen, werden entwickelt \cite{Aggarwal2005}.

Kabus \cite{Kabus2007Design, Kabus2009} beschreibt grundsätzliche Techniken, die in p2p-Systemen genutzt werden um Betrug aufzudecken beziehungsweise zu verhindern. Beispielsweise können für eine Konsenses\index{Betrug!Konsenses} über Spieleraktionen zufällig Knoten gewählt werden. Diese validieren die Aktionen und können effektiv Betrug verhindern. Solche Verfahren gehen mit einer erhöhten Anzahl an Nachrichten einher.

In weiteren Arbeiten beschäftigt sich Castro mit sicherem Routing in p2p Overlay-Netzwerken \cite{Castro2002Secure}, Kabus \cite{Kabus2005Addressing} und Dautermann \cite{Dautermann2007} im speziellen mit verteilten \acp{mmog}. Ferretti geht genauer auf den Aspekt des Betrugs mit \enquote{Zeitspielereien} ein \cite{Ferretti2008Cheating}.

In diesem Abschnitt würde ein Überblick möglicher Betrugsarten gegeben und auch Möglichkeiten zur Verhinderung dargestellt. Betrügerisches Verhalten\index{Betrügerisches Verhalten} und dessen Verhinderung beziehungsweise Vermeidung sind nicht Thema dieser Arbeit. Das zu entwickelnde Framework bietet genug Freiheiten betrugsresistente Publish/Subscribe-Algorithmen einzusetzen oder entsprechende andere Systeme an das Framework anzubinden.
