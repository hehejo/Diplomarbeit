\section{Aufbau von M$^2$etis}
\label{chap:grundlagen:aufbau_metis}

Dieser Abschnitt erläutert den Aufbau von \ac{m2etis} \cite{Fischer2010a, Fischer2010Event} und zeigt den Rahmen dieser Diplomarbeit auf. 

\begin{figure}[htbp]
\centering
\resizebox{\textwidth}{!}{%
\includegraphics{grafics/metis_aufbau.pdf}}
\caption{Architekturübersicht von \ac{m2etis}}
\label{fig:metis_aufbau}
\end{figure}

\Fref{fig:metis_aufbau} zeigt die Aufteilung des Systemes in \emph{Spiel} und \emph{\ac{m2etis}} sowie die zeitliche Trennung in \emph{Konzeption} und \emph{Laufzeit}.

Im ersten Schritt werden die Events im Spiel identifiziert und mit ihrer Semantik beschrieben. Auf der linken Seite finden sich beispielhaft verschiedene Dimensionen (nach \cite{Fischer2010a}) sowie deren unterschiedliche Implementierungen. Diese sind dem emph{\ac{m2etis} Optimierer} bekannt. Dieser verarbeitet im zweiten Schritt unter Zuhilfenahme des \emph{semantischen Modells} und des \emph{Kostenmodells} die semantische Beschreibung der Events und erzeugt einen optimierten \emph{Kanal} für jeden Event. Wie in der Grafik ersichtlich ist, benötigt das Spiel keine  Implementierungsdetails der einzelnen Kanäle um diese im 3. Schritt zur Laufzeit zu nutzen.

Diese Arbeit ist für die Anbindung an das Netzwerk, die Konzeption und prototypische Entwicklung des Publish/Subscribe-Systems sowie der Implementerungen einiger Algorithmen verschiedener Dimensionen zuständig, so dass das Framework einsetzbar ist.
