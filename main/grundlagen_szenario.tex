\section{Szenario}
\label{chap:grundlagen:szenario}
Angenommen, wir sehen einem Spieler eines typischen \ac{mmog} über die Schulter. Er wird seinen Avatar durch die Welt bewegen und mit anderen Spielern interagieren. Ein einfaches Quest zu Spielbeginn ist beispielsweise die Suche nach einem Gegenstand, der von einem Gegener bewacht wird. Der Spieler muss zum Beispiel eine Blume finden, den Gegner besiegen und die Blume aufnehmen. Diese wird zum Abschluss des Quests dem Auftraggeber gebracht und der Avatar berichtet darüber in der Gilde (ein Zusammenschluss von Spielern).\\
An diesem kleinen Beispielszenario lassen sich die verschiedenen Eventtypen ``Bewegung'', ``Sprechen'', ``Kampf'' und ``Gegenstand aufnehmen'' identifizieren und anhand verschiedner Dimensionen wie ``Kontext'', ``Persitenz'', ``Synchronität'' oder ``Validität'' klassifizieren \cite{Fischer2010Event}. Die Bewegung des Avatars durch die Welt ist nur für diejenigen Spieler interessant, die diesen Avatar sehen können. Das Gespräch interessiert jedoch nur die Spieler, deren Avatare in diesem Gespräche beteiligt sind. Die einzelnen Events vom Typ ``Kampf'' müssen hingegen synchronisiert werden um einen fairen Schlagabtausch zu gewährleisten. Das Event ``Blume aufgenommen'' muss zudem abgespeichert werden um die Blume im Inventar des Avatars abzulegen und über eine Spielesession hinweg aktuell zu halten. Dieser Event ist jedoch nur für diesen Benutzer interessant.\\
Die Kommunikation erfolgt in \acp{mmog} oft nach dem Client-Server Prinzip. Der Spielhersteller stellt eine gewisse Anzahl von Servern bereit auf denen die virtuelle Welt erzeugt wird. Damit die Welt konsistent blebt, erfolgt die Übertragung der Events meist per Broadcast an alle Spieler. Bei vielen Spielern skaliert dieses System naturgemäß schlecht. Oftmals werden verschiedene Instanzen der virtuellen Welt gestartet um die Anzahl der Spieler begrenzen zu können. Fischer schlägt, in Anlehnung an Triebel \cite{Triebel2008Peertopeer}, den Einsatz eines \ac{p2p}-Netzwerks zur Kommunikation zwischen den einzelnen Spielern vor. Zur Verteilung der Events in solch einem Netzwerk wird ein kanalbasiertes Publish/Subscribe-System vorgeschlagen, bei dem jeder Eventtyp als eigenständiger Kanal interpretiert wird, vorgeschlagen. Die Verbindung aus \ac{p2p}-Netzwerk und Publish/Subscribe-System wird in \ac{m2etis} aufgriffen und ermöglicht neue Dimensionen der Verteilungsoptimierung von Events. Events vom Typ ``Bewegung'' sollen anders behandelt werden als Events vom Typ ``Gegenstand aufnehmen'' und daraus eine Entlastung des Kommunikationssystems sowie ein gesteigerter Spielgenuss gewonnen werden  \cite{Fischer2010a}.
