\label{chap:grundlagen:szenario}
Angenommen, wir sehen einem Spieler eines typischen \ac{mmog} über die Schulter. Er wird seinen Avatar durch die Welt bewegen und mit anderen Spielern interagieren. Ein einfaches Quest zu Spielbeginn ist beispielsweise die Suche nach einem, von einem Gegner bewachten, Gegenstand. Unser Spieler muss eine Blume finden, den Gegner besiegen und die Blume aufnehmen. Diese wird zum Abschluss des Quests dem Auftraggeber gebracht und der Avatar berichtet darüber in der Gilde (ein Zusammenschluss von Spielern).\\
An diesem kleinen Beispielszenario lassen sich die verschiedenen Eventtypen \enquote{Bewegung'', ``Sprechen'', ``Kampf'' und ``Gegenstand aufnehmen'' identifizieren und anhand verschiedner Dimensionen wie ``Kontext'', ``Persitenz'', ``Synchronität'' oder ``Validität'' klassifizieren \cite{Fischer2010Event}. Die Bewegung des Avatars durch die Welt ist nur für diejenigen Spieler interessant, die diesen Avatar sehen können. Das Gespräch interessiert jedoch nur die Spieler, deren Avatare in diesem Gespräche beteiligt sind. Die einzelnen Events vom Typ ``Kampf'' müssen hingegen synchronisiert werden um einen fairen Schlagabtausch zu gewährleisten. Das Event ``Blume aufgenommen} muss zudem abgespeichert werden um die Blume im Inventar des Avatars abzulegen und über eine Spielesession hinweg aktuell zu halten. Dieser Event ist jedoch nur für diesen Benutzer interessant.

\missing{Jetzt was sind Eventtypen, Events und die Dimensionen zur Klassifizierung von Eventtypen?}


