\subsection{Generische API}
\label{chap:grundlagen:api}
Dabek moniert in \cite{Dabek2003Towards} die unterschiedlichen Schnittstellen der verschiedenen strukturierten p2p-Netzwerke die auf dem Prinzip der \ac{kbr} agieren. Dies mache es aus Entwicklersicht schwer, vom Netzwerk zu abstrahieren und dieses gegebenenfalls zu wechseln. Er untersucht verschiedene strukturierte p2p-Netzwerke, identifiziert ein Set von Funktionen  und zeigt wie darauf aufbauend verschiedene Systeme wie Publish/Subscribe implementiert werden können. Dabek beschreibt ebenfalls wie einige bekannte Systeme (CAN, Chord, Pastry und Tapestry) diese Anforderungen erfüllen.

Dabek teilt die API in zwei Teile auf: ``Routing messages'' und ``Routing state access''.\\
``Routing messages'' umfasst drei Methoden, von denen zwei ``Upcalls''\footnote{Im Grunde Methoden der Applikation die vom Netzwerk aufgerufen werden} sind. \emph{route} ermöglicht das Senden einer Nachricht an einen Knoten. Dabei kann ein Hinweis an das Netzwerk übergeben werden, über welchen Knoten die Nachricht als nächstes geroutet werden soll. Der Upcall \emph{forward} wird auf jedem Knoten aufgerufen, der eine Nachricht weiterleitet. Als Parameter werden der Schlüssel, die Nachricht und der nächste Routingknoten übergeben. Alle Parameter können verändert werden und der Nachrichtenversand kann auch terminiert werden. Auf dem eigentlichen Empfänger der Nachricht wird zusätzlich nach forward der Upcall \emph{deliver} aufgerufen. Als Parameter werden der Schlüssel und die Nachricht übergeben.\\
``Routing state access'' umfasst vier Methoden und einen Upcall. Dieser, \emph{update}, informiert über Knoten die das Netzwerk betreten oder es verlassen. Um eine Liste möglicher Knoten die als möglicher nächster Routinghop in Frage kommen, wird die Methode \emph{local\_lookup} genutzt. \emph{neighborSet} liefert eine Liste der Nachbarn des akutellen Knotens. Soll ein Datensatz auf mehreren Knoten gespeichert werden, ist die Methode \emph{replicaSet} nützlich. Diese liefert für ein gegebeben Schlüssel eine Liste aller Knoten, die für den Datensatz geeinet sind. Um den Bereich zu ermitteln, für den der eigene Knoten zuständig ist, wird die Methode \emph{range} angeboten.
