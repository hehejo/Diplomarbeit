\section{p2p-Netzwerke}
\label{chap:grundlagen:p2p}
IP-Multicast ist eine Möglichkeit, ein verteiltes \ac{p2p}-System aufzubauen \cite{Deering1990Multicast}. \ac{p2p}-Netzwerke setzen jedoch meist auf Overlay-Netzwerke\index{Netzwerk!Overlay} auf, da IP-Multicast nicht weit verbreitet ist\footnote{Es erfordert spezielle Router und entsprechende Infrastruktur.}. Overlay-Netzwerke sind ein logischer Aufsatz über ein bestehendes Netzwerk und abstrahieren von diesem durch einen eigenen Adressraum: Knoten im Netzwerk können benachbart sein ohne eine direkte physikalische Verbindung zu haben. Ein Overlay-Netzwerk bietet neben dem eigenen Adressraum auch Funktionalität zum Versand, Empfang und Routing von Nachrichten \cite{Tannenbaum2003} und sind dabei vielschichtiger Natur: Der E-Mail-Dienst mit seinem eigenen Namensraum, Twitter oder Facebook ist als Overlay-Netzwerk zu bezeichnen. p2p-Netzwerke\index{Netzwerk!p2p} selbst beschreiben den Verbund von Knoten, sogenannten Peers, die miteinander gleichberechtigt kommunizieren können \cite{Steinmetz2005}. p2p-Netzwerke sind vom physischen Netzwerk unabhängig, da sie oft auf Overlay-Netzwerken aufbauen. Obwohl Overlay-Netzwerke und \ac{p2p}-Netzwerke verschiedene Netzwerkarten sind, werden diese jedoch häufig in einem Atemzug genannt. In vielen zitierten Arbeiten wird ebenfalls von \emph{p2p overlay network} gesprochen; dies drückt die Anforderung an solch ein Netzwerk aus:\\
Ein p2p-Overlay-Netzwerk ist ein vom physischen Netzwerk (räumlich) unabhängiger Verbund aus gleichberechtigten Knoten, die miteinander kommunizieren können.

Die erste Generation dieser Netzwerke wurde oft zum Austausch von Dateien verwendet und wird daher vielfach mit dem Begriff \emph{Filesharing} assoziiert. p2p-Netzwerke werden jedoch auch als Kommunikationsnetze genutzt \cite{Darlagiannis2006Peertopeer}. Der einfache Netzaufbau, die Fehlertoleranz bei ausfallenden Knoten sowie der Wegfall teurer Server lässt diese Netzwerkart auch für Computerspiele interessant werden \cite{Knutsson2004Peertopeer, Triebel2008Peertopeer}. Roussopoulos stellt in \cite{Roussopoulos20032} einen allgemeinen Entscheidungsbaum für und wider den Einsatz von p2p-Netzwerken zur Verfügung. Auch für Sensornetzwerke ist der Aufbau eines p2p-Netzwerkes von Vorteil \cite{MuneebAliandKoenLangendoen2007Case}. Es können wertvolle Ressourcen (z.B. Batterie) besser genutzt werden \cite{Sioutas2009Building}, da die Kommunikation zwischen den Knoten weniger Sendeleistung benötigt als eine Verbindungen zu einem -- eventuell weit entfernten -- Hauptrechner. Solche Netzwerke werden beispielsweise zur Erkennung von Waldbränden eingesetzt. Sensorknoten werden über dem zu überwachenden Gebiet abgeworfen und sammeln Daten wie Temperatur und Luftfeuchtigkeit. Verbunden in einem großen p2p-Netzwerk werden die Daten untereinander ausgetauscht. Zur dezentralen Verwaltung und Abfrage genügt es an einem Sensorknoten die Daten abzufragen.

p2p-Netzwerke lassen sich grundsätzlich als \emph{unstrukturiert} oder \emph{strukturiert} klassifizieren \cite{Steinmetz2005, Lua2005Survey} und in Generationen einteilen \cite{Bo2003PeertoPeer}:
\begin{itemize*}
	\item \emph{1G} unstrukturierte Netzwerke
	\item \emph{2G} strukturierte Netzwerke
	\item \emph{3G} strukturierte Netzwerke mit Fokus auf Anonymität, Authentifizierung, Schutz vor Zensur und verschiedenen Rollen der einzelnen Knoten
\end{itemize*}

\subsection{Unstrukturierte Netzwerke}
Unstrukturierte Netzwerke\index{Netzwerk!Overlay!unstrukturiert} zeichnen sich dadurch aus, dass alle Informationen und Dateien durch Suchalgorithmen gefunden werden müssen \cite{Lv2002}. \\
Ein Client tritt dem Netzwerk bei und stellt seine Suchanfrage in das Netz. Werden entsprechende Peers gefunden, die diese Suchanfrage beantworten können, werden zum Transfer Direktverbindungen aufgebaut. Aufgrund der einfachen (meist textbasierten) Struktur der Suchanfragen können diese einen großen Wertebereich abdecken.

Im Wesentlichen lassen sich unstrukturierte Netzwerke in die Typen \emph{zentralisiert} und \emph{dezentralisiert} einteilen.

\paragraph{zentralisiert} Knoten im Netz melden eine Liste der verfügbaren Dateien an bekannte Hauptrechner. Suchanfragen werden ebenfalls an diese Rechner gerichtet. Der Suchende erhält eine Liste möglicher Peers, die Dateien seiner Suchanfrage entsprechend anbieten. Diese Dateien werden über direkte Verbindungen zwischen den Peers übertragen. Die einzelnen Hauptrechner stellen bei dieser Art von System einen \emph{single point of failure} dar \cite{Eberspaecher2005}.

\paragraph{dezentralisiert} In dezentralen Netzen gibt es keine bekannten Hauptrechner. Damit sich ein neuer Knoten in das Netz einklinken kann, muss diesem mindestens ein bestehender Knoten im Netzwerk bekannt sein. Über diesen Peer tauscht der neue Knoten Informationen aus und baut eine Nachbarschaft auf. Damit Dateien gefunden werden können, wird die Suchanfrage an alle Nachbarn geschickt, die diese wiederum an alle Nachbarn senden. Das heißt, die Suchanfrage wird durch das Netzwerk geflutet.\\
Die Suchanfrage kann mit einer Anzahl an Hops oder einer TTL\footnote{Time to live} in ihrer Reichweite eingegrenzt werden. Potenzielle Zyklen müssen bei dieser Art von Suche erkannt und aufgelöst werden \cite{Lv2002}. Sind Peers gefunden, wird ebenfalls eine direkte Verbindung zur Datenübertragung aufgebaut. 

Beispiele für solche Netzwerke sind Napster (zentralisiert), Gnutella oder BitTorrent (dezentralisiert).

\subsection{Strukturierte Netzwerke}
Strukturierte Netzwerke\index{Netzwerk!Overlay!strukturiert} der zweiten Generation sind oft dezentraler Natur. Ein Datensatz muss nicht gesucht werden, da anhand der Struktur des Netzes die zuständigen Knoten berechnet werden können. Daten können ebenfalls via Direktverbindung übertragen werden, meist wird jedoch das Netzwerk selbst zum Versand genutzt. Hierbei werden die in Nachrichten gepackten Datensätze über verschiedene Peers geroutet. Ein neuer Knoten muss mindestens einen Peer aus dem Netzwerk kennen, um diesem beitreten zu können. Viele Systeme gehen davon aus, dass der neue Knoten aus einer Liste von Peers den ihm nächst gelegenen Knoten\footnote{Nähe im Sinne von Latenz beziehungsweise räumlicher Nähe} wählen kann und über diesen den Eintritt in das Netz anstößt.

Strukturierte Netzwerke nutzen häufig die Technik der \acf{dht}\footnote{Siehe \Fref{chap:dht} im Anhang}, da diese als Grundlage einer Vielzahl von unterschiedlichen Anwendungen dienen kann \cite{Wehrle2005, Ghodsi2006AlgorithmsDHT}. Grundsätzlich ist jedem Knoten und jedem Datensatz ein Schlüssel aus dem Schlüsselraum des Netzwerkes zugeordnet. Jeder Knoten ist dabei für einen Bereich aus dem Schlüsselraum zuständig, dieser Bereich wird netzwerkabhängig ermittelt\footnote{Bespielsweise anhand der numerischen Nähe von Schlüsseln oder der Reihenfolge der Schlüssel.}. Für einen Datensatz ist damit der zuständige Knoten bekannt und das zu Grunde liegende Overlay-Netzwerk kann zur Datenübertragung genutzt werden.\\
Hier wird deutlich, dass diese Netzwerke eher der Kommunikation dienen als dem Austausch von Dateien. Darauf aufbauend gibt es unterschiedliche Systeme die sich hinsichtlich Organisation, Routing, Ein- und Austritt von Knoten und dem Verhalten im Fehlerfall unterscheiden \cite{Goetz2005, Lua2005Survey}. Beispiele für solche Netzwerke sind Chord \cite{Hosseini2007Survey}, Pastry \cite{Rowstron2001}, Tapestry \cite{Zhao2001Tapestry,Zhao2004Tapestry} oder CAN \cite{Ratnasamy2001Scalable}. 

Für \ac{m2etis} sind strukturierte Netzwerke besser geeignet, da sie als Kommunikationsmedium besser nutzbar sind als unstrukturierte p2p-Systeme, bei denen meist ein massenhafter Versand der Nachricht über alle Nachbarn (\enquote{flooding}) genutzt wird. Dies widerspricht klar einer optimerten Eventverteilung im gesamten System. Weiterhin lassen sich Datensätze und Zuständigkeiten in strukturierten p2p-Netzwerken einfach berechnen, was auch die Entwicklung der Publish/Subscribe-Komponente begünstigt. Netzwerke der dritten Generation, wie zum Beispiel GNUnet \cite{Bennett2002GNet}, werden in dieser Arbeit nicht näher betrachtet. Die zusätzlich angebotene Funktionalität wird einerseits nicht benötigt oder wird nur für einzelne Eventtypen benötigt. Bespielsweise benötigt der Eventtyp \enquote{Bewegung} keine authentifizierte Quelle und soll ebenso nicht anonym übertragen werden. In \Fref[plain]{chap:evaluation_p2p}, das sich der Auswahl eines geeigneten p2p-Netzwerkes zur Verwendung mit \ac{m2etis} widmet, werden daher drei strukturierten \ac{p2p}-Netzwerke der zweiten Generation (Chord, Pastry und CAN) beschrieben und miteinander verglichen.


