\chapter{Voronoi-Diagramme und Delaunay-Triangulation}
\label{chap:voronoi}


\begin{figure}[htbp]
\centering
\includegraphics{grafics/voronoi_grund_aus_aurenhammer.pdf}
\caption{Voronoi-Diagramm für acht Punkte (aus \cite{Aurenhammer1991Voronoi})}
\label{fig:voronoi_grund}
\end{figure}





\begin{figure}[htbp]
\centering
\subfigure[Dualität von Voronoi-Diagramm und De\-launay-Triangulation (aus \cite{Aurenhammer1991Voronoi})]{
	\includegraphics{grafics/voronoi_dual_aus_aurenhammer.pdf}
	\label{fig:voronoi_dual}
}
\subfigure[Bezug eines Voronoi-Diagrammes zur konvexen Hülle (aus \cite{Aurenhammer1991Voronoi})]{
	\includegraphics{grafics/voronoi_konvex_aus_aurenhammer.pdf}
	\label{fig:voronoi_konvex}
}
\end{figure}


\cite{Aurenhammer1991Voronoi}

\cite{Fortune1987, Dwyer1987} %Erzeugen von Voronoi-Diagrammen (Sweepline und Co)
