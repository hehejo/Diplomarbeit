\section{p2p Overlay Netzwerke}
\label{chap:grundlagen:overlay}

\subsection{Anforderungen}
\begin{itemize}
\item geringer Overhead (Latenz!)
\item einfach Ordnung der Nachrichten
\item sichere Zustellung (einiger Nachrichten)
\item Fehlertolerant bei Ausfall einzelner Knoten
\item Skalierbar
\item Eingriff in Routingentscheidungen
\item Bestimmung der Netztiefe (max. Anzahl an Hops zwischen zwei Knoten)



\end{itemize}

\paragraph{Geringe Latenz} Schnelle Reaktionszeiten und Nachrichtenübermittlung sind bei \ac{mmog} unverzichtbar. Ebenfalls müssen größere Nachrichten (bsp. Update der Welt) schnell übertragen werden.

\paragraph{Ordnung der Nachrichten} Das System muss eine einfache Ordnung der Nachrichten unterstützen. So ist bsp. die Reihenfolge der Nachrichten \emph{HEADSHOT} und \emph{MOVE} relevant.

\paragraph{Gichere Zustellung} Einige Nachrichten sind systemkritisch und müssen unbedingt zugestellt werden. Ein Fehlen von Nachrichten wie z.B. der Farbwechsel der Spielerrüstung oder eine Änderung im Kontostand sind im Gegensatz zu einem verlorenen \emph{HEADSHOT} nicht spielentscheidend. 

\paragraph{Fehlertoleranz bei Knotenausfall} Fallen Knoten aus, so soll sich das Netz ohne großen Kommunikationsaufwand selst reparieren. Eine Netzwerkpartition ist jedoch in dieser Arbeit kein Hindernisgrund, da ein Hauptserver immer erreichbar ist und über diesen das Netz wieder verbunden werden kann.

\paragraph{Skalierbar} 

\paragraph{Eingriff in Routingentscheidungen}

\paragraph{Bestimmung der Netztiefe}




\subsection{generic API}
\label{chap:grundlagen:overlay:generic_api}
\cite{citeulike:6643572} %towards

\subsection{Evaluation}
Da der Prototyp dieser Arbeit sowie das Umfeld in C++ entwickelt wird, gibt es damit eine weitere Anforderung an der Overlay-Netzwerk: Es sollte eine C/C++-Library geben, damit das Overlay-Netzwerk einfach aus bestehendem Code zu nutzen ist ohne kostenintensieve Brücken zwischen bsp. Java und C++ schlagen zu müssen. Da betriebssystemübergreifend entwickelt und getestet wird, ist es vorteilhaft, wenn der Quellcode zur Verfügung steht und verändert werden kann. Dadurch ist es möglich eventuellen Netzwerk- oder Threadcode auf Libraries von Boost\footnote{Zahlreiche Libraries für C++: http://www.boost.org} zu portieren und damit eine übergreifende Benutzung zu ermöglichen.

\cite{Castro2003} %evaluation

\cite{Fahmy2003} %characterizing

\cite{citeulike:780210} %pastry

\cite{Zhao2001} %Tapestry Chimera
