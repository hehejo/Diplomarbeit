Grundlagen!
\section{Begriffsklärung}
In dieser Arbeit wird ein \emph{Publish/Subscribe System über einem peer2peer overlay Netzwerk} beschrieben.

\paragraph{Peer2Peer-Netzwerk} Peer2Peer-Netzwerke beschreiben den Verbund von Knoten die miteinander, über dem System zugrunde liegenden Strukturen, kommunizieren können. Grundlagen dieser Systeme sowie deren unterschiedliche Ansätze werden in Kapitel \ref{chap:grundlagen:pubsub} näher besprochen.

\paragraph{Overlay Netzwerk} Ein \emph{overlay network} beschreibt den Umstand, dass Knoten eines Netzwerkes logisch verbunden sind, obwohl es z.B. keine physische direkte Verbindung zwischen diesen Knoten geben muss. Als Beispiel können hier selbst einfache Client/Server-Verbindungen über TCP/IP gesehen werden, da hier zwischen den beiden Rechnern weitere Knotenpunkte (Router, uvm.) liegen können. In p2p-Netzwerken hat ein Knoten oft eine Nachbarschaft von anderen Knoten ohne dass diese in einer direkten Verbindung miteinander stehen müssen. 


\label{chap:grundlagen:overlay}
\section{Peer-To-Peer Overlay Netzwerke}
\label{chap:grundlagen:overlay}
structured peer-to-peer overlay vs unstructured peer-to-peer overlay

\subsection{Anforderungen}
\begin{itemize}
\item Eingriff in Routingentscheidungen
\item Bestimmung der Netztiefe (max. Anzahl an Hops zwischen zwei Knoten)
\item geringer Overhead (Latenz!)
\item Fehlertolerant bei Ausfall einzelner Knoten
\item Skalierbar
\end{itemize}

\subsection{generic API}
\cite{citeulike:6643572} %towards

\subsection{Pastry}
\cite{citeulike:780210} %pastry
Tapestry, Pastry, Chimera


\label{chap:grundlagen:pubsub}
\section{Publish/Subscribe-Systeme}
\label{chap:grundlagen:pubsub}

\subsection{Kanalbasiert}
Ein prominenter Vertreter dieser Art ist Scribe \cite{citeulike:345316}, dessen Funktionsweise in Kapitel \ref{chap:related:scribe} genau beschrieben wird.

\subsection{Filterbasiert}
\label{chap:grundlagen:pubsub:filterbased}
\cite{citeulike:854573} %mercury
\cite{citeulike:6674153} %Caguya
\cite{citeulike:4291} %Cluster

\subsection{Anforderungen}
\begin{itemize}
\item Stabil gegenüber \emph{churn}, häufige Wechsel der Mitgliedschaft
\item minimale Anzahl an Zombieknoten \footnote{Knoten die Nachrichten routen/transportieren müssen obwohl sie diese Nachricht nicht interesiert}
\item geringer Overhead (Latenz!)
\item Skalierbar
\end{itemize}

