\section{Begriffsklärung}
In dieser Arbeit wird ein \emph{Publish/Subscribe-System über einem \ac{p2p} overlay Netzwerk} beschrieben.

\paragraph{Overlay Netzwerk} Ein Overlay-Netzwerk\index{Netzwerk!Overlay} ist ein logischer Aufsatz auf einem bestehenden Netzwerk und abstrahiert von diesem. Weiterhin zeichnen sie sich dadurch aus, dass ein eigener Adressraum genutzt wird und so das unterliegende Netzwerk überlagert wird. Knoten können so benachbart sein, ohne eine direkte physikalische Verbindung zu haben. Das Overlay-Netzwerk bietet neben dem eigenen Adressraum auch Funktionalität zum Versand, Empfang und Routing von Nachrichten.

\missing{Irgendein Zitat (Buch)!}

\paragraph{\ac{p2p}-Netzwerk} p2p-Netzwerke\index{Netzwerk!p2p} beschreiben den Verbund von Knoten, sog. Peers, die miteinander gleichberechtigt kommunizieren können. p2p-Netzwerke werden meist durch Overlay-Netzwerke realisiert, da Techniken wie bsp. IP-Multicast \cite{Deering1990Multicast} nicht weit verbreitet sind. Durch deren angebotenen Funktionen können viele Arten von p2p-Systemen implementiert werden. Grundlagen dieser Systeme sowie deren unterschiedliche Ansätze werden in Kapitel \ref{chap:evaluation_p2p} näher besprochen. Meist wird unter p2p \emph{Filesharing} verstanden, was aber nicht immer der Fall sein ist. Gegeinete p2p-Netzwerke können auch als Kommunikationsnsetz genutzt werden.

\missing{Irgendein Zitat (Buch)!}

Im Folgenden werden die Begriffe \emph{p2p-Netzwerk} und \emph{Overlay-Netzwerk} in dieser Arbeit synonym benutzt. In vielen zitierten Arbeiten wird von \emph{p2p-overlay networks} gesprochen und das drückt die Anforderung an das Netzwerk gut aus: Ein Netzwerk aus gleichberechtigen Knoten die miteinander kommunizieren können.

\paragraph{Publish/Subscribe-System} Ein Publish/Subscribe-System\index{Publish/Subscribe} beschreibt ein Abosystem. Ein Subscriber schreibt sich bei einem Publisher ein und wird von diesem über Veränderungen benachrichtigt. Solch ein System kann kanalbasiert oder filterbasiert sein. In Kapitel \ref{chap:grundlagen:pubsub} sind diese Systeme weitaus genauer aufgeführt. So werden aktuelle Systeme gegen die Anforderungen dieser Arbeit geprüft und daraus Konzepte für das zu entwickelnde System gezogen.

\missing{Irgendein Zitat (Buch)!}

\section{p2p-Netzwerke}
Dieses Kapitel widmet sich den grundsätzlichen Ausprägungen von p2p-Netzwerken. Obwohl in dieser Arbeit keine Netze nach dem Motto \emph{Information finden und per Direktverbindung übertragen} sondern eher nach dem Motto \emph{Daten geschickt zwischen den Knoten verteilen} gebraucht werden, werden die verschiedenen Typen am Aspekt des \emph{Filesharing} erklärt, da dieser recht eingängig ist.

\missing{oh, das klingt hässlich}

\subsection{Unstrukturierte Netzwerke}
Unstrukturierte Netzwerke\index{Netzwerk!Overlay!unstruktuiert} zeichnen sich dadurch aus, dass alle Informationen/Dateien durch Suchalgorithmen gefunden werden müssen. Hierbei gibt es im Wesentlichen zwei unterschiedliche Arten.

\paragraph{zentralisiert} Knoten im Netz melden ihre verfügbaren Dateien an diverse Hauptrechner. Suchanfragen werden ebenfalls an diese Hauptrechner gerichtet. Der Suchende erhält eine List von potentiellen Knoten. Die eigentlichen Dateien werden dann über direkte Verbindungen zwischen den Peers übertragen. Einige Systeme tauschen auch weitere potentielle Knoten über Peers aus. Die einzelnen Hauptrechner stellen bei diesen System einen \emph{single point of failure} dar.

\paragraph{dezentralisiert} Da es keine Hauptserver gibt, muss dem neuen Knoten mindestens ein Knoten im Netz bekannt sein, damit er sich in das Netz einklinken kann. Über diesen bekannten Peer baut der neue Knoten eine Nachbarschaft auf. Damit Dateien im Netz gefunden werden können, muss die Suchanfrage durch das Netz geflutet werden. Sind Peers gefunden, so wird ebenfalls eine direkte Verbindung zur Datenübertragung aufgebaut.

Beispiele für solche Netzwerke sind bsp. Napster, Gnutella oder BitTorrent\footnote{http://www.bittorrent.com/}.

\subsection{Strukturierte Netzwerke}
Im Gegensatz zu unstrukturierten Netzwerken sind strukturierte Netzwerke\index{Netzwerk!Overlay!strukturiert} oft dezentraler Natur. Für den Einstieg in das Netz muss dem neuen Knoten ebenfalls ein Knoten im Netz bekannt sein. Jeder Knoten hat einen eindeutigem Schlüssel, welcher einem Hashwert entspricht. Dieser Hashwert kann über verschiedenste Daten berechnet werden, seien es Dateiinhalte oder bsp. Benennungen. Ein Knoten ist für die Daten zuständig deren Hashwert seinem Schlüssel entsprechen. Dabei werden die Schlüssel gleichmäßig über die Knoten verteilt.

Somit kann ein beliebiger Knoten im Netz schnell und einfach den Knoten ermitteln, der für ein gefordertes Datenpaket zuständig ist und seine Nachrichten an diesen Knoten durch das Netz routen lassen. Jedoch können mit diesem Ansatz Knoten für Daten zuständig sein, die ihnen weder gehören noch an denen sie Interesse zeigen.

Die Kommunikation zwischen den einzelnen Knoten erfolgt dann meist nicht über direkte Verbindungen, sondern nutzt ebenfalls das zugrunde liegende Overlay-Netzwerk zur Datenübertragung.

\missing{DHT, CAN?}

\missing{Muss schöner werden!}


\section{Publish/Subscribe-Systeme}
\label{chap:grundlagen:pubsub}

\subsection{Kanalbasiert}
Ein prominenter Vertreter dieser Art ist Scribe \cite{citeulike:345316}, dessen Funktionsweise in Kapitel \ref{chap:related:scribe} genau beschrieben wird.

\subsection{Filterbasiert}
\label{chap:grundlagen:pubsub:filterbased}
\cite{citeulike:854573} %mercury
\cite{citeulike:6674153} %Caguya
\cite{citeulike:4291} %Cluster

\subsection{Anforderungen}
\begin{itemize}
\item Stabil gegenüber \emph{churn}, häufige Wechsel der Mitgliedschaft
\item minimale Anzahl an Zombieknoten \footnote{Knoten die Nachrichten routen/transportieren müssen obwohl sie diese Nachricht nicht interesiert}
\item geringer Overhead (Latenz!)
\item Skalierbar
\end{itemize}


\section{Betrug}\index{Betrug}
\label{chap:grundlagen:cheating}
Betrügerisches Verhalten (Cheating) ist in p2p-Systemen einfacher als in Client/Server-Systemen, da die Kommunikation und damit das Statusupdate sowie Entscheidungsfindungen nicht zwingend über einen (vom Betreiber kontrollierten) Masterserver laufen. Mit gefälschten Nachrichten können andere Knoten werden. Das Wissen über das zugrundeliegende System kann genutzt werden um gezielt wichtige Entscheidungen zu manipulieren.

Webb \cite{Webb2007Cheating} gibt eine Übersicht über Betrug in Netzwerkspielen und geht hier auch auf die Unterschiede zwischen Client/Server-Systemen und p2p-Systemen ein. Cheating wird in vier grundlegende Bereiche eingeteilt: \emph{Game level cheats}, \emph{Application level cheats}, \emph{Protocol level cheats} und \emph{Infrastructure level cheats}. Mögliche Verfahren wie \emph{Lockstep} \cite{Baughman2007}, das die Spielzeit in Runden einteilt werden vorgestellt und ausgewertet.

Als Beispiel für einen \emph{protocol level cheat} wir das Ausnutzen des \emph{\index{Dead-Reckoning}}-Verfahren \cite{Pantel2002} beschrieben. Dead-Reckoning wird benutzt um die Latenz im Netzwerk zu kaschieren um anderen Spielern durch vorberechnete Aktionen einen konstanten Spielfluss zu bieten. Hierbei aktzeptiert das System eine gewisse Anzahl an ausgefallenen/verlorenen Updates eines Knotens bevor untschiedliche Aktionen aus Spielsicht durchgeführt werden.\\
Betrüger empfangen jedoch die Statusupdates der anderen Spieler ohne selbst Updates zu senden. In der gewonnenen Zeit bis zum geforderten Update können nun die besten Züge berechnet werden und das Spiel entsprechend beeinflusst werden. Algorithmen die diese Problematik angehen werden entwickelt \cite{Aggarwal2005}.

Kabus \cite{Kabus2005Addressing} beschreibt grundsätzliche Techniken die in p2p-Systemen genutzt werden um Betrug aufzudecken bzw. zu verhinden. Zufällig gewählte Knoten können zu einem Konsenus beitragen und so die Spieleraktionen validieren und effektiv Betrug verhindern.\\
Eine weitere Möglichkeit der Betrusverhinderung ist das Einsetzen von \emph{trusted hardware} wie es bei Konsolenspielen häufig eingesetzt wird. Hier verhindert die Hardware Spielveränderungen oder den Start von externen Betrugsprogramme, die bsp. die Avatarsteuerung übernehmen. Dennoch besitzen auch solche Systeme Schwachstellen und können hintergangen werden.\\
Zur Betrugsaufdeckung führt Kabus die Prüfung von Logdateien an.

Auf weitere zahlreiche Arbeiten zur Betrugsverhinderung bzw. -aufdeckung wird verwiesen \cite{Ferretti2008Cheating, Gauthierdickey2004Low, Kabus2007Design}.


\section{Aufbau eines \ac{mmog}}
\label{chap:grundlagen:aufbau_mmog}



