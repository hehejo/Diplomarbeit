Grundlagen!
\section{Begriffsklärung}
In dieser Arbeit wird ein \emph{Publish/Subscribe-System über einem peer2peer overlay Netzwerk} beschrieben.

\paragraph{Overlay Netzwerk} Ein Overlay-Netzwerk ist ein logischer Aufsatz auf einem bestehenden Netzwerk und abstrahiert von diesem. Weiterhin zeichnen sie sich dadurch aus, dass ein eigener Adressraum genutzt wird und so das unterliegende Netzwerk überlagert wird. Knoten können so benachbart sein, ohne eine direkte physikalische Verbindung zu haben. Das Overlay-Netzwerk bietet neben dem eigenen Adressraum auch Funktionalität zum Versand, Empfang und Routing von Nachrichten. Eine genauere Beschreibung und Anforderungen an Overlay-Netzwerke sind in Kapitel \ref{chap:grundlagen:overlay} zu finden.

\missing{Irgendein Zitat (Buch)!}

\paragraph{Peer2Peer-Netzwerk} Peer2Peer-Netzwerke beschreiben den Verbund von Knoten die miteinander gleichberechtigt kommunizieren können. Peer2Peer-Netzwerke werden meist durch Overlay-Netzwerke realisiert, da Techniken wie bsp. IP-Multicast nicht weit verbreitet sind. Durch deren angebotenen Funktionen können viele Arten von Peer2Peer-Systemen implementiert werden. Grundlagen dieser Systeme sowie deren unterschiedliche Ansätze werden in Kapitel \ref{chap:grundlagen:overlay} näher besprochen.

\missing{Irgendein Zitat (Buch)!}

\paragraph{Publish/Subscribe-System} Ein Publish/Subscribe-System beschreibt ein Abosystem. Ein Subscriber schreibt sich bei einem Publisher ein und wird von diesem über Veränderungen benachrichtigt. Solch ein System kann kanalbasiert oder filterbasiert sein. Weitere Einzelheiten dieser Systeme werden in Kapitel \ref{chap:grundlagen:pubsub} besprochen.

\missing{Irgendein Zitat (Buch)!}

\section{p2p Overlay Netzwerke}
\label{chap:grundlagen:overlay}
structured peer-to-peer overlay vs unstructured peer-to-peer overlay

\subsection{Anforderungen}
\begin{itemize}
\item Eingriff in Routingentscheidungen
\item Bestimmung der Netztiefe (max. Anzahl an Hops zwischen zwei Knoten)
\item geringer Overhead (Latenz!)
\item Fehlertolerant bei Ausfall einzelner Knoten
\item Skalierbar
\item einfach Ordnung der Nachrichten
\end{itemize}

\subsection{generic API}
\cite{citeulike:6643572} %towards

\subsection{Pastry}
\cite{citeulike:780210} %pastry
Tapestry, Pastry, Chimera


\section{Publish/Subscribe-Systeme}

\subsection{Kanalbasiert}
\subsubsection{Scribe}
\cite{citeulike:345316} %Scribe


\subsection{Filterbasiert}
\cite{citeulike:854573}
\cite{citeulike:6674153} %Caguya
\cite{citeulike:4291} %Cluster

\subsection{Anforderungen}
\begin{itemize}
\item Stabil gegenüber \emph{churn}. D.h. häufige Wechsel der Mitgliedschaft
\item minimale Anzahl an Zombieknoten \footnote{Knoten die Nachrichten routen/transportieren müssen obwohl sie diese Nachricht nicht interesiert}
\item geringer Overhead (Latenz!)
\item 
\end{itemize}

