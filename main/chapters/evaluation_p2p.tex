Dieses Kapitel bietet einen Überblick über einige p2p-Netzwerke und evaluiert diese anhand gestellter Anforderungen. Diese Evaluierung beeinflußt die Entscheidung für ein Overlay-Netzwerk auf das schließlich, das in dieser Arbeit zu entwickelnde, generische Publish/Subscribe-System gesetzt wird. Die Evaluierung bedient sich zahlreicher Arbeiten, die sich alleine dem Vergleich dieser Netzwerke widmen \cite{Lua2005Survey, Goetz2005, Li2004Comparing, Darlagiannis2006Peertopeer, Castro2002Secure, Bo2003PeertoPeer} und geht auch auf ihre Nutzbarkeit als Basis für \emph{Application level multicast} sprich ein Publish/Subscribe-System ein \cite{Hosseini2007Survey, Fahmy2007, Castro2003Evaluation, Ratnasamy2001}.

Zuvor müssen jedoch die eigenen Anforderungen an solche Systeme identifiziert werden. Zu den offensichtlichen Anforderungen wie beispielsweise \emph{Skalierbarkeit} gesellen sich jedoch auch spezielle Anforderungen aus Spielsicht hinzu. Diese sind beispielsweise das Vorhandensein eines Masterservers oder das Übertragen von Applikationswissen auf das Netzwerk um damit dessen Entscheidungen bezüglich Nachbarschaften oder Versand von Nachrichten (Routing) zu beeinflussen.

\section{Anforderungen}

\cite{Darlagiannis2006Peertopeer} % Requirements

\paragraph{Geringe Latenz} Schnelle Reaktionszeiten und Nachrichtenübermittlung sind bei \ac{mmog} unverzichtbar. Ebenfalls müssen größere Nachrichten (beispielsweise Update der Welt) schnell übertragen werden damit der Spielfluss nicht behindert wird. Dies lässt sich anhand der Anzahl der Hops beim Nachrichtenversand messen.

\paragraph{Skalierbar} Selbst bei einer großen Anzahl an Knoten soll das Netz nicht kollabieren. Hierbei ist es auch wichtig, dass Knoten nicht unbedingt lange im Spiel sein müssen. Zwar kann davon ausgegangen werden, dass ein durchschnittlicher Spieler längere Zeit im Spiel verbringt, aber durch Netzausfälle oder sonstigen Unbill kann dies stark variieren.\\
Insofern ist es wichtig wie sich die Netzwerke bei großen Fluktationen verhalten \cite{Li2004Comparing}.

\paragraph{Fehlertoleranz bei Knotenausfall} Fallen Knoten aus, muss sich das Netz ohne großen Kommunikationsaufwand selbst reparieren. Ebenfalls mögliche Netzwerkpartitionierungen sind in dieser Arbeit jedoch kein Hindernis, denn der Hauptserver kann über eine gesonderte Verbindung immer erreicht werden. Damit kann das Netz wieder verbunden werden.\\
Interessant hierbei ist auch die eingebaute Redundanz einiger Systeme, die Daten auf mehrere Knoten verteilen. Wie sich dies im Vergleich von statischen Daten zu sich häufig verändernden Objekten der Spielewelt verhält ist zu untersuchen. 

\paragraph{Kommunikation über das Netzwerk} Das Netzwerk soll nicht nur das schnelle Auffinden von Peers ermöglichen, sondern auch einen Transport der Nachricht (Routing) durch das Netzwerk selbst bereitstellen. Die Alternative Direktverbindungen soll nur genutzt werden, wenn eine Datenübertragung im Netzwerk nicht performant genug ist, weil z.B. die Bandbreite der zwischengeschalteten Peers zu gering ist.

\paragraph{Bestimmung der Nachbarschaft} Eine dynamische Bestimmung der Nachbarschaftsgröße kann von Vorteil sein. So könnten Knoten mit mehr Bandbreite (bzw. entsprechenden anderen Metriken) mehr direkte Verbindungen halten als Knoten mit geringer Bandbreite (oder geringer Spieldauer).

\paragraph{Eingriff in Routingentscheidungen} Applikationswissen hilft auch beim Eingriff in das Routing des Netzes. So können Knoten bevorzugt zur Weiterleitung einer Nachricht ausgewählt werden. Diese Knoten zeichnen sich beispielsweise durch eine große Bandbreite oder spezielle Applikationsmetriken\footnote{Bsp: Spieler befindet sich in der selben Stadt} aus.

\paragraph{Verfügbarkeit als C/C++-Bibliothek} Da der Prototyp dieser Arbeit sowie das Umfeld in C++ entwickelt wird, gibt es damit eine weitere Anforderung an das Netzwerk: Die Verfügbarkeit als C/C++-Bibliothek.\\
Damit ist das Netzwerk einfach aus bestehendem Code nutzbar ohne dass kostenintensive Brücken zwischen beispielsweise Java und C++ geschlagen werden müssen. Da zudem betriebssystemübergreifend\footnote{In unserem Fall auf Windows und Linux} entwickelt und getestet wird, ist außerdem ein zur Verfügung stehender Quellcode vorteilhaft. Nur dadurch ist es möglich eventuellen betriebsystembezogenen Netzwerk- oder Threadcode auf  Boost\footnote{Zahlreiche Bibliotheken für C++: http://www.boost.org} zu portieren um somit eine betriebsystemübergreifende Benutzung zu ermöglichen.

\paragraph{Unterbau eines Publish/Subscribe-Systems} Das Overlay-Netzwerk muss so gestaltet sein, dass es dem Anwendungsfall \emph{Application level multicast} genügt und dessen besondere Anforderungen (auch von Spielseite)\footnote{siehe Kapitel \ref{chap:evaluation_pubsub}} her unterstützt.

Anhand dieser Anforderungen sind unstrukturierte Netzwerke nicht als Netzwerksystem für dieser Arbeit geeignet. Eine Suche, bzw. Datenübertragung durch Flooding widerspricht klar der Anforderung nach geringer Latenz. Ebenso sind diese Netzwerke auf eine Übertragung via Direktverbindung ausgelegt.

\section{generic API}
\label{chap:evaluation_p2p:generic_api}
\cite{Dabek2003Towards} %towards

\lstinputlisting[caption={Upcalls der generischen API}, label=code:towards_upcall]{listings/towards_upcall.cpp}



\section{Evaluation}


\cite{Lua2005} % Das Paper

\cite{Castro2003Evaluation} %evaluation

\cite{Fahmy2007} %characterizing

\cite{Rowstron2001} %pastry

\cite{Zhao2004Tapestry} %Tapestry Chimera

\cite{Malkhi2002Viceroy} % Viceroy

\cite{Bo2003PeertoPeer}
