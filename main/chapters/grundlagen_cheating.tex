\section{Betrug}\index{Betrug}
\label{chap:grundlagen:cheating}
Betrügerisches Verhalten (Cheating) ist in p2p-Systemen einfacher als in Client/Server-Systemen, da die Kommunikation und damit das Statusupdate sowie Entscheidungsfindungen nicht zwingend über einen (vom Betreiber kontrollierten) Masterserver laufen. Mit gefälschten Nachrichten können andere Knoten werden. Das Wissen über das zugrunde liegende System kann genutzt werden um gezielt wichtige Entscheidungen zu manipulieren.

Webb \cite{Webb2007Cheating} gibt eine Übersicht über Betrug in Netzwerkspielen und geht hier auch auf die Unterschiede zwischen Client/Server-Systemen und p2p-Systemen ein. Cheating wird in vier grundlegende Bereiche eingeteilt: \emph{Game level cheats}, \emph{Application level cheats}, \emph{Protocol level cheats} und \emph{Infrastructure level cheats}. Mögliche Verfahren wie \emph{Lockstep} \cite{Baughman2007}, das die Spielzeit in Runden einteilt werden vorgestellt und ausgewertet.

Als Beispiel für einen \emph{Protocol level cheat} wir das Ausnutzen des \emph{\index{Dead-Reckoning}}-Verfahren \cite{Pantel2002} beschrieben. Dead-Reckoning wird benutzt um die Latenz im Netzwerk zu kaschieren um anderen Spielern durch vorberechnete Aktionen einen konstanten Spielfluss zu bieten. Hierbei akzeptiert das System eine gewisse Anzahl an ausgefallenen/verlorenen Updates eines Knotens bevor unterschiedliche Aktionen aus Spielsicht durchgeführt werden.\\
Betrüger empfangen jedoch die Statusupdates der anderen Spieler ohne selbst Updates zu senden. In der gewonnenen Zeit bis zum geforderten Update können nun die besten Züge berechnet werden und das Spiel entsprechend beeinflusst werden. Algorithmen die diese Problematik angehen werden entwickelt \cite{Aggarwal2005}.

Kabus \cite{Kabus2005Addressing} beschreibt grundsätzliche Techniken die in p2p-Systemen genutzt werden um Betrug aufzudecken bzw. zu verhindern. Beispielsweise können für eine Konsensus\index{Betrug!Konsensus} über Spieleraktionen zufällig Knoten gewählt werden. Diese validieren die Aktionen und können so effektiv Betrug verhindern.\\
Eine weitere Möglichkeit der Betrugsverhinderung ist das Einsetzen von \emph{trusted hardware}\index{Betrug!trusted hardware} wie es bei Konsolenspielen häufig eingesetzt wird. Hier verhindert die Hardware Spielveränderungen oder den Start von externen Betrugsprogramme, die beispielsweise die Avatarsteuerung übernehmen. Dennoch besitzen auch solche Systeme Schwachstellen und können hintergangen werden.\\
Zur Betrugsaufdeckung führt Kabus die Prüfung von Logdateien an.

Auf weitere zahlreiche Arbeiten zur Betrugsverhinderung bzw. -aufdeckung wird verwiesen \cite{Ferretti2008Cheating, Gauthierdickey2004Low, Kabus2007Design, Dautermann2007, Kabus2009}.
