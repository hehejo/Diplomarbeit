\section{Betrug}\index{Betrug}
\label{chap:grundlagen:cheating}
Betrügerisches Verhalten (Cheating) ist in p2p-Systemen einfacher als in Client/Server-Systemen, da die Kommunikation nicht zwingend über einen (vom Betreiber kontrollierten) Server läuft. Mit gefälschten Nachrichten für Statusmeldungen, wie Positionsänderungen, oder zur Entscheidungsfindung über die Reihenfolge von Aktionen können andere Knoten massiv benachteiligt werden. Das Wissen über das zugrunde liegende System wird genutzt um gezielt wichtige Aspekte des Spieles zu manipulieren.

Webb \cite{Webb2007Cheating} gibt eine Übersicht über Betrug in Netzwerkspielen und geht hier auch auf die Unterschiede zwischen Client/Server-Systemen und p2p-Systemen ein. Cheating wird in vier grundlegende Bereiche eingeteilt: \emph{Game level cheats}, \emph{Application level cheats}, \emph{Protocol level cheats} und \emph{Infrastructure level cheats}. Mögliche Verfahren wie \emph{Lockstep} \cite{Baughman2007}, das die Spielzeit in Runden einteilt, werden vorgestellt und ausgewertet.

\paragraph{Beispiel}
Als Beispiel für einen \emph{Protocol level cheat} wird das Ausnutzen des \emph{Dead-Reckoning}-Verfahrens\index{Dead-Reckoning} \cite{Pantel2002} beschrieben. Dead-Reckoning kaschiert die Latenz im Netzwerk und bietet anderen Spielern durch vorberechnete Aktionen einen konstanten Spielfluss. Hierbei akzeptiert das System eine gewisse Anzahl an ausgefallenen/verlorenen Updates eines Knotens bevor unterschiedliche Aktionen aus Spielsicht durchgeführt werden.\\
Dieses Verfahren kann ausgenutzt werden, indem eigene Updates zurückhalten aber jedoch die Statusupdates der anderen Spieler auswerten werden. In der gewonnenen Zeit bis zum geforderten Update sind nun die besten Züge berechenbar und das Spiel kann somit beeinflusst werden. Algorithmen die diese Problematik angehen werden entwickelt \cite{Aggarwal2005}.

Kabus \cite{Kabus2005Addressing} beschreibt grundsätzliche Techniken die in p2p-Systemen genutzt werden um Betrug aufzudecken beziehungsweise zu verhindern. Beispielsweise können für eine Konsensus\index{Betrug!Konsensus} über Spieleraktionen zufällig Knoten gewählt werden. Diese validieren die Aktionen und können effektiv Betrug verhindern. Natürlich müssen solche Verfahren mit einem erhören Aufwand an Nachrichten teuer erkauft werden.\\
Eine weitere Möglichkeit der Betrugsverhinderung ist das Einsetzen von \emph{trusted hardware}\index{Betrug!trusted hardware} wie es bei Spielkonsolen häufig eingesetzt wird. Hier verhindert die Hardware Spielveränderungen oder den Start von externen Betrugsprogramme, die beispielsweise die Avatarsteuerung übernehmen. Dennoch besitzen auch solche Systeme Schwachstellen und können hintergangen werden.\\
Zur Betrugsaufdeckung führt Kabus die Prüfung von Logdateien an.

Auf weitere zahlreiche Arbeiten zur Betrugsverhinderung beziehungsweise -aufdeckung wird verwiesen \cite{Ferretti2008Cheating, Gauthierdickey2004Low, Kabus2007Design, Dautermann2007, Kabus2009, Castro2002Secure}.
