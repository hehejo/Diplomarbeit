\section{Vivaldi}
\label{chap:related:vivaldi}
Vivaldi \cite{citeulike:162250} ist ein dezentrales System, in dem sich Knoten des Netzwerkes Koordinaten so wählen, dass der Abstand zwischen zwei Koordinaten der Latenz zwischen diesen beiden Knoten entspricht.

\paragraph{Arbeitsweise}
Neue Knoten im Netz wählen sich zufällig Koordinaten. Zu allen Nachbarn werden nun Ping-Nachrichten zur Messung der Laufzeit verschickt. Die Koordinaten der einzelnen Knoten werden dabei übertragen. So kann nun die eigene Koordinate anhand des Abstandes angepasst werden.

Das Paper verwendet hier eine Analogie zu gespannten Federn. Alle Knoten sind über Federn miteinander verbunden und suchen nun das gemeinsame Optimum der Federspannung.
