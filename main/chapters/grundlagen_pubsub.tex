\section{Publish/Subscribe-Systeme}
\label{chap:grundlagen:pubsub}

\subsection{Kanalbasiert}
Ein prominenter Vertreter dieser Art ist Scribe \cite{citeulike:345316}, dessen Funktionsweise in Kapitel \ref{chap:related:scribe} genau beschrieben wird.

\subsection{Filterbasiert}
\label{chap:grundlagen:pubsub:filterbased}
\cite{citeulike:854573} %mercury
\cite{citeulike:6674153} %Caguya
\cite{citeulike:4291} %Cluster

\subsection{Anforderungen}
\begin{itemize}
\item Stabil gegenüber \emph{churn}, häufige Wechsel der Mitgliedschaft
\item minimale Anzahl an Zombieknoten \footnote{Knoten die Nachrichten routen/transportieren müssen obwohl sie diese Nachricht nicht interesiert}
\item geringer Overhead (Latenz!)
\item Skalierbar
\end{itemize}
